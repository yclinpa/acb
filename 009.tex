\section{Continuous functions}
\label{sec:continuous}

With the language of sequences in hand, it is now time to state the definition of continuous functions or mappings.

\begin{defn}
  Let $M, N$ be two metric spaces, and $f : M \to N$ be a function.
  The function $f$ is said to be \textsf{continuous} if it preserves sequential convergence: $f$ sends convergent sequences in $M$ to convergent sequences in $N$, limits being sent to limits.
  That is, for each sequence $\langle p_n \rangle$ in $M$ which converges to a limit $p$ in $M$, the image sequence $\langle f(p_n) \rangle$ converges to $f(p)$ in $N$.
\end{defn}

There are two obvious continuous functions.  Namely, the identity function $\operatorname{id}_M: M \to M$ and any constant function $f: M \to N$.
We omit their proofs here.

It is easy to see that the composite of two continuous functions are still continuous.

\begin{thm}
  Let $f: M \to N$ and $g: N \to P$ be continuous functions among three metric spaces $M, N$, and $P$.  Then their composition $g \circ f : M \to P$ is also continuous.
\end{thm}

\begin{proof}
  Let $\langle p_n \rangle$ be a sequence in $M$ that converges to $p$.
  Then $\langle f(p_n) \rangle$ converges to $f(p)$ in $N$, since $f$ is continuous.
  This in turn implies that $\langle g(f(p_n)) \rangle$ converges to $g(f(p))$ in $P$ since $g$ is also continuous, as required.
\end{proof}

The following definition of continuity sounds more (un-)familiar: it uses Greek letters.
We state it as a theorem, but really it is equivalent to what we just defined.

\begin{thm}
  Let $(M, d_M), (N, d_N)$ be metric spaces.
  A function $f: M \to N$ is continuous if and only if it satisfies the \textbf{$(\varepsilon,\delta)$-condition}: For each $\varepsilon > 0$ and each $p \in M$ there exists a $\delta > 0$ such that if $x \in M$ and $d_M(x,p) < \delta$ then $d_N\left( f(x), f(p) \right) < \varepsilon$.
\end{thm}

\begin{proof}
  For the forward direction, let us assume that $f$ preserves sequential limits.
  If the conclusion is false, then there are $\varepsilon > 0$ and $p \in M$ such that, for each $\delta > 0$, there is a point that violates the condition.
  Let us consider $\delta = 1, \frac12, \frac13, \dots, \frac1n, \dots$.
  For each $n \in \mathbb{N}$, let $x_n$ be some point in $M$ such that $d_M(x_n, p) < 1/n$ but $d_N( f(x_n), f(p) ) \geqslant \varepsilon$. 
  It is clear that $x_n \to p$ in $M$ but $f(x_n)$ cannot converge to $f(p)$, which is a contradiction.
  Therefore $f$ must satisfy the $(\varepsilon, \delta)$-condition.

  On the other hand, let us assume that $f$ satisfies the $(\varepsilon, \delta)$-condition.
  Take a sequence $\langle p_n \rangle$ in $M$ that converges to $p$ in $M$.
  For any $\varepsilon > 0$, there is a $\delta > 0$ (we have a point $p$ already) such that $d_N(f(x), f(p)) < \varepsilon$ whenever $x \in M$ and $d_M(x,p) < \delta$.
  For this $\delta > 0$, there is an integer $N \in \mathbb{N}$ such that $d_M( x_n, p ) < \delta$ whenever $n \geqslant N$.
  Piecing these informations together, we see that whenever $n \geqslant N$,
  \[
    d_M(x_n, p) < \delta \implies d_N \left( f(x_n), f(p) \right) < \varepsilon.
  \]
  This is exactly the definition for $\langle f(x_n) \rangle$ converges to $f(p)$ in $N$, 
  and we are done.
\end{proof}
A property of a metric space or of a mapping between metric spaces that can be described solely in terms of open sets (or equivalently, in terms of closed sets) is called a \textit{topological property}.
Our next result describes continuity of mappings topologically.

Firstly we introduce a notion.
Let $f : M \to N$ be a function/mapping.
For any $n \in N$, the \textsf{preimage} of $n$ under $f$ is the subset
\[
  f^{-1}(n) = \{ m \in M \colon f(m) = n \}
\]
of $M$.\footnote{Our textbook uses $f^{\text{pre}}$ instead of $f^{-1}$.  I assume that Pugh wanted to preserve the notation $f^{-1}$ to mean the inverse function of $f$.  Nevertheless we could regard $f^{-1}$ as a \textit{set-valued} mapping.}
Note that $f^{-1}(n) = \varnothing$ when $n$ is not in the image of $f$; also that $f$ is an injection if and only if $|f^{-1}(n)| \leqslant 1$.
We also use the notation
\[
  f^{-1}(A) = \{ m \in M \colon f(m) \in A \}
\]
for any subset $A \subseteq N$.
Here is our statement.

\begin{thm}
  The following are equivalent for continuity of $f : M \to N$.
  \begin{enumerate}[(i)]
    \item The $(\varepsilon,\delta)$-condition.
    \item The sequential convergence preservation condition.
    \item The \textbf{closed set condition}: The preimage $f^{-1}(C)$ of each closed set $C$ in $N$ is closed in $M$.
    \item The \textbf{open set condition}: The preimage $f^{-1}(U)$ of each open set $U$ in $N$ is open in $M$.
  \end{enumerate}
\end{thm}

\begin{proof}
  We have shown the equivalence (i) $\iff$ (ii).
  We hereby deal with other implications.

  \smallskip
  \noindent\textbf{(ii) $\implies$ (iii)}.
  Let $C$ be any closed set in $N$, and $p$ be a limit point of $f^{-1}(C)$.
  By definition there is a sequence $\langle a_n \rangle$ in $M$ such that converges to $p \in M$ and $f(a_n) \in C$ for all $n$.
  By (ii) $f$ preserves sequential convergence, hence $\langle f(a_n) \rangle$ converges to $f(p) \in N$.
  Since $f(a_n) \in C$ for all $n$ and $C$ is a closed set, we get $f(p) \in C$.
  This is saying that $p \in f^{-1}(C)$.
  Since $p$ is arbitrary, we see that $f^{-1}(C)$ contains all of its limit points, hence $f^{-1}(C)$ is a closed set in $M$.

  \smallskip
  \noindent\textbf{(iii) $\iff$ (iv)}.  This follows from Theorem 4 in Topic 8 and taking the complement: $$\left( f^{-1}(X) \right)^c = f^{-1}(X^c)$$ for any subset $X \subseteq N$.

  \smallskip
  \noindent\textbf{(iv) $\implies$ (i)}.
  Let $p$ be a point in $M$ and a positive number $\varepsilon > 0$ be given.
  Then the open ball $B_\varepsilon(f(p))$ is an open ball in $N$.
  By (iv) the set $U = f^{-1}(B_\varepsilon(f(p)))$ is open in $M$, and $p$ belongs to $U$.
  Since $U$ is open, $p$ is an interior point of $U$, i.e., there is a $\delta > 0$ such that $B_\delta(p) \subseteq U = f^{-1}(B_\varepsilon(f(p))$.
  This is saying that $f(B_\delta(p)) \subseteq B_\varepsilon(f(p))$.
  Translating this relation, it means that for any $x \in M$,
  \[
    d_M(x,p) < \delta \implies d_N(f(x),f(p)) < \varepsilon,
  \]
  which is exactly the $(\varepsilon, \delta)$-condition!

  \smallskip
  The proof is now complete.
\end{proof}

Note that no explicit mention is made of the metric in the closed and open set chararacterization for continuous functions.
The open set condition is purely topological.
It would be perfectly valid to take the open set characterization as a \textit{definition} of continuity; in fact this is how it is done in general topology.
With this, it is natural to introduce the following definition.

\begin{defn}
  Two metric spaces (or topological spaces) $M$ and $N$ are said to be \textsf{homeomorphic} if and only if there is a bijection $f: M \to N$ such that $f$ is continuous from $M$ to $N$, while $f^{-1}$ is continuous from $N$ to $M$.  (We could also say that the bijection $f$ is \textit{bicontinuous}.)  In this case $f$ is called a \textsf{homeomorphism} between $M$ and $N$. 
\end{defn}

In the mathematical terms, any word that ends in ``morphism'' usually means ``preserving the structure.''
A homeomorphism between $M$ and $N$ preserves the topological structures on both sides; it establishes a one-to-one correspondence between open sets in $M$ and open sets in $N$.
Two spaces that are homeomorphic to each other are really the same thing, but maybe under different names.
We see this phenomenon in other areas of mathematics.
For example, the same algebraic structures on two sets are called \textit{isomorphic}.

\medskip
\noindent\textbf{\large Subspace topology}

If a set $S$ is contained in a metric subspace $N \subseteq M$ you need to be careful when you say that $S$ is open or closed.
For example,
\[
  S = [0,1) = \{ x \in \mathbb{R} \colon 0 \leqslant x < 1 \}
\]
is not an open set in $\mathbb{R}$, but it is an open set in $[0,1]$, as we shall see.
The way to see this clearly is through the notion of \textit{inheritance}, or the so-called \textit{subspace topology}.

\begin{defn}
  Let $M$ be a metric space and $N$ be a subset of $M$.
  We say that $N$ \textsf{inherits its topology} from $M$ in the sense that each subset $V \subseteq N$ which is open in $N$ is actually the intersection $V = N \cap U$ for some $U \subseteq M$ that is open in $M$, and vice versa.
  In this situation we say that $N$ is equipped with the \textsf{subspace topology} from $M$.
\end{defn}

Here, the metric on $N$ is \textit{inherited} from $M$: we simply use the metric on $M$ for points on $N$.
Let us see why this notion of open sets in a subspace makes sense.
Indeed, for $p \in N \subseteq M$ and $r > 0$,
\[
  B_N(p,r) = \{ x \in N \colon d(x,p) < r \} = B_M(p,r) \cap N.\footnote{Here we use $B_N(p,r)$ to denote the open ball centered at $p$ with radius $r > 0$ in the metric space $N$.  I hope that this notation is self-evident.}
\]
Let $V$ be an open set in $N$.
For each $p \in N$, there is an $r_p > 0$ such that $B_N(p, r_p) \subseteq V$.
Set
\[
  U = \bigcup_{p \in V} B_M(p, r_p).
\]
It is clear that $U$ is an open set in $M$ and $V = U \cap N$.
The opposite direction is also clear.
The same can be said for closed sets in a subspace.
We state it for future reference if needed.

\begin{prop}
  Let $N$ be a subspace of a metric space $M$, and $N$ be equipped with the subspace topology.
  Then $C$ is a closed subset of $N$ if and only if there is a closed subset $K$ of $M$ such that $C = K \cap N$.
\end{prop}

\noindent\textbf{\large Product metric}

We next define a metric on the Cartesian product $M = X \times Y$ of two metric spaces $X$ and $Y$.
There are three natural ways to do so:
\begin{align*}
  d_E(p,p') &= \sqrt{ d_X(x,x')^2 + d_Y(y,y')^2 } \\
  d_{\operatorname{max}}(p,p') &= \max \{ d_X(x,x'), d_Y(y,y') \} \\
  d_{\operatorname{sum}}(p,p') &= d_X(x,x') + d_Y(y,y')
\end{align*}
where $p=(x,y)$, $p'=(x',y')$ belong to $M = X \times Y$.
The following result is immediate.

\begin{prop}
  \label{prop:product-metric}
  We have $d_{\operatorname{max}} \leqslant d_E \leqslant d_{\operatorname{sum}} \leqslant 2 d_{\operatorname{max}}$.
\end{prop}

In fact, the topologies induced by these three metrics on the product space are identical in the following sense.

\begin{thm}
  Let $U$ be a subset of a product space $M = X \times Y$ of two metric spaces.
  If $U$ is open under anyone of the three metrics, it is also open under the other two metrics.
\end{thm}

\begin{proof}
  Let us show one of the implications; others follow in a similar fashion.
  Suppose that $U$ is an open set under $d_E$.
  For any $p \in U$, there is an $r > 0$ such that $B_{E,r}(p) \subseteq U$ (the notation $B_{E,r}(p)$ is self clear.)
  Since $d_E \leqslant d_{\operatorname{sum}}$, we see that $d_E(p,p') < r$ whenever $d_{\operatorname{sum}}(p,p') < r$ for any $p' \in M$, i.e., $B_{\operatorname{sum},r}(p) \subseteq B_{E,r}(p)$.
  Therefore $p \in B_{\operatorname{sum},r}(p) \subseteq B_{E,r}(p) \subseteq U$ for that $r > 0$.
  Since $p$ is arbitrary, we conclude that $U$ is also an open set under $d_{\operatorname{sum}}$.
\end{proof}

From Proposition~5, we see that the notion of convergence in a product space prevails under these three different metrics, as the following theorem states; its proof is omitted here.

\begin{thm}
  \label{thm:product-convergence}
  The following are equivalent for a sequence $\langle p_n = (p_{1n}, p_{2n}) \rangle$ in $M = X \times Y$:
  \begin{enumerate}[(a)]
    \item $\langle p_n \rangle$ converges with respect to the metric $d_{\operatorname{max}}$.
    \item $\langle p_n \rangle$ converges with respect to the metric $d_{E}$.
    \item $\langle p_n \rangle$ converges with respect to the metric $d_{\operatorname{sum}}$.
    \item $\langle p_{1n} \rangle$ and $\langle p_{2n} \rangle$ converge in $X$ and $Y$, respectively.
    
    
  \end{enumerate}
\end{thm}

By repeating the products, the following result is useful and sometimes is forgotten when used.

\begin{cor}
  A sequence of vectors $\langle v_n \rangle$ in $\mathbb{R}^m$ converges in $\mathbb{R}^m$ if and only if each of its component sequences $\langle v_{in} \rangle_n$ converges, $1 \leqslant i \leqslant m$.
  The limit of the vector sequence is the vector
  \[
    v = \lim_{n \to \infty} v_n = \left( \lim_{n \to \infty} v_{1n}, \lim_{n \to \infty} v_{2n}, \dots, \lim_{n \to \infty} v_{mn}  \right).
  \]
\end{cor}

The distance function $d: M \times M \to \mathbb{R}$ is a function from the product space $M \times M$ to the metric space $\mathbb{R}$, so the following assertion makes sense.

\begin{thm}
  The distance function $d: M \times M \to \mathbb{R}$ is continuous.
\end{thm}

\begin{proof}
  We check $(\varepsilon, \delta)$-continuity with respect to the metric $d_{\operatorname{sum}}$ on $M \times M$.
  Given $\varepsilon > 0$ we take $\delta = \varepsilon$.
  If $d_{\operatorname{sum}}( (p,q), (p',q') ) < \delta$ then the triangle inequality gives
  \begin{align*}
    d(p,q)   &\leqslant d(p,p') + d(p',q') + d(q',q) < d(p',q') + \varepsilon, \\
    d(p',q') &\leqslant d(p',p) + d(p,q) + d(q,q') < d(p,q) + \varepsilon,
  \end{align*}
  which gives
  \[
    d(p,q) - \varepsilon < d(p',q') < d(p,q) + \varepsilon.
  \]
  Thus $|d(p',q') - d(p,q)| < \varepsilon$ and we get continuity with respect to the metric $d_{\operatorname{sum}}$.
  By Theorem~\ref{thm:product-convergence} it does not matter which metric we use on $M \times M$.
\end{proof}

\medskip
\noindent\textbf{\large Completeness}

In an earlier class we discussed the Cauchy criterion for convergence of a sequence of real numbers.
There is a natural way to carry these ideas over to a metric space $M$.
A sequence $\langle p_N \rangle$ in $M$ satisfies a \textsf{Cauchy condition} provided that for each $\varepsilon > 0$ there is an integer $N \in \mathbb{N}$ such that for all $k, n \geqslant N$ we have $d(p_k, p_n) < \varepsilon$, and $\langle p_n \rangle$ is said to be a \textsf{Cauchy sequence}.  In symbols,
\[
  \forall \varepsilon > 0 \,\, \exists N \text{ such that } k, n \geqslant N \implies d(p_k, p_n) < \varepsilon.
\]

Under this notion, all properties of Cauchy sequences in $\mathbb{R}$ carry to general metric spaces, except that a Cauchy sequence in a metric space may or may not converge.
In fact, we have the following definition.

\begin{defn}
  A metric space $(M,d)$ is \textsf{complete} if every Cauchy sequence in $M$ converges to a limit in $M$.
\end{defn}

Under this definition, $\mathbb{R}$ and $\mathbb{R}^m$ are complete metric spaces, while $\mathbb{Q}$ is not.
Now there is a nice property about closed subsets of such spaces.

\begin{thm}
  Every closed subset of a complete metric space is a complete metric space.
\end{thm}

\begin{proof}
  Let $A$ be a closed subset of the complete metric space $M$ and let $\langle p_n \rangle$ be a Cauchy sequence in $A$ with respect to the inherited metric.
  It is of course a Cauchy sequence in $M$ so it converges to a limit $p$ in $M$.
  Since $A$ is closed we have $p \in A$, i.e., the Cauchy sequence $\langle p_n \rangle$ converges to a point $p \in A$.
  This shows that $A$ is also complete.
\end{proof}

We note that completeness is \textit{not} a topological property.
For example, consider $\mathbb{R}$ with its usual metric and $(-1,1)$ with the metric it inherited from $\mathbb{R}$.
Although they are homeomorphic metric spaces, $\mathbb{R}$ is complete but $(-1,1)$ is not.

There is a general construction to complete a general metric space.
Please consult Chapter 2, Section 10 of Pugh's textbook.
