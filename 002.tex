\section{Cauchy sequences}
\label{sec:Cauchy}

There is a second sense in which $\mathbb R$ is complete.
It involves the concept of convergent sequences.
Let $a_1, a_2, a_3, \ldots = \langle a_n \rangle$, $n \in \mathbb N$, be a sequence of real numbers.

\begin{defn}
  The sequence $\langle a_n \rangle$ \textsf{converges to a limit} $b \in \mathbb R$ as $n \to \infty$ provided that for each $\varepsilon > 0$ there exists an integer $N \in \mathbb N$ such that for all $n \geqslant N$ we have
  \[
    |a_n - b| < \varepsilon \qquad (\text{or } b - \varepsilon < a_n < b + \varepsilon.)
  \]
\end{defn}
In symbols we may write
\[
  \forall\,\,\varepsilon > 0 \,\, \exists\,\, N \in \mathbb N \text{ such that } n \geqslant N \implies |a_n - b| < \varepsilon.
\]

If a limit $b$ exists it is not hard to see that it is unique, and we write
\[
  \lim_{n\to\infty} a_n = b \qquad \text{or} \qquad a_n \to b.
\]

There are a few basic mentioned in the elementary calculus.
We summarize as follows.


  \begin{prop}
    \label{prop:basicpropertyoflimit}
    The following properties hold.
    \begin{enumerate}[$(1)$]
      \item (Uniqueness) If $\displaystyle \lim_{n\to\infty} a_n = L$ and $\displaystyle \lim_{n\to\infty} a_n = M$, then $L=M$.
      \item If a sequence $\langle a_n \rangle$ converges, then any of its subsequence converges to the same limit.
      \item (Arithmetics) If $a_n \to A$ and $b_n \to B$, then $a_n \odot b_n \to A \odot B$, where $\odot \in \{ +, -, \times, \div \}$.  That $B \ne 0$ is required for $\odot = \div$.
      \item (Squeeze theorem) Let $\langle x_n \rangle$, $\langle a_n \rangle$, $\langle b_n \rangle$ be three number sequences such that $a_n \leqslant x_n \leqslant b_n$ for all sufficiently large $n$.  If $a_n \to L$ and $b_n \to L$, then $x_n \to L$ as well.
      \item (Comparison theorem) Let $\langle a_n \rangle$ and $\langle b_n \rangle$ be two convergent sequences such that $a_n \leqslant b_n$ for all sufficiently large $n$.  Then $\displaystyle \lim_{n\to\infty} a_n \leqslant \lim_{n\to\infty} b_n$.
    \end{enumerate}
  \end{prop}

Below are a few basic limits.
\begin{thm}
  \begin{enumerate}[$(1)$]
    \item $\displaystyle\lim_{n\to\infty} \frac{1}{n} = 0$.
    \item For any $a > 0$, $\displaystyle \lim_{n\to\infty} a^{1/n} = 1$.
    \item $\displaystyle \lim_{n\to\infty} n^{1/n} = 1$.
    \item For any $p > 0$ and $\alpha \in \mathbb{R}$, $\displaystyle \lim_{n\to\infty} \frac{n^\alpha}{(1+p)^n} = 0$.
  \end{enumerate}
\end{thm}

\begin{proof}
  \begin{enumerate}[(1)]
    \item For any $\varepsilon > 0$, there is an integer $N \in \mathbb{N}$ such that $N \varepsilon > 1$, by the Archimedean property.
      Hence for any $n \geqslant N$, we have
      \[
	0 < \frac{1}{n} \leqslant \frac{1}{N} < \varepsilon,
      \]
      as required.
    \item There is nothing more to do for $a = 1$.
      Let us first assume that $a > 1$.
      Define a sequence $\langle x_n \rangle$ by $x_n = a^{1/n} - 1$, that is, $a = (1 + x_n)^n$ for each $n \in \mathbb{N}$.
      Using the binomial theorem, we see that
      \begin{align*}
	a &= (1 + x_n)^n \geqslant 1 + n x_n > n x_n, \\
	0 &< x_n < \frac{a}{n} \to 0 \qquad \text{as $n \to \infty$},
      \end{align*}
      therefore $\displaystyle \lim_{n \to \infty} a^{1/n} = \lim_{n \to \infty} (1 + x_n) = 1 + 0 = 1$.

      For $0 < a < 1$, we use the arithmetic property for limit to see that
      \[
	\lim_{n \to \infty} a^{1/n} = \lim_{n\to\infty} \frac{1}{(1/a)^{1/n}} = \frac{1}{1} = 1,
      \]
      since $1/a > 1$.  
      So this case is also done.

    \item Define a positive sequence $\langle y_n \rangle$ by $y_n = n^{1/n} - 1$, that is, $1 + y_n = n^{1/n}$ for each $n \in \mathbb{N}$.  Using the binomial theorem again, we see that for $n \geqslant 2$,
      \begin{align*}
	n &= (1 + y_n)^n \geqslant 1 + n y_n + \frac{n(n-1)}{2} y_n^2 > \frac{n(n-1)}{2} y_n^2, \\
	0 &< y_n^2 < \frac{2}{n-1} \to 0 \qquad \text{as $n \to \infty$}.
\end{align*}
	Since $\langle y_n^2 \rangle \to 0$, so does $\langle y_n \rangle$.

      \item The statement is clear when $\alpha \leqslant 0$. 
	For the case $\alpha > 0$, first find an integer $k \in \mathbb{N}$ that exceeds $\alpha$.
	Then for $n > k$, we have
	\begin{align*}
	  \frac{n^\alpha}{(1+p)^n} &\leqslant \frac{n^k}{\binom{n}{k+1} p^{k+1}} \to 0 \qquad
	  \text{as $n \to \infty$},
	\end{align*}
	since the denominator of the last fraction is a polynomial of degree $k+1$ in $n$.
  \end{enumerate}
\end{proof}

Note that the limit $b$ may be extraneous to the sequence $\langle a_n \rangle$.
A related concept is the following.

\begin{defn}
  A sequence $\langle a_n \rangle$ is called a \textsf{Cauchy sequence} (or obeys a Cauchy condition) if for each $\varepsilon > 0$ there is an integer $N \in \mathbb N$ such that for all $n, k \geqslant N$ we have
  \[
    |a_n - a_k| < \varepsilon.
  \]
\end{defn}

It is an easy exercise to show that a convergent sequence obeys a Cauchy condition.
The converse of the fact is a fundamental property of $\mathbb{R}$, as stated below.

\begin{thm}
  $\mathbb{R}$ is \textsf{complete} with respect to Cauchy sequences in the sense that if $\langle a_n \rangle$ is a sequence of real numbers which obeys a Cauchy condition then it converges to a limit in $\mathbb{R}$.
\end{thm}

\begin{proof}
  First we show that $\langle a_n \rangle$ is bounded.
  Taking $\varepsilon = 1$ in the Cauchy condition implies that there is an $N \in \mathbb{N}$ such that for all $n, k \geqslant N$ we have $|a_n - a_k| < 1$.
  Take $K$ large enough that $-K \leqslant a_1, \dots, a_N \leqslant K$.
  Set $M = K + 1$.  Then for all $n$ we have
  \[
    -M < a_n < M,
  \]
  which shows that the sequence $\langle a_n \rangle$ is bounded.

  Define a set $X$ as 
  \[
    X = \{ x \in \mathbb{R} \colon \exists\,\, \text{infinitely many $n$ such that } a_n \geqslant x \}.
  \]
  $-M \in X$ since for all $n$ we have $a_n > -M$, which $M \notin X$ since no $x_n \geqslant M$.  Thus $X$ is a nonempty subset of $\mathbb{R}$ which is bounded above by $M$.  The least upper bound property applies to $X$ and we have $b = \sup X$ with $-M \leqslant b \leqslant M$.

  We claim that $\langle a_n \rangle$ converges to $b$ as $n \to \infty$.
  Given $\varepsilon > 0$ we must show that there is an $N$ such that for all $n \geqslant N$ we have $|a_n - b| < \varepsilon$.
  Since $\langle a_n \rangle$ is a Cauchy sequence and $\varepsilon/2$ is positive there does exist an $N$ such that if $n, k \geqslant N$ then
  \[
    |a_n - a_k| < \frac{\varepsilon}{2}.
  \]
  Since $b - \varepsilon/2$ is less than $b$ it is not an upper bound for $X$, so there is an $x \in X$ with $b - \varepsilon/2 \leqslant x$.
  For infinitely many $n$ we have $a_n \geqslant x$.
  Since $b + \varepsilon/2 > b$ it does not belong to $X$, and therefore for only finitely many $n$ do we have $a_n > b + \varepsilon/2$.
  Thus, for infinitely many we have
  \[
    b - \frac{\varepsilon}{2} \leqslant x \leqslant a_n \leqslant b + \frac{\varepsilon}{2}.
  \]
  Since there are infinitely many of these $n$ there are infinitely many that are at least $N$.  Pick one, say $a_{n_0}$ with $n_0 \geqslant N$ and $b - \frac{\varepsilon}{2} \leqslant a_{n_0} \leqslant b + \frac{\varepsilon}{2}$.  Then for all $n \geqslant N$ we have
  \[
    |a_n - b| \leqslant |a_n - a_{n_0}| + |a_{n_0} - b| < \frac{\varepsilon}{2} + \frac{\varepsilon}{2} = \varepsilon,
  \]
  which completes the verification that $\langle a_n \rangle$ converges to $b$.
\end{proof}

Let us summarize as follows.
\begin{thm}
  A real sequence $\langle a_n \rangle$ converges if and only if it is a Cauchy sequence.
\end{thm}

Not every real sequence converges.  However, the following are available for every real sequence.

\begin{defn}
  Let $\langle a_n \rangle$ be a real sequence.  Define the \textsf{limit supremum} and the \textsf{limit infinum} for $\langle a_n \rangle$ as follows.
  \begin{align*}
    \limsup_{n\to\infty} a_n &= \lim_{n\to\infty} \left( \sup_{k\geqslant n} a_k \right); \\
    \liminf_{n\to\infty} a_n &= \lim_{n\to\infty} \left( \inf_{k\geqslant n} a_k \right).
  \end{align*}
\end{defn}

Since $\displaystyle \sup_{k \geqslant n} a_k$ is decreasing as $n$ gets larger and larger, $\displaystyle\limsup_{n\to\infty} a_n$ always exists in $\overline{\mathbb{R}}$; same for $\displaystyle\liminf_{n\to\infty} a_n$.
We state important properties about limit supremum; the analogues hold for limit infinum.

\begin{thm}
  \label{thm:limsup}
  Let $\langle a_n \rangle$ be a real sequence and $\displaystyle S = \limsup_{n\to\infty} a_n$.
  \begin{enumerate}[$(1)$]
    \item If $S = +\infty$, then $\langle a_n \rangle$ is not bounded above.
    \item If $S = -\infty$, then $\displaystyle \lim_{n\to\infty} a_n = -\infty$.
    \item If $S \in \mathbb{R}$, then for each $\varepsilon > 0$,
      \begin{enumerate}[$(i)$]
	\item there is an $N \in \mathbb{N}$ such that for every $n \geqslant N$ we have $a_n < S + \varepsilon$; 
	\item there are infinitely many terms $a_n$ with $a_n > S - \varepsilon$.
      \end{enumerate}
  \end{enumerate}
\end{thm}

\begin{proof}
Set $A_n = \displaystyle \sup_{k \geqslant n} a_k$; as noted before, $\langle A_n \rangle$ is a decreasing sequence in $\overline{\mathbb{R}}$.  Let us discuss the three cases separately.
  \begin{enumerate}[(1)]
    \item If $S = +\infty$, then $A_n$ must be $+\infty$ for all $n$ since $\langle A_n \rangle$ is decreasing.  In particular $\sup \{ a_k \colon k \geqslant 1 \} = A_1 = +\infty$, which is equivalent of saying that $\langle a_n \rangle$ is not bounded above.
    \item Assume $S = -\infty$.  For every $M \in \mathbb{R}$ there is an integer $N \in ~\mathbb{N}$ such that $A_n \leqslant M$ for all $n \geqslant N$.  Since $a_k \leqslant A_n$ for any $k \geqslant n$, the same can be said to $a_n$.  This shows that $a_n \to -\infty$ as $n \to \infty$.
    \item Now assume $S \in \mathbb{R}$ and a positive number $\varepsilon > 0$ is given.  There is an integer $n \in \mathbb{N}$ such that $S - \varepsilon < A_n < S + \varepsilon$ for all $n \geqslant N$.  Firstly that $a_n \leqslant A_N < S + \varepsilon$ for all $n \geqslant N$ shows (i).  As for part (ii), note that for each $n \geqslant N$ there is some $k \geqslant n$ such that $a_k > S - \varepsilon$ since the latter is smaller than the least upper bound $A_n$ of the set $\{ a_k \colon k \geqslant n \}$.  This means that we can always find some term that is larger than $S - \varepsilon$ beyond any point, hence part (ii) is shown.
  \end{enumerate}
\end{proof}

Finally we list a few properties for upper and lower limits.

\begin{prop}
  Let $\langle x_n \rangle$ and $\langle y_n \rangle$ be two real sequences.
  \begin{enumerate}[$(1)$]
    \item Let $\displaystyle S = \limsup_{n \to \infty} x_n$.  Then there is a subsequence $\langle x_{n_i} \rangle$ of $\langle x_n \rangle$ such that $x_{n_i} \to S$ as $i \to \infty$.

    \item If $T$ is a subsequential limit of $\langle x_n \rangle$, then $\displaystyle T \leqslant \limsup_{n \to \infty} x_n$.

    \item We have
      \[
	\limsup_{n \to \infty} (x_n + y_n) \leqslant \left( \limsup_{n \to \infty} x_n \right) + \left( \limsup_{n \to \infty} y_n \right),
      \]
      except the case where the right-hand side is $\infty + (-\infty)$, whose result is undefined.

      Furthermore, there are examples where the strict inequality $<$ actually happen.

    \item If $\displaystyle \lim_{n\to\infty} y_n$ exists in $[-\infty, \infty]$, then
      \[
	\limsup_{n \to \infty} (x_n + y_n) = \left( \limsup_{n \to \infty} x_n \right) + \left( \lim_{n \to \infty} y_n \right),
      \]
      except the case where the right-hand side is $\infty + (-\infty)$, whose result is undefined.
      
    \item $\displaystyle \lim_{n \to \infty} x_n$ exists in $[-\infty, \infty]$ if and only if $\displaystyle \limsup_{n \to \infty} x_n = \liminf_{n \to \infty} x_n$.

  \end{enumerate}
\end{prop}

\begin{proof}
  \begin{enumerate}[(1)]
    \item There are three cases to consider: $S= \infty$, $S = - \infty$, or $S \in \mathbb{R}$.

      If $S = \infty$, by Theorem~\ref{thm:limsup} (1) we know that $\langle x_n \rangle$ is not bounded above.  Hence we may find an increasing sequence $n_1 < n_2 < n_3 < \cdots$ of positive integers such that $x_{n_i} \geqslant i$ for all $i \in \mathbb{N}$.  It is clear that this subsequence works, i.e., $x_{n_i} \to \infty$ as $i \to \infty$.

      If $S = -\infty$, then by Theorem~\ref{thm:limsup} (2) we know that $\displaystyle \lim_{n \to \infty} x_n = -\infty$.  Obviously the statement holds because a sequence is a subsequence of itself.

      Now we assume that $S \in \mathbb{R}$.  For convenience we set $n_0 = 0$.  By Theorem~\ref{thm:limsup} (3), for each $i \in \mathbb{N}$, there is an integer $n_i \in \mathbb{N}$ such that $n_i > n_{i-1}$ and $S - \frac{1}{i} < x_{n_i} < S + \frac{1}{i}$ (considering $\varepsilon = \frac{1}{i}$).  By the squeeze theorem we see that $\displaystyle \lim_{i \to \infty} x_{n_i} = S$.

    \item Let $\langle x_{n_i} \rangle$ be a subsequence of $\langle x_n \rangle$ such that $x_{n_i} \to T$ as $i \to \infty$.  Since for each $i \in \mathbb{N}$,
      \[
	x_{n_i} \leqslant \alpha_{n_i} = \sup \{ x_k \colon k \geqslant n_i \},
      \]
      hence by the limit comparison theorem
      \[
	T = \lim_{n \to \infty} x_{n_i} \leqslant \lim_{i \to \infty} \alpha_{n_i} = S.
      \]

    \item Define for each $n \in \mathbb{N}$,
      \[
	\alpha_n = \sup \{ x_k \colon k \geqslant n \}, \qquad
	\beta_n  = \sup \{ y_k \colon k \geqslant n \}.
      \]
      Then for any $k \geqslant n$, we have
      \[
	x_k + y_k \leqslant \alpha_n + \beta_n.
      \]
      This implies that
      \[
	\sup \{ x_k + y_k \colon k \geqslant n \} \leqslant \alpha_n + \beta_n.
      \]
      Hence by the limit comparison theorem,
      \[
	\limsup_{n \to \infty} (x_n + y_n) \leqslant \lim_{n \to \infty} \alpha_n + \lim_{n \to \infty} \beta_n = \limsup_{n \to \infty} x_n + \limsup_{n \to \infty} y_n.
      \]
      The exceptional case of $\infty + (-\infty)$ (or $(-\infty) + \infty$) on the right-hand side can be seen easily.

    \item We follow the proof of (3) and the notations in it.  Note that if at least one of the limits $\displaystyle \limsup_{n\to\infty} x_n$ and $\displaystyle \lim_{n\to\infty} y_n$ is infinite, the result is clear.  Therefore we assume that $\displaystyle \limsup_{n\to\infty} x_n = A$ and $\displaystyle \lim_{n\to\infty} y_n = B$ are finite, and we are proving the reverse inequality of (3).

      For any $\varepsilon > 0$, there are infinitely many $n$'s such that $x_n \geqslant A - \dfrac{\varepsilon}{2}$.  On the other hand, there is an integer $N \in \mathbb{N}$ such that $y_n \geqslant B - \dfrac{\varepsilon}{2}$ for all $n \geqslant N$.  Therefore there are infinitely many $n$'s such that $x_n + y_n \geqslant A + B - \varepsilon$.  This implies that
      \[
	\limsup_{n\to\infty} (x_n + y_n) \geqslant A + B - \varepsilon.
      \]
      Since $\varepsilon > 0$ is arbitrary, we conclude that
      \[
	\limsup_{n\to\infty} (x_n + y_n) \geqslant A + B = \limsup_{n\to\infty} x_n + \lim_{n\to\infty} y_n.
      \]

    \item If any of the three numbers is $\infty$ or $-\infty$, then Theorem~\ref{thm:limsup} (2) takes care the argument.

      Now assume $\displaystyle S = \lim_{n\to\infty} x_n$ is a real numbers.  Then part (1) of this Proposition shows that $\displaystyle \limsup_{n\to\infty} x_n = \liminf_{n\to\infty} x_n = S$ by Proposition~\ref{prop:basicpropertyoflimit} (2).

      Conversely, assume that $\displaystyle S = \limsup_{n\to\infty} x_n = \liminf_{n\to\infty} x_n$.  By Theorem~\ref{thm:limsup} (3)(i), for any $\varepsilon > 0$, there is an integer $N \in \mathbb{N}$ such that $S - \varepsilon < x_n < S + \varepsilon$ whenever $n \geqslant N$.  This is precisely the definition of $\displaystyle \lim_{n\to\infty} x_n = S$.
  \end{enumerate}
\end{proof}
