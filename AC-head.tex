\documentclass[12pt,openany]{book}
\usepackage[margin=1in]{geometry}
\usepackage{amsmath, latexsym, amssymb, amsthm}
\usepackage{junicode}
\usepackage{unicode-math}
\setmathfont{Libertinus Math}
\usepackage{enumerate}
\usepackage{emptypage}
\usepackage{dutchcal}
\usepackage{tikz}
\usetikzlibrary{arrows.meta}
%%%%% Header stuff %%%%%
\setlength{\headheight}{15pt}
\usepackage{fancyhdr}
\fancyfoot{}
\fancyhead[L]{}
\fancyhead[R]{\thepage}
\renewcommand{\headrulewidth}{0.4pt}
\pagestyle{fancy}
%%%%% End Header   %%%%%
\setlength{\parskip}{0.5ex}
\newtheorem{thm}{Theorem}[chapter]
\newtheorem{prop}[thm]{Propsition}
\newtheorem{cor}[thm]{Corollary}
\newtheorem{lem}[thm]{Lemma}
\newtheorem{property}{Property}
\theoremstyle{definition}
\newtheorem{defn}[thm]{Definition}
\newtheorem{example}[thm]{Example}
%%%%%
\newcommand{\op}[1]{\operatorname{#1}}
\newcommand{\eu}{\mathrm e}
\newcommand{\dd}{\mathrm d}

\title{Rudimentary Advanced Calculus}
\author{Yen-chi Roger Lin}
\date{August 2024}

\begin{document}
  \frontmatter
  \maketitle

  \setcounter{page}{1}
  \tableofcontents

  \mainmatter
  \chapter{The Real Number System}
\label{chap:real-number}

\section{Construction of the Real Number System}
We first construct the number systems: $\mathbb N \to \mathbb Z \to \mathbb Q$.
Assume the usual arithmetic rules and ordering on $\mathbb Q$.

\begin{thm}
  No positive number $r \in \mathbb Q$ has square equal to $2$; i.e., $\sqrt{2} \notin \mathbb Q$.
\end{thm}

This fact was shown in high school and we will omit it here.

\begin{defn}
  A \textsf{cut} in $\mathbb Q$ is a pair of subsets $A, B$ of $\mathbb Q$ such that
  \begin{enumerate}[(a)]
    \item $A \cup B = \mathbb Q$, $A \ne \varnothing$, $B \ne \varnothing$, $A \cap B = \varnothing$ ($A$ and $B$ form a partition of $\mathbb Q$.)
    \item If $a \in A$ and $b \in B$ then $a < b$.
    \item $A$ contains no largest element.
  \end{enumerate}
\end{defn}

$A$ is the left-hand part of the cut and $B$ is the right-hand part.
We denote the cut as $x = A \mid B$. 

\begin{example}
  Let $A = \{ r \in \mathbb Q \colon r \leqslant 0 \text{ or } r^2 < 2 \}$ and $B = \mathbb Q \setminus A$.  We show that $A$ does not have the largest element and hence $A \mid B$ is a legitimate cut.

  Suppose $p \in A$ and $p > 0$.  By definition $p \in \mathbb{Q}$ and $p^2 < 2$.
  Consider $q = \dfrac{2p+2}{p+2} \in \mathbb{Q}$.  Below we show that $q \in A$ and $p < q$.

  To check whether $q \in A$ or not, we compute
  \begin{equation*}
    q^2 - 2 = \left( \frac{2p+2}{p+2} \right)^2 - 2 = \frac{(2p+2)^2-2(p+2)^2}{(p+2)^2} = \frac{2p^2 - 4}{(p+2)^2} < 0,
  \end{equation*}
  therefore $q \in A$.  On the other hand,
  \begin{equation*}
    q - p = \frac{2p+2}{p+2} - p = \frac{2p+2 - p(p+2)}{p+2} = \frac{2-p^2}{p+2} > 0,
  \end{equation*}
  as desired.
\end{example}

\begin{defn}
  A \textsf{real number} is a cut in $\mathbb Q$.
  The collection of all real numbers is denoted by $\mathbb R$.
\end{defn}

There is a natural way to embed $\mathbb Q$ into $\mathbb R$.
For any $c \in \mathbb Q$, let $A$ be the set of all rationals that are less than $c$ while $B$ be the rest of $\mathbb Q$.
We denote this cut by $c^*$.
Obviously $c \mapsto c^*$ gives an embedding of $\mathbb Q$ into $\mathbb R$.
Also such a cut is convenient to call it a \textsf{rational cut}: it is a cut where the $B$-part has the smallest element.

\begin{defn}
  Let $x = A \mid B$ and $y = C \mid D$ be cuts.
  If $A \subseteq C$ then $x$ is \textsf{less than or equal to} $y$ and we write $x \leqslant y$.
  If $A \subseteq C$ but $A \neq C$ then $x$ is \textsf{less than} $y$ and we write $x < y$.
\end{defn}

The property that distinguishes $\mathbb R$ from $\mathbb Q$ and is at the bottom of every significant theorem about $\mathbb R$ involves upper bounds and least upper bounds or, equivalently, lower bounds and greatest lower bounds.

\begin{defn}
  A real number $M \in \mathbb R$ is called an \textsf{upper bound} for a set $S \subseteq \mathbb R$ if each $s \in S$ satisfies $s \leqslant M$.
\end{defn}

We also say that the set $S$ is \textsf{bounded above} by $M$.
Note that every real number is automatically an upper bound for the empty set $\varnothing$.
An upper bound for $S$ that is less than all other upper bounds for $S$ is the \textsf{least upper bound} for $S$, which is denoted by $\operatorname{sup} S$, if it exists.

\begin{thm}
  \label{thm:lubp}
  The set $\mathbb R$, constructed by means of Dedekind cuts, is \textsf{complete} in the sense that is satisfies the
  \begin{quote}
    \textsf{Least Upper Bound Property}: If $S$ is a nonempty subset of $\mathbb R$ and is bounded above then in $\mathbb R$ there exists a least upper bound for $S$.
  \end{quote}
\end{thm}

\begin{proof}
  Let $\mathcal C \subseteq \mathbb R$ be any nonempty collection of cuts which is bounded above, say by the cut $X \mid Y$.  Define
  \[
    C = \{ a \in \mathbb Q \colon \text{ for some cut $A \mid B \in \mathcal C$ we have $a \in A$ } \}
  \]
  and $D = \mathbb Q \setminus C$.
  It is easy to see that $z = C \mid D$ is a cut, and $C \subseteq X$.
  Clearly, it is an upper bound for $\mathcal C$ since for every cut $A \mid B \in \mathcal C$ we have $A \subseteq C$.
  On the other hand, let $z' = C' \mid D'$ be any upper bound for $\mathcal C$.
  By assumption $A \mid B \leqslant C' \mid D'$ for all $A \mid B \in \mathcal C$.
  From definition $A \subseteq C'$ for every $A \mid B \in \mathcal C$, hence $C \subseteq C'$.
  Therefore $z = C \mid D \leqslant C' \mid D' = z'$, which shows that $z$ is the least upper bound for $\mathcal C$.
\end{proof}

Now we need to establish arithmetics on $\mathbb R$, the cuts in $\mathbb Q$.
Let cuts $x = A \mid B$ and $y = C \mid D$ be given.
We define their sum $x + y = E \mid F$ as follows: let
\begin{align*}
  E &= \{ r \in \mathbb Q \colon r = a + c \text{ for some $a \in A$ and $c \in C$} \}, \\
  F &= \mathbb Q \setminus E.
\end{align*}
It is easy to see that $E \mid F$ is a cut in $\mathbb Q$ and that it does not depend on the order in which $x$ and $y$ appear.
That is, cut addition is well defined and $x + y = y + x$.
The zero cut, which is the identity for addition, is $0^*$ and $0^* + x = x$ for all $\mathbb R$.
It is a little bit tricky to define the additive inverse $-x = C \mid D$ for $x = A \mid B$: let
\begin{align*}
  C &= \{ r \in \mathbb Q \colon r = -b \text{ for some $b \in B$ which is not the least element of $B$} \}, \\
  D &= \mathbb Q \setminus C.
\end{align*}
One checks that this is the correct additive inverse for $A \mid B$, i.e., $x + (-x) = 0^*$.
Corrspondingly, the difference of cuts is $x - y = x + (-y)$.
Also we need to check the associativity: $(x+y)+z = x+(y+z)$ for cuts $x,y,z$.
After $0^*$ is defined, it is conventional to call a real number $x$ \textsf{positive} (resp.\ \textsf{negative}) if $x > 0^*$ (resp.\ $x < 0^*$).
This is needed when we try to define the multiplication on $\mathbb R$.

\noindent\textbf{Exercise.} Define the multiplication on $\mathbb R$.  Name the multiplicative indentity.  Find the multiplicative inverse of a non-zero number $x \in \mathbb R$.

The ordering on $\mathbb R$ enjoys \textsf{transitivity}, \textsf{trichotomy}, and \textsf{translation}.
The ultimate result is stated as follows.
\begin{thm}
  The set $\mathbb R$ of all cuts in $\mathbb Q$ is a complete ordered field that contains $\mathbb Q$ as an ordered subfield.
\end{thm}

The \textsf{magnitude} or absolute value of $x \in \mathbb R$ is
\[
  |x| = \begin{cases}
    x, & \text{if $x \geqslant 0$}, \\
    -x, & \text{if $x < 0$}.
  \end{cases}
\]

Thus, $-|x| \leqslant x \leqslant |x|$.
A basic, constantly used fact about magnitude is the following.

\begin{thm}[Triangle inequality]
  For all $x,y \in \mathbb R$ we have $|x+y| \leqslant |x| + |y|$.
\end{thm}

Finally, can we enlarge $\mathbb R$ by taking the cuts in $\mathbb R$?
Let us say that $\mathcal A \mid \mathcal B$ is a cut in $\mathbb R$, that is, $\mathcal A$ and $\mathcal B$ are disjoint, nonempty subsets of $\mathbb R$ such that $a < b$ whenever $a \in \mathcal A$ and $b \in \mathcal B$, and $\mathcal A$ contains no largest element.
Clearly every element in $\mathcal B$ is an upper bound for $\mathcal A$.
By Theorem~\ref{thm:lubp}, $y = \sup \mathcal A$ exists in $\mathbb R$ and $a \leqslant y \leqslant b$ for each $a \in \mathcal A$ and $b \in \mathcal B$.
By trichotomy,
\[
  \mathcal A \mid \mathcal B = \{ x \in \mathbb R \colon x < y \} \mid \{ x \in \mathbb R \colon x \geqslant y \}.
\]
In other words, $\mathbb R$ has no gaps.  Every cut in $\mathbb R$ occurs exactly at a real number.

Finally, we say a few words about infinity.
$\pm \infty$ are not real numbers, since neither $\mathbb Q \mid \varnothing$ nor $\varnothing \mid \mathbb Q$ are cuts.
Although some mathematicians think of $\mathbb R$ together with $-\infty$ and $\infty$ as an ``extended real number system,'' which is usually denoted by $\overline{\mathbb{R}}$, it is simpler to leave well enough alone and just deal with $\mathbb R$ itself.  Nevertheless, it is convenient to write expressions like ``$x \to \infty$'' to indicate that a real variable $x$ grows larger and larger without bound.
When a nonempty subset $S$ is not bounded above (resp.\ not bounded below) we say that $\sup S = +\infty$ (resp.\ $\inf S = -\infty$).

We name a few further properties about $\mathbb{R}$ and $\mathbb{Q}$.

\begin{thm}
  Every interval $(a,b)$, no matter how small, contains both rational and irrational numbers.
  In fact it contains infinitely many rationals and infinitely many irrationals.
\end{thm}

\begin{proof}
  Think of $a,b$ as cuts $a = A \mid A'$, $b = B \mid B'$.  The relation $a < b$ implies that $B \setminus A$ is a nonempty set of rational numbers; choose a rational $r \in B \setminus A$.  Since $B$ has no largest element, there is another rational $s$ with $a < r < s < b$.
  Now for each $t \in [0,1]$, $r + t (s-r)$ is always a rational between $a$ and $b$.

  For the irrationals, one uses the fact that $\sqrt{2}$ is an irrational and perform a similar trick.
\end{proof}

A somewhat obscure and trivial fact about $\mathbb{R}$ is its \textsf{Archimedean property}: for each $x \in \mathbb{R}$ there is an integer $n$ that is greater than $x$.
In other words, there exist arbitrarily large integers.
The Archimedean property is true for $\mathbb{Q}$ since $p/q \leqslant |p|$.
It follows that it is true for $\mathbb{R}$.
Given $x = A \mid B$, just choose a rational number $r \in B$ and an integer $n > r$, then $n > x$.
An equivalent way to state the Archimedean property is that there exist arbitrarily small reciprocals of integers.

Finally let us note a nearly trivial principle that turns out to be invaluable in deriving inequalities and equalities in $\mathbb{R}$.

\begin{thm}[$\varepsilon$-principle]
If $a, b$ are real numbers and if for each $\varepsilon > 0$ we have $a \leqslant b + \varepsilon$ then $a \leqslant b$.  If $x,y$ are real numbers and for each $\varepsilon > 0$ we have $|x-y| \leqslant \varepsilon$ then $x=y$.
\end{thm}

\begin{proof}
  Trichotomy implies that either $a \leqslant b$ or $a > b$.  In the latter case we can choose $\varepsilon$ with $0 < \varepsilon < a-b$ and get the absurdity
  \[
    \varepsilon < a - b \leqslant \varepsilon.
  \]
  Hence $a \leqslant b$.  Similarly, if $x \ne y$ then choosing $\varepsilon$ with $0 < \varepsilon < |x-y|$ gives the contradiction $\varepsilon < |x-y| \leqslant \varepsilon$.  Hence $x=y$.
\end{proof}

  \section{Cauchy sequences}
\label{sec:Cauchy}

There is a second sense in which $\mathbb R$ is complete.
It involves the concept of convergent sequences.
Let $a_1, a_2, a_3, \ldots = \langle a_n \rangle$, $n \in \mathbb N$, be a sequence of real numbers.

\begin{defn}
  The sequence $\langle a_n \rangle$ \textsf{converges to a limit} $b \in \mathbb R$ as $n \to \infty$ provided that for each $\varepsilon > 0$ there exists an integer $N \in \mathbb N$ such that for all $n \geqslant N$ we have
  \[
    |a_n - b| < \varepsilon \qquad (\text{or } b - \varepsilon < a_n < b + \varepsilon.)
  \]
\end{defn}
In symbols we may write
\[
  \forall\,\,\varepsilon > 0 \,\, \exists\,\, N \in \mathbb N \text{ such that } n \geqslant N \implies |a_n - b| < \varepsilon.
\]

If a limit $b$ exists it is not hard to see that it is unique, and we write
\[
  \lim_{n\to\infty} a_n = b \qquad \text{or} \qquad a_n \to b.
\]

There are a few basic mentioned in the elementary calculus.
We summarize as follows.


  \begin{prop}
    \label{prop:basicpropertyoflimit}
    The following properties hold.
    \begin{enumerate}[$(1)$]
      \item (Uniqueness) If $\displaystyle \lim_{n\to\infty} a_n = L$ and $\displaystyle \lim_{n\to\infty} a_n = M$, then $L=M$.
      \item If a sequence $\langle a_n \rangle$ converges, then any of its subsequence converges to the same limit.
      \item (Arithmetics) If $a_n \to A$ and $b_n \to B$, then $a_n \odot b_n \to A \odot B$, where $\odot \in \{ +, -, \times, \div \}$.  That $B \ne 0$ is required for $\odot = \div$.
      \item (Squeeze theorem) Let $\langle x_n \rangle$, $\langle a_n \rangle$, $\langle b_n \rangle$ be three number sequences such that $a_n \leqslant x_n \leqslant b_n$ for all sufficiently large $n$.  If $a_n \to L$ and $b_n \to L$, then $x_n \to L$ as well.
      \item (Comparison theorem) Let $\langle a_n \rangle$ and $\langle b_n \rangle$ be two convergent sequences such that $a_n \leqslant b_n$ for all sufficiently large $n$.  Then $\displaystyle \lim_{n\to\infty} a_n \leqslant \lim_{n\to\infty} b_n$.
    \end{enumerate}
  \end{prop}

Below are a few basic limits.
\begin{thm}
  \begin{enumerate}[$(1)$]
    \item $\displaystyle\lim_{n\to\infty} \frac{1}{n} = 0$.
    \item For any $a > 0$, $\displaystyle \lim_{n\to\infty} a^{1/n} = 1$.
    \item $\displaystyle \lim_{n\to\infty} n^{1/n} = 1$.
    \item For any $p > 0$ and $\alpha \in \mathbb{R}$, $\displaystyle \lim_{n\to\infty} \frac{n^\alpha}{(1+p)^n} = 0$.
  \end{enumerate}
\end{thm}

\begin{proof}
  \begin{enumerate}[(1)]
    \item For any $\varepsilon > 0$, there is an integer $N \in \mathbb{N}$ such that $N \varepsilon > 1$, by the Archimedean property.
      Hence for any $n \geqslant N$, we have
      \[
	0 < \frac{1}{n} \leqslant \frac{1}{N} < \varepsilon,
      \]
      as required.
    \item There is nothing more to do for $a = 1$.
      Let us first assume that $a > 1$.
      Define a sequence $\langle x_n \rangle$ by $x_n = a^{1/n} - 1$, that is, $a = (1 + x_n)^n$ for each $n \in \mathbb{N}$.
      Using the binomial theorem, we see that
      \begin{align*}
	a &= (1 + x_n)^n \geqslant 1 + n x_n > n x_n, \\
	0 &< x_n < \frac{a}{n} \to 0 \qquad \text{as $n \to \infty$},
      \end{align*}
      therefore $\displaystyle \lim_{n \to \infty} a^{1/n} = \lim_{n \to \infty} (1 + x_n) = 1 + 0 = 1$.

      For $0 < a < 1$, we use the arithmetic property for limit to see that
      \[
	\lim_{n \to \infty} a^{1/n} = \lim_{n\to\infty} \frac{1}{(1/a)^{1/n}} = \frac{1}{1} = 1,
      \]
      since $1/a > 1$.  
      So this case is also done.

    \item Define a positive sequence $\langle y_n \rangle$ by $y_n = n^{1/n} - 1$, that is, $1 + y_n = n^{1/n}$ for each $n \in \mathbb{N}$.  Using the binomial theorem again, we see that for $n \geqslant 2$,
      \begin{align*}
	n &= (1 + y_n)^n \geqslant 1 + n y_n + \frac{n(n-1)}{2} y_n^2 > \frac{n(n-1)}{2} y_n^2, \\
	0 &< y_n^2 < \frac{2}{n-1} \to 0 \qquad \text{as $n \to \infty$}.
\end{align*}
	Since $\langle y_n^2 \rangle \to 0$, so does $\langle y_n \rangle$.

      \item The statement is clear when $\alpha \leqslant 0$. 
	For the case $\alpha > 0$, first find an integer $k \in \mathbb{N}$ that exceeds $\alpha$.
	Then for $n > k$, we have
	\begin{align*}
	  \frac{n^\alpha}{(1+p)^n} &\leqslant \frac{n^k}{\binom{n}{k+1} p^{k+1}} \to 0 \qquad
	  \text{as $n \to \infty$},
	\end{align*}
	since the denominator of the last fraction is a polynomial of degree $k+1$ in $n$.
  \end{enumerate}
\end{proof}

Note that the limit $b$ may be extraneous to the sequence $\langle a_n \rangle$.
A related concept is the following.

\begin{defn}
  A sequence $\langle a_n \rangle$ is called a \textsf{Cauchy sequence} (or obeys a Cauchy condition) if for each $\varepsilon > 0$ there is an integer $N \in \mathbb N$ such that for all $n, k \geqslant N$ we have
  \[
    |a_n - a_k| < \varepsilon.
  \]
\end{defn}

It is an easy exercise to show that a convergent sequence obeys a Cauchy condition.
The converse of the fact is a fundamental property of $\mathbb{R}$, as stated below.

\begin{thm}
  $\mathbb{R}$ is \textsf{complete} with respect to Cauchy sequences in the sense that if $\langle a_n \rangle$ is a sequence of real numbers which obeys a Cauchy condition then it converges to a limit in $\mathbb{R}$.
\end{thm}

\begin{proof}
  First we show that $\langle a_n \rangle$ is bounded.
  Taking $\varepsilon = 1$ in the Cauchy condition implies that there is an $N \in \mathbb{N}$ such that for all $n, k \geqslant N$ we have $|a_n - a_k| < 1$.
  Take $K$ large enough that $-K \leqslant a_1, \dots, a_N \leqslant K$.
  Set $M = K + 1$.  Then for all $n$ we have
  \[
    -M < a_n < M,
  \]
  which shows that the sequence $\langle a_n \rangle$ is bounded.

  Define a set $X$ as 
  \[
    X = \{ x \in \mathbb{R} \colon \exists\,\, \text{infinitely many $n$ such that } a_n \geqslant x \}.
  \]
  $-M \in X$ since for all $n$ we have $a_n > -M$, which $M \notin X$ since no $x_n \geqslant M$.  Thus $X$ is a nonempty subset of $\mathbb{R}$ which is bounded above by $M$.  The least upper bound property applies to $X$ and we have $b = \sup X$ with $-M \leqslant b \leqslant M$.

  We claim that $\langle a_n \rangle$ converges to $b$ as $n \to \infty$.
  Given $\varepsilon > 0$ we must show that there is an $N$ such that for all $n \geqslant N$ we have $|a_n - b| < \varepsilon$.
  Since $\langle a_n \rangle$ is a Cauchy sequence and $\varepsilon/2$ is positive there does exist an $N$ such that if $n, k \geqslant N$ then
  \[
    |a_n - a_k| < \frac{\varepsilon}{2}.
  \]
  Since $b - \varepsilon/2$ is less than $b$ it is not an upper bound for $X$, so there is an $x \in X$ with $b - \varepsilon/2 \leqslant x$.
  For infinitely many $n$ we have $a_n \geqslant x$.
  Since $b + \varepsilon/2 > b$ it does not belong to $X$, and therefore for only finitely many $n$ do we have $a_n > b + \varepsilon/2$.
  Thus, for infinitely many we have
  \[
    b - \frac{\varepsilon}{2} \leqslant x \leqslant a_n \leqslant b + \frac{\varepsilon}{2}.
  \]
  Since there are infinitely many of these $n$ there are infinitely many that are at least $N$.  Pick one, say $a_{n_0}$ with $n_0 \geqslant N$ and $b - \frac{\varepsilon}{2} \leqslant a_{n_0} \leqslant b + \frac{\varepsilon}{2}$.  Then for all $n \geqslant N$ we have
  \[
    |a_n - b| \leqslant |a_n - a_{n_0}| + |a_{n_0} - b| < \frac{\varepsilon}{2} + \frac{\varepsilon}{2} = \varepsilon,
  \]
  which completes the verification that $\langle a_n \rangle$ converges to $b$.
\end{proof}

Let us summarize as follows.
\begin{thm}
  A real sequence $\langle a_n \rangle$ converges if and only if it is a Cauchy sequence.
\end{thm}

Not every real sequence converges.  However, the following are available for every real sequence.

\begin{defn}
  Let $\langle a_n \rangle$ be a real sequence.  Define the \textsf{limit supremum} and the \textsf{limit infinum} for $\langle a_n \rangle$ as follows.
  \begin{align*}
    \limsup_{n\to\infty} a_n &= \lim_{n\to\infty} \left( \sup_{k\geqslant n} a_k \right); \\
    \liminf_{n\to\infty} a_n &= \lim_{n\to\infty} \left( \inf_{k\geqslant n} a_k \right).
  \end{align*}
\end{defn}

Since $\displaystyle \sup_{k \geqslant n} a_k$ is decreasing as $n$ gets larger and larger, $\displaystyle\limsup_{n\to\infty} a_n$ always exists in $\overline{\mathbb{R}}$; same for $\displaystyle\liminf_{n\to\infty} a_n$.
We state important properties about limit supremum; the analogues hold for limit infinum.

\begin{thm}
  \label{thm:limsup}
  Let $\langle a_n \rangle$ be a real sequence and $\displaystyle S = \limsup_{n\to\infty} a_n$.
  \begin{enumerate}[$(1)$]
    \item If $S = +\infty$, then $\langle a_n \rangle$ is not bounded above.
    \item If $S = -\infty$, then $\displaystyle \lim_{n\to\infty} a_n = -\infty$.
    \item If $S \in \mathbb{R}$, then for each $\varepsilon > 0$,
      \begin{enumerate}[$(i)$]
	\item there is an $N \in \mathbb{N}$ such that for every $n \geqslant N$ we have $a_n < S + \varepsilon$; 
	\item there are infinitely many terms $a_n$ with $a_n > S - \varepsilon$.
      \end{enumerate}
  \end{enumerate}
\end{thm}

\begin{proof}
Set $A_n = \displaystyle \sup_{k \geqslant n} a_k$; as noted before, $\langle A_n \rangle$ is a decreasing sequence in $\overline{\mathbb{R}}$.  Let us discuss the three cases separately.
  \begin{enumerate}[(1)]
    \item If $S = +\infty$, then $A_n$ must be $+\infty$ for all $n$ since $\langle A_n \rangle$ is decreasing.  In particular $\sup \{ a_k \colon k \geqslant 1 \} = A_1 = +\infty$, which is equivalent of saying that $\langle a_n \rangle$ is not bounded above.
    \item Assume $S = -\infty$.  For every $M \in \mathbb{R}$ there is an integer $N \in ~\mathbb{N}$ such that $A_n \leqslant M$ for all $n \geqslant N$.  Since $a_k \leqslant A_n$ for any $k \geqslant n$, the same can be said to $a_n$.  This shows that $a_n \to -\infty$ as $n \to \infty$.
    \item Now assume $S \in \mathbb{R}$ and a positive number $\varepsilon > 0$ is given.  There is an integer $n \in \mathbb{N}$ such that $S - \varepsilon < A_n < S + \varepsilon$ for all $n \geqslant N$.  Firstly that $a_n \leqslant A_N < S + \varepsilon$ for all $n \geqslant N$ shows (i).  As for part (ii), note that for each $n \geqslant N$ there is some $k \geqslant n$ such that $a_k > S - \varepsilon$ since the latter is smaller than the least upper bound $A_n$ of the set $\{ a_k \colon k \geqslant n \}$.  This means that we can always find some term that is larger than $S - \varepsilon$ beyond any point, hence part (ii) is shown.
  \end{enumerate}
\end{proof}

Finally we list a few properties for upper and lower limits.

\begin{prop}
  Let $\langle x_n \rangle$ and $\langle y_n \rangle$ be two real sequences.
  \begin{enumerate}[$(1)$]
    \item Let $\displaystyle S = \limsup_{n \to \infty} x_n$.  Then there is a subsequence $\langle x_{n_i} \rangle$ of $\langle x_n \rangle$ such that $x_{n_i} \to S$ as $i \to \infty$.

    \item If $T$ is a subsequential limit of $\langle x_n \rangle$, then $\displaystyle T \leqslant \limsup_{n \to \infty} x_n$.

    \item We have
      \[
	\limsup_{n \to \infty} (x_n + y_n) \leqslant \left( \limsup_{n \to \infty} x_n \right) + \left( \limsup_{n \to \infty} y_n \right),
      \]
      except the case where the right-hand side is $\infty + (-\infty)$, whose result is undefined.

      Furthermore, there are examples where the strict inequality $<$ actually happen.

    \item If $\displaystyle \lim_{n\to\infty} y_n$ exists in $[-\infty, \infty]$, then
      \[
	\limsup_{n \to \infty} (x_n + y_n) = \left( \limsup_{n \to \infty} x_n \right) + \left( \lim_{n \to \infty} y_n \right),
      \]
      except the case where the right-hand side is $\infty + (-\infty)$, whose result is undefined.
      
    \item $\displaystyle \lim_{n \to \infty} x_n$ exists in $[-\infty, \infty]$ if and only if $\displaystyle \limsup_{n \to \infty} x_n = \liminf_{n \to \infty} x_n$.

  \end{enumerate}
\end{prop}

\begin{proof}
  \begin{enumerate}[(1)]
    \item There are three cases to consider: $S= \infty$, $S = - \infty$, or $S \in \mathbb{R}$.

      If $S = \infty$, by Theorem~\ref{thm:limsup} (1) we know that $\langle x_n \rangle$ is not bounded above.  Hence we may find an increasing sequence $n_1 < n_2 < n_3 < \cdots$ of positive integers such that $x_{n_i} \geqslant i$ for all $i \in \mathbb{N}$.  It is clear that this subsequence works, i.e., $x_{n_i} \to \infty$ as $i \to \infty$.

      If $S = -\infty$, then by Theorem~\ref{thm:limsup} (2) we know that $\displaystyle \lim_{n \to \infty} x_n = -\infty$.  Obviously the statement holds because a sequence is a subsequence of itself.

      Now we assume that $S \in \mathbb{R}$.  For convenience we set $n_0 = 0$.  By Theorem~\ref{thm:limsup} (3), for each $i \in \mathbb{N}$, there is an integer $n_i \in \mathbb{N}$ such that $n_i > n_{i-1}$ and $S - \frac{1}{i} < x_{n_i} < S + \frac{1}{i}$ (considering $\varepsilon = \frac{1}{i}$).  By the squeeze theorem we see that $\displaystyle \lim_{i \to \infty} x_{n_i} = S$.

    \item Let $\langle x_{n_i} \rangle$ be a subsequence of $\langle x_n \rangle$ such that $x_{n_i} \to T$ as $i \to \infty$.  Since for each $i \in \mathbb{N}$,
      \[
	x_{n_i} \leqslant \alpha_{n_i} = \sup \{ x_k \colon k \geqslant n_i \},
      \]
      hence by the limit comparison theorem
      \[
	T = \lim_{n \to \infty} x_{n_i} \leqslant \lim_{i \to \infty} \alpha_{n_i} = S.
      \]

    \item Define for each $n \in \mathbb{N}$,
      \[
	\alpha_n = \sup \{ x_k \colon k \geqslant n \}, \qquad
	\beta_n  = \sup \{ y_k \colon k \geqslant n \}.
      \]
      Then for any $k \geqslant n$, we have
      \[
	x_k + y_k \leqslant \alpha_n + \beta_n.
      \]
      This implies that
      \[
	\sup \{ x_k + y_k \colon k \geqslant n \} \leqslant \alpha_n + \beta_n.
      \]
      Hence by the limit comparison theorem,
      \[
	\limsup_{n \to \infty} (x_n + y_n) \leqslant \lim_{n \to \infty} \alpha_n + \lim_{n \to \infty} \beta_n = \limsup_{n \to \infty} x_n + \limsup_{n \to \infty} y_n.
      \]
      The exceptional case of $\infty + (-\infty)$ (or $(-\infty) + \infty$) on the right-hand side can be seen easily.

    \item We follow the proof of (3) and the notations in it.  Note that if at least one of the limits $\displaystyle \limsup_{n\to\infty} x_n$ and $\displaystyle \lim_{n\to\infty} y_n$ is infinite, the result is clear.  Therefore we assume that $\displaystyle \limsup_{n\to\infty} x_n = A$ and $\displaystyle \lim_{n\to\infty} y_n = B$ are finite, and we are proving the reverse inequality of (3).

      For any $\varepsilon > 0$, there are infinitely many $n$'s such that $x_n \geqslant A - \dfrac{\varepsilon}{2}$.  On the other hand, there is an integer $N \in \mathbb{N}$ such that $y_n \geqslant B - \dfrac{\varepsilon}{2}$ for all $n \geqslant N$.  Therefore there are infinitely many $n$'s such that $x_n + y_n \geqslant A + B - \varepsilon$.  This implies that
      \[
	\limsup_{n\to\infty} (x_n + y_n) \geqslant A + B - \varepsilon.
      \]
      Since $\varepsilon > 0$ is arbitrary, we conclude that
      \[
	\limsup_{n\to\infty} (x_n + y_n) \geqslant A + B = \limsup_{n\to\infty} x_n + \lim_{n\to\infty} y_n.
      \]

    \item If any of the three numbers is $\infty$ or $-\infty$, then Theorem~\ref{thm:limsup} (2) takes care the argument.

      Now assume $\displaystyle S = \lim_{n\to\infty} x_n$ is a real numbers.  Then part (1) of this Proposition shows that $\displaystyle \limsup_{n\to\infty} x_n = \liminf_{n\to\infty} x_n = S$ by Proposition~\ref{prop:basicpropertyoflimit} (2).

      Conversely, assume that $\displaystyle S = \limsup_{n\to\infty} x_n = \liminf_{n\to\infty} x_n$.  By Theorem~\ref{thm:limsup} (3)(i), for any $\varepsilon > 0$, there is an integer $N \in \mathbb{N}$ such that $S - \varepsilon < x_n < S + \varepsilon$ whenever $n \geqslant N$.  This is precisely the definition of $\displaystyle \lim_{n\to\infty} x_n = S$.
  \end{enumerate}
\end{proof}

  \section{Cardinality}
\label{sec:cardinality}

In this chapter the following basic concepts about functions between sets are assumed:

\begin{quote}
  \textsf{function/mapping, domain, target/codomain, range/image, injection/one-to-one, \\ surjection/onto, bijection, identity map, composite.}
\end{quote}
(Words that are separated by a slash are equivalent.)

Let $A, B$ be two sets.
If there is a bijection from $A$ onto $B$ then $A$ and $B$ are said to have \textsf{equal cardinality}, and we hereby write $A \sim B$.
The relation $\sim$ is an equivalence relation, that is,
\begin{itemize}
  \item $A \sim A$.
  \item $A \sim B$ implies $B \sim A$.
  \item $A \sim B \sim C$ implies $A \sim C$.
\end{itemize}
We also write $\operatorname{card} A = \operatorname{card} B$, $\# A = \# B$, or $|A| = |B|$ when $A, B$ have equal cardinality.

\begin{defn}
  A set $S$ is called
  \begin{itemize}
    \item \textsf{finite} if it is empty or for some $n \in \mathbb{N}$ we have $S \sim \{1, 2, \dots, n \}$.
    \item \textsf{infinite} if it is not finite.
    \item \textsf{countable} if $S \sim \mathbb{N}$.
    \item \textsf{at most countable} if it is finite or countable.
    \item \textsf{uncountable} if it is infinite but not countable.
  \end{itemize}
\end{defn}

That $S$ is countable means that there is a bijection $f: \mathbb{N} \to S$, and this gives a way to list the elements of $S$ as $s_1 = f(1)$, $s_2 = f(2)$, $s_3 = f(3)$, etc.
Conversely, if a set $S$ is presented as an infinite list (without repetition) $S = \{ s_1, s_2, s_3, \dots \}$, then it is countable:  Define $f(k) = s_k$ for all $k \in \mathbb{N}$.
In brief, ``countable'' is ``listable.''

We are going to establish two useful facts: $\mathbb{N}, \mathbb{Z}$, and $\mathbb{Q}$ are all countable, but $\mathbb{R}$ is uncountable.
That $\mathbb{N}$ and $\mathbb{Z}$ are countable are easy: using the identity map on $\mathbb{N}$ shows that $\mathbb{N}$ is itself countable, and we may list elements of $\mathbb{Z}$ as
\[
  0, 1, -1, 2, -2, 3, -3, \dots, k, -k, \dots.
\]
In fact, we are going to show first that the set of all {\em positive} rational numbers are countable, then using the trick above to conclude that $\mathbb{Q}$ is also countable.

\begin{prop}
  \label{prop:3-2}
  \begin{enumerate}[(1)]
    \item Any infinite subset $A$ of a countable set $B$ is also countable.
    \item If there is a surjection $f: \mathbb{N} \to B$ and $B$ is infinite then $B$ is countable.
  \end{enumerate}
\end{prop}

\begin{proof}
  \begin{enumerate}[(1)]
    \item After a bijection, we may assume that $B = \mathbb{N}$.
      Let $a_k$ be the $k^{\text{th}}$ smallest number in $A$.
      Then $k \mapsto a_k$ establishes a bijection from $\mathbb{N}$ onto $A$, hence $A$ is countable.

    \item For each $b \in B$ the set $\{k \in \mathbb{N} \colon f(k) = b \}$ is nonempty and hence contains a smallest element; say $h(b) = k$ is the smallest integer that is sent to $b$ by $f$.
      Clearly if $b, b' \in B$ and $b \ne b'$ then $h(b) \ne h(b')$.
      That is, $h: B \to \mathbb{N}$ is an injection which bijects $B$ to $h(B) \subseteq \mathbb{N}$.  Since $B$ is infinite, so is $h(B)$; since $h(B)$ is countable by (1), so is $B$.
  \end{enumerate}
\end{proof}

\begin{prop}
  $\mathbb{N} \times \mathbb{N}$ is countable.
\end{prop}

\begin{proof}
  Define a function $f: \mathbb{N} \times \mathbb{N} \to \mathbb{N}$ as 
  \[
    f(i,j) = \frac{(i+j-1)(i+j-2)}{2} + i, \qquad i, j \in \mathbb{N}.
  \]
  One checks that $f$ is indeed a bijection, therefore $\mathbb{N} \times \mathbb{N}$ is countable.
\end{proof}

\noindent{\em Remark.} Using $f^{-1}$ in the proof above, the elements of $\mathbb{N} \times \mathbb{N}$ are listed in order as
\[
  (1,1), (1,2), (2,1), (1,3), (2,2), (3,1), (1,4), (2,3), (3,2), (4,1), \dots.
\]

\begin{cor}
  \label{cor:3-4}
  The Cartesian product of two countable sets is still countable.
\end{cor}

\begin{thm}
  $\mathbb{Q}$ is countable.
\end{thm}

\begin{proof}
  We are going to show that the set $\mathbb{Q}_{>0}$ is countable.
  After that, the trick from $\mathbb{N}$ to $\mathbb{Z}$ can be applied again to show that $\mathbb{Q}$ is countable.

  Indeed, $\mathbb{Q}_{>0}$ is an infinite set.  The function
  \[
    f(i,j) = \frac{i}{j}, \qquad i, j \in \mathbb{N},
  \]
  is a surjection from $\mathbb{N} \times \mathbb{N}$ onto $\mathbb{Q}_{>0}$.
  Therefore $\mathbb{Q}_{>0}$ is countable by (2) in Proposition~\ref{prop:3-2}.
\end{proof}

\begin{cor}
  For each $m \in \mathbb{N}$ the set $\mathbb{Q}^m$ is countable.
\end{cor}

\begin{proof}
  Use Corollary~\ref{cor:3-4} and mathematical induction sufficiently many times.
\end{proof}

Unlike $\mathbb{Q}$, the cardinality of $\mathbb{R}$ behaves completely different.

\begin{thm}
  $\mathbb{R}$ is uncountable.
\end{thm}

\begin{proof}
  There are other proofs of the uncountability of $\mathbb{R}$, but none so beautiful as this one, which was due to Cantor.
  It is assumed here that each real number $x$ has a decimal expansion $x = N.x_1x_2x_3\dots$, and it is uniquely determined by $x$ except for $x = \dfrac{K}{10^n}$ with $K \in \mathbb{Z}, n \in \mathbb{Z}_{\geqslant 0}$ (there are two in this case.)
  
  Assume the contrary, i.e. $\mathbb{R}$ is not uncountable.
  Being infinite, $\mathbb{R}$ must then be countable.
  Hence there is a bijection $f : \mathbb{N} \to \mathbb{R}$.
  Using $f$, we list the elements of $\mathbb{R}$ along their decimal expansions as an array, and consider the digits $x_{ii}$ that occur along the diagonal in this array, see below.

  \begin{figure}[t]
    \centering
    \begin{tikzpicture}
      \foreach \x in {1,2,3,4,5,6,7}
      {\node at (0, -\x) {$N_{\x}.$};
	\node at (-0.75,-\x) {$=$};
       \node at (-1.6,-\x) {$f(\x)$};
      }
      \foreach \x in {1,2,3,4,5,6,7}
        \foreach \y in {1,2,3,4,5,6,7}
	  \node at (\y, -\x) {$x_{\x\y}$};
      \node at (-0.75,-8) {$\vdots$};
    \end{tikzpicture}
  \end{figure}

  For each $i$, choose a digit $y_i$ such that $y_i \ne x_{ii}$ and $y_i \notin \{0,9\}$.  Then the number $y = 0.y_1y_2y_3\dots$ does not appear in the list of all elements of $\mathbb{R}$, which is a contradiction!
\end{proof}

\begin{cor}
  The (non-degenerate) intervals $(a,b)$ and $[a,b]$ are both uncountable. 
\end{cor}

  \section{The Completeness Axioms}
\label{sec:complete-axioms}

The playground for analysis is the real number system.
The real number system is an ordered field that contains the rational number system as a subfield.
The main difference between these two fields are the so-called {\em completeness axiom}.
Indeed, there are four properties that are often refered as the axiom.  They are:

\begin{property}[Least upper bound property]
  Let $E$ be a nonempty subset of $\mathbb{R}$ which is bounded above.
  Then the supremum $\sup E$ for $E$ exists in $\mathbb{R}$.
\end{property}

\begin{property}[Monotone convergence property]
  Let $\langle a_n \rangle$ be an increasing real sequence which is bounded above.
  Then $\langle a_n \rangle$ converges in $\mathbb{R}$.
\end{property}

\begin{property}[Nested interval property]
  Let $\langle I_n \rangle$ be a nested sequence of nondegenerate bounded closed intervals in $\mathbb{R}$, that is, they are closed intervals such that $I_1 \supseteq I_2 \supseteq I_3 \supseteq \cdots$.  Then their intersection $\cap_n I_n$ is nonempty in $\mathbb{R}$.
\end{property}

\begin{property}[Convergence of Cauchy sequences]
  Any Cauchy sequence of real numbers converges in $\mathbb{R}$.
\end{property}

These four properties are equivalent in the following sense: if any one is assumed to be true (i.e.\ an axiom), then the other three are valid (i.e.\ theorems).
So far in our class the implication $(1) \implies (4)$ has been shown, and $(1) \implies (2)$ is assigned as an exercise.
In this note we are going to fill in the rest so that anyone can be implied by each of the others.

\noindent\textbf{(2) $\implies$ (3)}.
Write each $I_n = [a_n, b_n]$, where $a_n, b_n \in \mathbb{R}$ with $b_n - a_n > 0$ for any $n \in \mathbb{N}$ since $I_n$ is nondegenerate.
Since $\langle I_n \rangle$ is a nested sequence, we have
\[
  a_1 \leqslant a_2 \leqslant \cdots \leqslant a_n \leqslant \cdots \leqslant b_n \leqslant \cdots \leqslant b_2 \leqslant b_1.
\]
From this we see that the sequence $\langle a_n \rangle$ of left endpoints is increasing and bounded above (by $b_1$), while the sequence $\langle b_n \rangle$ of right endpoints is decreasing and bounded below (by $a_1$).
By the monotone convergence property (2), we deduce that both
\[
  \alpha = \lim_{n \to \infty} a_n, \qquad \text{and} \qquad
  \beta  = \lim_{n \to \infty} b_n
\]
exist in $\mathbb{R}$.
Since $a_n < b_n$ for any $n \in \mathbb{N}$, we have $\alpha \leqslant \beta$ by the limit comparison theorem; in particular $[\alpha, \beta] \ne \varnothing$.
Since $\alpha \geqslant a_n$ and $\beta \leqslant b_n$ for any $n$, we see that $[\alpha, \beta] \subseteq I_n$ for all $n \in \mathbb{N}$, and this implies that
\[
  \varnothing \ne [\alpha, \beta] \subseteq \bigcap_{n=1}^\infty I_n,
\]
which is needed to show. \qed

\noindent\textit{Remark.} There is a further implication for the nested interval property: if the length of $I_n$, i.e, $b_n - a_n$, converges to $0$ as $n \to \infty$, then the intersection of all those intervals is a singleton, namely $\{ \alpha \} = \{ \beta \}$.  This fact can easily be seen in the argument above.
  
\medskip
\noindent\textbf{(3) $\implies$ (1)}.
Suppose we are given a nonempty subset $E$ of $\mathbb{R}$ which is bounded above by $M \in \mathbb{R}$.
Since $E$ is nonempty, we may pick an element $a_1 \in E$; let $b_1 = M$.
Clearly $a_1 \leqslant b_1$.
Now we proceed recursively: suppose $a_n \leqslant b_n$ are given so that $a_n \in E$ and $b_n$ is an upper bound for $E$.
If $a_n = b_n$ then the process is terminated since this common number is already the maximum of $E$, hence it is the supremum of $E$.
Otherwise let us consider $c_n = \dfrac{a_n + b_n}{2}$; this $c_n$ is either an upper bound for $E$ or not.
If $c_n$ is an upper bound for $E$, then we set $a_{n+1} = a_n$ and $b_{n+1} = c_n$; if $c_n$ is not an upper bound for $E$, then we set $b_{n+1} = b_n$, and choose an element $a_{n+1} \in E$ such that $a_{n+1} > c_n$.
In any case $a_{n+1} \in E$ and $b_{n+1}$ is still an upper bound for $E$, so the induction goes through.
Moreover it holds that $b_{n+1} - a_{n+1} \leqslant \frac{1}{2} (b_n - a_n)$, hence $b_n - a_n \to 0$ as $n \to \infty$..
Through this process, when it does not stop after a finite number of steps, a nested sequence $\langle I_n = [a_n, b_n] \rangle$ of bounded closed intervals has been constructed.
By the nested interval property, there is exactly one point $\alpha \in \mathbb{R}$ that belongs to each of the intervals $I_n = [a_n, b_n]$.
We hereby claim that $\alpha = \sup E$.

As $\alpha$ is the limit of a sequence $\langle b_n \rangle$ of upper bounds for $E$, $\alpha$ itself is also an upper bound for $E$: if not, then there is a number $x \in E$ with $x > \alpha$; take $\varepsilon = x - \alpha > 0$, and there is an $N \in \mathbb{N}$ such that $b_n < \alpha + \varepsilon = x$ for any $n \geqslant N$, contradicting to the construction that $b_n$ is an upper bound for $E$.
On the other hand, if $t < \alpha$, then we consider $\varepsilon = \alpha - t > 0$.  As $a_n \to \alpha$ as $n \to \infty$, there is an $N' \in \mathbb{N}$ such that $a_n > \alpha - \varepsilon = t$ for any $n \geqslant N'$. 
This means that $t$ is not an upper bound for $E$, because $a_n \in E$ by construction.
Since we have shown that any number $t$ smaller than $\alpha$ cannot be an upper bound for $E$, together we conclude that $\alpha = \sup E$.  \qed

\medskip
\noindent\textbf{(4) $\implies$ (2)}.
Suppose $\langle a_n \rangle$ is an increasing sequence that is bounded above.
We claim that it is a Cauchy sequence.

If $\langle a_n \rangle$ is not a Cauchy sequence, then there is an $\varepsilon > 0$ such that, for any $N \in \mathbb{N}$, there are $m, n \geqslant N$ such that $|a_m - a_n| \geqslant \varepsilon$.
Let us start with $N = 1$.
By assumption, there are $m_1 \geqslant n_1 \geqslant 1$ such that $a_{m_1} \geqslant a_{n_1} + \varepsilon \geqslant a_1 + \varepsilon$ (note that we use the assumption that $\langle a_n \rangle$ is increasing).
For this $m_1$, there are $m_2 \geqslant n_2 \geqslant m_1$ such that $a_{m_2} \geqslant a_{n_2} + \varepsilon \geqslant a_{m_1} + \varepsilon \geqslant a_1 + 2 \varepsilon$.
Proceed inductively, we see that there is an increasing sequence $\langle m_k \rangle$ of positive integers such that $a_{m_k} \geqslant a_1 + k \varepsilon$ for any $k \in \mathbb{N}$.
But this contradicts to the hypothesis that $\langle a_n \rangle$ is bounded above, per the Archimedean principle.

Now we have shown that $\langle a_n \rangle$ is a Cauchy sequence, hence by Property~4 it converges in $\mathbb{R}$.  \qed

\medskip
Now we have come to a full circle that all these four properties are proved equivalent, as the following figure shows.  We hope that a reader can appreciate the importance of these properties, as they establish the essential properties of reals used in analysis.

\begin{center}
  \begin{tikzpicture}
    \draw (0,0) node [fill=red!20!white, draw] (l) {Least upper bound property (1)};
    \draw (8,-4) node [fill=green!50!white, draw] (m) {Monotone convergence property (2)};
    \draw (8,0) node [fill=blue!20!white, draw] (c) {Convergence of Cauchy sequences (4)};
    \draw (0,-4) node [fill=yellow, draw] (n) {Nested interval property (3)};
    \draw [line width=1pt, double distance=2pt, arrows=-Stealth] (l) -- (m);
    \draw [line width=1pt, double distance=2pt, arrows=-Stealth] (m) -- (n);
    \draw [line width=1pt, double distance=2pt, arrows=-Stealth] (n) -- (l);
    \draw [line width=1pt, double distance=2pt, arrows=-Stealth] (l) -- (c);
    \draw [line width=1pt, double distance=2pt, arrows=-Stealth] (c) -- (m);
  \end{tikzpicture}
\end{center}

  \chapter{Continuous Real Functions}
\label{chap:continuous}

The behavior of a continuous function defined on an interval $[a,b]$ is at the root of all calculus theory.
Using solely the Least Upper Bound Property of the real numbers we rigorously derive the basic properties of such functions.

\begin{defn}
  A function $f: [a,b] \to \mathbb{R}$ is \textsf{continuous} if for each $x \in [a,b]$ and each $\varepsilon>0$ there is a $\delta>0$ such that
  \[
    t \in [a,b] \text{ and } |t-x| < \delta \implies |f(t)-f(x)| < \varepsilon.
  \]
\end{defn}

Continuous functions are found everywhere in analysis and topology.
We prove three of the most important properties derived from continuity.

\begin{thm}
  \label{thm:cont-bound}
  The values of a continuous function $f$ defined on a bounded closed interval $[a,b]$ form a bounded subset of $\mathbb{R}$.
  That is, there exist $m,M\in\mathbb{R}$ such that for all $x \in [a,b]$ we have $m \leqslant f(x) \leqslant M$.
\end{thm}

\begin{proof}
  For $x \in [a,b]$, let $V_x$ be the value set of $f(t)$ as $t$ varies from $a$ to $x$, that is,
  \[
    V_x = \{ \alpha \in \mathbb{R} \colon \text{ for some $t \in [a,x]$ we have $\alpha = f(t)$ } \}.
  \]
  Set
  \[
    X = \{ x \in [a,b] \colon \text{ $V_x$ is a bounded subset of $\mathbb{R}$ } \}.
  \]
  Our goal is to show that $b \in X$.
  Clearly $a \in X$ and $b$ is an upper bound for $X$.
  Since $X$ is nonempty and bounded above, there exists in $\mathbb{R}$ a least upper bound $c \leqslant b$ for $X$.
  First we take $\varepsilon=1$ in the definition of continuity at $a$ we see that $c > a$
  (Think about it.)
  Taking $\varepsilon = 1$ in the definition of continuity at $c$, there exists a $\delta>0$ such that $x\in[a,b]$ and $|x-c|<\delta$ implies that $|f(x)-f(c)|<1$.
  Since $c$ is the least upper bound for $X$, there exists $x_0\in X$ in the interval $[c-\delta,c]$ (otherwise $c-\delta$ would be a smaller upper bound for $X$.)
  Now as $t$ varies from $a$ to $c$, the value $f(t)$ varies first in the bounded set $V_{x_0}$ and then in the bounded set $J=(f(c)-1,f(c)+1)$.
  The union of two bounded sets is a bounded set and it follows that $V_c$ is bounded, so $c\in X$.
  Besides, if $c<b$ then $f(t)$ continues to vary in the bounded set $J$ for $t>c$, contrary to the fact that $c$ is an upper bound for $X$.
  Thus, $b = c \in X$, and the values of $f$ form a bounded subset of $\mathbb{R}$.
\end{proof}

\begin{thm}
  \label{thm:EVT}
  A continuous function $f$ defined on a bounded closed interval $[a,b]$ takes on absolute minimum and absolute maximum values: For some $x_0,x_1\in[a,b]$ and for all $x\in[a,b]$ we have
  \[
    f(x_0) \leqslant f(x) \leqslant f(x_1).
  \]
\end{thm}

\begin{proof}
  Let $M = \sup f(t)$ as $t$ varies in $[a,b]$.
  By Theorem~\ref{thm:cont-bound} $M$ exists.
  Consider the set $X = \{ x \in [a,b] \colon \text{ $\sup V_x < M$} \}$ where, as above, $V_x$ is the set of values of $f(t)$ as $t$ varies on $[a,x]$.

  \noindent\underline{Case 1.} $f(a) = M$.  Then $f$ takes on a maximum at $a$ and the theorem is proved.

  \noindent\underline{Case 2.} $f(a) < M$.  Then $X \ne \varnothing$ and we can consider the least upper bound for $X$, say $c$.
  If $f(c) < M$, we choose $\varepsilon > 0$ with $\varepsilon < M - f(c)$.
  By continuity of $f$ at $c$, there exists a $\delta > 0$ such that $|t - c| < \delta$ implies $|f(t) - f(c)| < \varepsilon$.
  Thus, $\sup V_c < M$.
  If $c < b$ this implies that there exist points $t$ to the right of $c$ at which $\sup V_t < M$, contrary to the fact that $c$ is an upper bound of such points.
  Therefore $c = b$, which implies that $M < M$, a contradiction.
  Having arrived at a contradiction from the supposition that $f(c) < M$, we duly conclude that $f(c) = M$, so $f$ assumes a maximum at $c$.

  The situation with minima is similar and we omit the details here.
\end{proof}

\begin{thm}[Intermediate value theorem]
  \label{thm:IVT}
  A continuous function defined on a bounded closed interval $[a,b]$ takes on (or ``achieves,'' ``assumes,'' or ``attains'') all intermediate values: That is, if $f(a) = \alpha$, $f(b) = \beta$, and $\gamma$ is given, $\alpha \leqslant \gamma \leqslant \beta$, then there is some $c \in [a,b]$ such that $f(c) = \gamma$.
  The same conclusion holds if $\beta \leqslant \gamma \leqslant \alpha$.
\end{thm}  

The theorem is pictorially obvious.
A continuous function has a graph that is a curve without break points.
Such a graph cannot jump from one height to another.
It must pass through all intermediate heights.

\begin{proof}
  Set $X = \{ x \in [a,b] \colon \text{ $\sup V_x \leqslant \gamma$} \}$ and $c = \sup X$.
  Now $c$ exists because $X$ is nonempty (it contains $a$) and it is bounded above (by $b$.)
  We claim that $f(c) = \gamma$.

  To prove it we just eliminate the other two possibilities which are $f(c) < \gamma$ and $f(c) > \gamma$, by showing that each leads to a contradiction.
  Suppose that $f(c) < \gamma$ and take $\varepsilon = \gamma - f(c)$.
  Continuity of $f$ at $c$ gives $\delta > 0$ such that $|t - c| < \delta$ implies $|f(t) - f(c)| < \varepsilon$.  That is,
  \[
    t \in (c-\delta, c+\delta) \implies f(t) < \gamma,
  \]
  so $c + \delta/2 \in X$, contrary to $c$ being an upper bound of $X$.


  Suppose now that $f(c) > \gamma$ and take $\varepsilon = f(c) - \gamma$.
  Continuity of $f$ at $c$ gives $\delta > 0$ such that $|t - c| < \delta$ implies $|f(t) - f(c)| < \varepsilon$.  That is,
  \[
    t \in (c-\delta, c+\delta) \implies f(t) > \gamma,
  \]
  so $c - \delta/2$ is an upper bound of $X$, contrary to $c$ being the least upper bound for $X$.

  Since $f(c)$ is neither $< \gamma$ nor $> \gamma$ we get $f(c) = \gamma$.
\end{proof}

The following is one of the most important results about continuous functions: their properties can be \textit{upgraded} over a bounded closed interval.

  \begin{thm}
    \label{thm:unif-cont}
   Let $f$ be a continuous real function on a bounded closed interval $[a,b]$.
   Then $f$ is uniformly continuous on $[a,b]$.
  \end{thm}

  \begin{proof} 
    Let $\varepsilon > 0$ be given.  Define
  \begin{align*}
    \mathcal A(\delta) &= \{ u \in [a,b] \colon \text{ if $x,t \in [a,u]$ and $|x-t| < \delta$ then $|f(x) - f(t)| < \varepsilon$} \}, \\
    \mathcal A &= \bigcup_{\delta>0} \mathcal A(\delta).
  \end{align*}
  Note that $\mathcal A(\delta) \subseteq \mathcal A(\delta')$ if $\delta \leqslant \delta'$.
  Clearly $a \in \mathcal A$ and $\mathcal A$ is bounded above by $b$.  Hence $c = \sup \mathcal A$ exists and $a \leqslant c \leqslant b$.

  By continuity of $f$ at $a$, there is a $\delta > 0$ such that $t \in [a, a+2\delta)$ implies that $|f(t) - f(a)| < \varepsilon/2$.
  Thus whenever $x,t \in [a,a+\delta]$ we have $|f(x)-f(t)| \leqslant |f(x)-f(a)| + |f(a)-f(t)| < \varepsilon$.  This implies $a + \delta \in \mathcal A(\delta) \subseteq \mathcal A$ and $c \geqslant a + \delta > a$.

  Let us assume that $c < b$ and try to get a contradiction.
  By continuity of $f$ at $c$, there is a $\delta_1 > 0$ such that $(c-\delta_1, c+\delta_1) \subseteq [a,b]$ and whenever $|x-c| < \delta_1$, we have $|f(x) - f(c)| < \varepsilon/2$.
  This implies that whenever $x,t \in (c-\delta_1, c+\delta_1)$, $|f(x)-f(t)| < \varepsilon$.
  Since $c = \sup \mathcal A$, there is a $u \in \mathcal A$ such that $u > c-\delta_1$; say $u \in \mathcal A(\delta_2)$ for some $\delta_2 > 0$.
  Take $\delta_3 = \min \{ 2\delta_1, \delta_2, u - (c - \delta_1) \} > 0$.
  This choice of $\delta_3$ guarantees that, whenever $x,t \in [a,c+\delta_1)$ and $|x-t| < \delta_3$, either both of them lie in $[a,u]$ or both of them lie in $(c-\delta_1, c+\delta_1)$.
  Because $\mathcal A(\delta_2) \subseteq \mathcal A(\delta_3)$, both cases make sure that $|f(x)-f(t)| < \varepsilon$ whenever $x,t \in [a, c+\frac{\delta_2}{2}]$, which shows that $c+\frac{\delta_2}{2} \in \mathcal A(\delta_3) \subseteq \mathcal A$, that contradicts to the fact that $c$ is an upper bound for $\mathcal A$.

  Since $c < b$ is impossible, we see that $c=b$ and an argument similar to the previous paragraph shows that $b = c \in \mathcal A$.
  Now $b \in \mathcal A$ guarantees that there is a $\delta > 0$ such that $b \in \mathcal A(\delta)$; that is, whenever $x,t \in [a,b]$ and $|x-t| < \delta$, it implies that $|f(x)-f(t)| < \varepsilon$.
  Since $\varepsilon$ is arbitrary, we conclude that $f$ is uniformly continuous on $[a,b]$.
  \end{proof}

  \textit{Remark.} The above proof of Theorem~\ref{thm:unif-cont} uses the least upper bound property only.  Below we give another proof which is easier to comprehend.  To start, we call a theorem that is introduced in a later topic (Theorem 6, Topic 10).

  \noindent\textbf{Bolzano-Weierstrass theorem.}
  \textit{Every bounded infinite real sequence has a convergent subsequence.}

  We proceed to prove Theorem~\ref{thm:unif-cont}.  Assume the contrary.
  If $f$ is not uniformly continuous on $[a,b]$, there is an $\varepsilon>0$ satisfying: for any $n \in \mathbb{N}$, there are two points $x_n, y_n \in [a,b]$ such that $|x_n - y_n| < \frac1n$ but $|f(x_n) - f(y_n)| \geqslant \varepsilon$.  Since $\langle x_n \rangle$ is a bounded sequence, there is a subsequence $\langle x_{n_i} \rangle$ of $\langle x_n \rangle$ that converges to $c \in [a,b]$, by the Bolzano-Weierstrass theorem.  Notice that
  \[
    x_{n_i} - \frac1{n_i} < y_{n_i} < x_{n_i} + \frac1{n_i}
  \]
  for all $i$, we conclude that $\langle y_{n_i} \rangle$ converges to $c$ too, by the squeeze theorem.
  By our hypothesis $f$ is continuous at $c \in [a,b]$, hence by the sequential characterization of limit of function we have
  \[
    \lim_{i\to\infty} f(x_{n_i}) = f(c) = \lim_{i\to\infty} f(y_{n_i}),
  \]
  which gives $\displaystyle \lim_{i \to \infty} \left( f(x_{n_i}) - f(y_{n_i}) \right) = f(c) - f(c) = 0$.  But this limit contradicts to our construction, in which $|f(x_{n_i}) - f(y_{n_i})| \geqslant \varepsilon$ for every $i$.  Therefore Theorem~\ref{thm:unif-cont} is again proved.

  \medskip
  Let us summarize these results into one theorem as follows.

  \begin{thm}[Fundamental theorem of continuous functions]
    Every continuous real valued function of a real variable $x \in [a,b]$ is bounded, achieves minimum, intermediate, and maximum values, and is uniformly continuous.
  \end{thm}


  \chapter{The Metric Topology of Euclidean Spaces}
\label{chap:metric_of_rn}

\section{Metric space}
\label{sec:metric}

It was an advance in geometry and analysis in the early 20th century to extract the essential properties needed for theorems to hold.
That was the birth of \emph{topology}, given by the German mathematician Felix Hausdorff's work \emph{Elements of set theory} in 1914.
We will need one of its branches, namely the metric space structure, in order to study analysis in Euclidean spaces.
A metric space is a proper generalization for Euclidean spaces, as we shall see now.

\begin{defn}
  A \textsf{metric space} $M$ is a set $M$, whose elements are referred as \emph{points}, equipped with a binary function $d: M \times M \to \mathbb{R}$ satisfying the following three properties:
  \begin{enumerate}[(1)]
    \item (Positivity) $d(x,y) \geqslant 0$ for any $x,y \in M$; $d(x,y) = 0$ if and only if $x=y$.
    \item (Symmetry) $d(x,y) = d(y,x)$ for any $x,y\in M$.
    \item (Triangle inequality) $d(x,y) \leqslant d(x,z) + d(z,y)$ for any $x, y, z \in M$.
  \end{enumerate}
  The function $d$ is usually referred as a \textsf{distance function} or \textsf{metric} on $M$.
  A metric space is denoted by $(M,d)$, sometimes simply by $M$ when the metric $d$ is understood.
\end{defn}

The following are examples of metric spaces.
\begin{enumerate}[(a)]
  \item Euclidean spaces $\mathbb{R}^n$.  When we say \emph{Euclidean spaces}, it is understood that $\mathbb{R}$ is endowed with the Euclidean metric:
    \[
      d(x,y) = \sqrt{ \sum_{k=1}^n |x_k - y_k|^2 }, \qquad
      x = (x_1,\dots, x_n), y = (y_1, \dots, y_n) \in \mathbb{R}^n.
    \]

  \item The trivial metric.  Every nonempty set $X$ can be given a trivial (or discrete) metric, defined as
    \[
      d(x,y) = 
      \begin{cases}
	0, & \text{if $x=y$} \\
	1, & \text{if $x\ne y$}
      \end{cases}, \qquad x, y \in X.
    \]
    This metric only tells whether two points coincide or not.
  \item Let $p > 0$.  Define a binary function $d_p : \mathbb{R}^n \times \mathbb{R}^n \to \mathbb{R}$ by
    \[
      d_p(x,y) = \left( \sum_{k=1}^n |x_k-y_k|^p \right)^{1/p}.
    \]
    Note that $p=2$ gives the usual Euclidean metric.
    
    It can be shown that
    \begin{enumerate}[(i)]
      \item $d_p$ is a metric on $\mathbb{R}^n$ if and only if $p \geqslant 1$.
      \item As $p \to \infty$, $d_p$ goes to
	\[
	  d_\infty(x,y) = \max \{ |x_1-y_1|, |x_2-y_2|, \dots, |x_n-y_n| \}.
	\]
	$d_\infty$ is also a metric on $\mathbb{R}^n$.
    \end{enumerate}
\end{enumerate}

We have defined the concept of limit for sequences in $\mathbb{R}$; its analog is easy for any metric space: we simply replace the absolute value in $\mathbb{R}$ by the metric on the metric space $M$.
So let us rephrase the definition for limit again.

\begin{defn}
  Let $M$ be a metric space.  A sequence $\langle a_n \rangle$ in $M$ is said to \textsf{converge} to a limit $p \in M$ if for any $\varepsilon > 0$ there is an integer $N \in \mathbb{N}$ such that
  \[
    n \geqslant N \implies d(a_n, p) < \varepsilon.
  \]
  In this case we write $a_n \to p$ (as $n \to \infty$.)

  If there is no $p \in M$ with the above property, we say that the sequence $\langle a_n \rangle$ \textsf{diverges} in $M$.
\end{defn}

It is \emph{mutatis mutandis} to show the uniqueness for limit of sequences in metric spaces from that in $\mathbb{R}$.
Also the concept of Cauchy sequences can also be introduced in metric spaces.
We now discuss subsequences of a sequence.

\begin{defn}
  Let $\langle a_n \rangle$ be a sequence in a metric space $M$.
  A \textsf{subsequence} of $\langle a_n \rangle$ is a sequence of the form $\langle a_{\varphi(k)} \rangle_k$, where $\varphi$ is a strictly increasing function from $\mathbb{N}$ to itself.
\end{defn}

Roughly speaking, a subsequence of a sequence is a sequence obtained by picking out infinitely many terms from the original sequence, while preserving their orders of appearance without repeating.

\begin{thm}
  Every subsequence of a convergent sequence in a metric $M$ converges and it converges to the same limit as does the mother sequence.
\end{thm}

The proof of this theorem is easy and we leave it to the readers as an exercise.
Nevertheless, this theorem is usually applied in the contrapositive way.
Namely, for a sequence $\langle a_n \rangle$, if it has a divergent subsequence, or it has two convergent subsequences that converge to different limits, then the original sequence diverges.
Another common way to state this theorem is that limits are unaffected when we pass to a subsequence.

\section{The Topology on Euclidean Space}
\label{sec:topology}

In real analysis we extract the essential properties on which others develop.
Although there are more general notions about (point set) topology, here we mainly deal with those that arise from metric spaces.
We start with a few definitions (cf.\ Rudin, Chapter 2, Definition 2.18).

\begin{defn}
  Let $(M, d)$ be a metric space.
  All points and sets mentioned below are understood to be elements and subsets of $M$.
  \begin{enumerate}[(a)]
    \item A \textsf{neighborhood} of a point $p$ is a set $B_r(p)$ consisting of all points $q$ such that $d(p,q) < r$ ($r > 0$).
      $B_r(p)$ is also called an \textsf{open ball} with center $p$ and radius $r$.

    \item A point $p$ is a \textsf{limit point} of $S$ if \textit{every} neighborhood of $p$ contains a point $q \ne p$ such that $q \in S$.

    \item If $p \in S$ but $p$ is not a limit point of $S$, then $p$ is called an \textsf{isolated point} of $S$.

    \item $S$ is \textsf{closed} if every limit point of $S$ belongs to $S$.

    \item A point $p$ is an \textsf{interior point} of $S$ if there is a neighborhood $N$ of $p$ such that $N \subseteq S$.  The collection of all interior points of $S$ is called the \textsf{interior} of $S$, and is denoted by $\mathring{S}$ or $\operatorname{int} S$.

    \item $S$ is \textsf{open} if every point of $S$ is an interior point of $S$.

    \item $S$ is \textsf{clopen} if $S$ is both closed and open.

    \item The \textsf{complement} of $S$ is the subset $S^c$ of all points that do not belong to $S$.

    \item A point $q$ is an \textsf{exterior point} of $S$ if there is a neighborhood $U$ of $q$ such that $U \subseteq S^c$, i.e., $U \cap S = \varnothing$.

    \item A point $b$ is called a \textsf{boundary point} of $S$ if it is neither an interior point nor an exterior point of $S$.
      The collection of all boundary points of $S$ is called the \textsf{boundary} of $S$, and is denoted by $\partial S$.
  \end{enumerate}
\end{defn}

With all these definitions, the most prominent ones are probably neighborhoods and limit points.
Especially, a point $p$ is {\em not} a limit point of a subset $S \subseteq M$ if and only if there is an $r > 0$ such that $B_r(p) \cap S \subseteq S$.
Two points are made here.  Firstly some books regard a neighborhood of a point $p$ is simply an open set $U$ that contains the point $p$.  But in this case we can find a smaller open ball $B_r(p)$ that is contained in $U$. 
Secondly our textbook (authored by Pugh) define a limit of a set to be a limit point or an isolated point of that set.
Since most literature does not agree with Professor Pugh, we simply leave him alone.  Nevertheless you should be careful when reading Pugh's texts.  (This is not an attack against him.)

Now we come to several properties that follow from these definitions.
First we need to explain the notion of an ``open ball.''

\begin{prop}
  An open ball in a metric space $M$ is always an open set.
\end{prop}

\begin{proof}
  Let $B_r(p)$ be an open ball, where $p \in M$ and $r > 0$.
  We need to show that every point $x \in B_r(p)$ is an interior point of $B_r(p)$.
  Let $x \in B_r(p)$; by definition we have $d(x,p) < r$.
  Take $\rho = r - d(x,p) > 0$.  We claim that $B_\rho(x) \subseteq B_r(p)$.

  Let $y$ be an arbitrary point of $B_\rho(x)$; by definition we have $d(x,y) < \rho$.
  Then by the triangle inequality,
  \[
    d(y,p) \leqslant d(y,x) + d(x,p) < \rho + d(x,p) = (r - d(x,p)) + d(x,p) = r,
  \]
  that is, $y \in B_r(p)$.
  Since $y$ is arbitrary, we conclude that $B_\rho(x) \subseteq B_r(p)$.
\end{proof}

Since every point $x$ in an open set is its interior point, there is an open ball centered at $x$ that falls in the open set completely.
We use this property to show statements about unions and intersection of open sets.

\begin{prop}
  \label{open-union-intersection}
  Let $M$ be a metric space.
  \begin{enumerate}[$(a)$]
    \item Let $\{ U_\alpha \colon \alpha \in A \}$ be an arbitrary collection of open sets in $M$ ($A$ is just some index set.)
      Then their union $U = \displaystyle \bigcup_{\alpha \in A} U_\alpha$ is open in $M$.

    \item Let $U_1, U_2, \dots, U_N$ be a {\em finite} collection of open sets in $M$.
      Then their intersection $V = \displaystyle \bigcap_{i=1}^N U_i$ is open in $M$.
  \end{enumerate}
\end{prop}

\begin{proof}
  \begin{enumerate}[(a)]
    \item Let $p$ be an arbitrary point of $U$.  Then there is some index $\alpha \in A$ such that $p \in U_\alpha$.
      Since $U_\alpha$ is open, there is an $r > 0$ such that $B_r(p) \subseteq U_\alpha \subseteq U$.
      This shows that $p$ is also an interior point of $U$.
      Since $p$ is arbitrary, we conclude that $U$ is open in $M$.

    \item Let $q$ be an arbitrary point of $V$.  Then $q \in U_i$ for each $i = 1, 2, \dots, N$.
      For each $i$, there is an $r(i) > 0$ such that $B_{r(i)}(q) \subseteq U_i$.
      Taking $r = \min \{ r(1), r(2), \dots, r(N) \} > 0$, we see that $B_r(q) \subseteq B_{r(i)}(q) \subseteq U_i$ for any $i$.
      Hence $B_r(q) \subseteq V$, i.e., $q$ is an interior point of $V$.
      Since $q$ is arbitrary, we see that $V$ is open in $M$.
  \end{enumerate}
\end{proof}

Proposition~\ref{open-union-intersection} says that the collection of all open sets in $M$ is closed under arbitrary union and finite intersection.\footnote{The word {\em closed} has been abused.  There are so many occasions in mathematics in which we use this word to mean something like closure.  But really it takes a while to make necessary distinction about this word that appears in different places.  In this sentence the word ``closed'' already has a different meaning from a set being closed in a metric space.}
The situation will be totally different when we talk about closed sets.
But first let us introduce the following result.

\begin{thm}
  \label{thm:open-closed}
  Let $M$ be a metric space.
  A subset $U$ of $M$ is open if and only its complement $C = U^c$ is closed.
\end{thm}

\begin{proof}
  Let us assume first that $U$ is open.
  If a point $x \in U$ then there is a neighborhood $B_r(x)$ of $x$ that falls completely in $U$.  But this implies that $x$ cannot be a limit point of $C = U^c$, since there is no sequence in $C$ that can converge to $x$.  Hence any limit point of $C$ must belong to $C$, by definition $C$ is closed.

  Now suppose $C = U^c$ is closed and $x \in U$.
  Since $x \notin C$, $x$ is not a limit point of $C$.
  Therefore there is an $r > 0$ such that $B_r(x) \cap C = \varnothing$.
  But this is exactly $B_r(x) \subseteq U$.
  Since $x$ is arbitrary, $U$ must be an open set.
\end{proof}

The choice of the words ``open'' and ``closed'' was unfortunate.
Unlike a door which must be either open or closed, a subset of a metric space may be open, closed, neither, or both.
For example an interval of the form $(a,b]$ in $\mathbb{R}$ is neither open nor closed.
It is important to remember that, to establish a set being open, one should prove the closedness of its complement rather than itself.

Combining Proposition~\ref{open-union-intersection}, Theorem~\ref{thm:open-closed} and De Morgan's Law, we immediately reach the following result.

\begin{prop}
  \label{closed-union-intersection}
  Let $M$ be a metric space.
  \begin{enumerate}[$(a)$]
    \item Let $\{ C_\alpha \colon \alpha \in A \}$ be an arbitrary collection of closed sets in $M$ ($A$ is just some index set.)
      Then their intersection $C = \displaystyle \bigcap_{\alpha \in A} C_\alpha$ is closed in $M$.

    \item Let $C_1, C_2, \dots, C_N$ be a {\em finite} collection of closed sets in $M$.
      Then their union $D = \displaystyle \bigcup_{i=1}^N C_i$ is closed in $M$.
  \end{enumerate}
\end{prop}

The following proposition holds trivially.

\begin{prop}
  \label{empty-universal-clopen}
  In any metric space $M$, the empty set $\varnothing$ and the whole space $M$ are both clopen.
\end{prop}

Indeed there is a more general notion of {\em topology.}\footnote{A \textsf{topology} on a set $M$ is a collection $\mathcal T$ of subsets of $M$, whose elements are called {\em open sets}, that is closed under arbitrary union, finite intersection, and contains $\varnothing$ and $M$.  In this scenario ``open set'' is introduced earlier than ``open ball'', which is only available for metric space.  Then a {\em closed set} is a subset of $M$ whose complement is open, that is, belongs to $\mathcal T$.  Theorem~\ref{thm:open-closed} becomes a definition!}
Please refer to standard textbooks such as Munkres' {\em Topology, A First Course}.

From Definition~1(e), it is clear that the interior $\mathring{S}$ is the largest open subset that is contained in $S$, in the sense that whenever $U$ is an open set and $U \subseteq S$, $U \subseteq \mathring{S}$.
Also from Definition~1(i) the collection of all exterior points of $S$ forms an open set, by a similar argument. 
On the other hand, the union of a subset and the set of all its limit points also has some significant properties.

\begin{defn}
  Let $S$ be a subset of a metric space $M$.
  Denote by $S'$ the set of all limit points of $S$.
  The \textsf{closure} of $S$ is defined to be the union $S \cup S'$, and is denoted by $\overline{S}$.
\end{defn}

\begin{thm}
  Let $M$ be a metric space, and $S$ be a subset of $M$.
  The closure $\overline{S}$ of $S$ is a closed set.
  In fact, it is the smallest closed set in $M$ that contains $S$, in the sense that whenever $K$ is a closed set and $K \supseteq S$, $K \supseteq \overline{S}$ as well.
\end{thm}

\begin{proof}
  Let $p \notin \overline{S} = S \cup S'$.
  Since $p$ is neither a point in $S$ nor a limit point of $S$, there is an $r > 0$ such that $B_r(p) \cap S = \varnothing$, i.e., $p$ is an exterior point of $S$.
  Conversely, if a point $p$ is an exterior point of $S$, $p$ is neither a point in $S$ nor a limit point of $S$.
  Therefore $\overline{S}$ is a closed set since its complement is the open set consisting of all exterior points of $S$.

  Now assume that $K$ is a closed set and $K \supseteq S$; we need to show that $K \supseteq S'$ as well.
  Let $p$ be a limit point of $S$.
  By definition $p$ is also a limit point of $K$, since every neighborhood of $p$ contains a point $q \ne p$ and $q \in S \subseteq K$.
  Since $K$ is closed, $p \in K$, and the proof is completed.
\end{proof}

Let us state an important property about the closure of a set.

\begin{thm}
  Let $M$ be a metric space, and $S$ be a subset of $M$.
  A point $x \in \overline{S}$ if and only if there is a sequence $\langle x_n \rangle$ in $S$ that converges to $x$.
\end{thm}

\begin{proof}
  ($\Rightarrow$): Since $\overline{S} = S \cup S'$, there are two possible cases.
  \begin{enumerate}
    \item If $x \in S$, then we can simply take the constant sequence $\langle x \rangle$, which converges to $x$ trivially.

    \item Suppose $x \in S'$.  For each $n \in \mathbb{N}$, $B_{1/n}(x)$ is a neighborhood of $x$.  Since $x$ is a limit point of $S$, there must be a point $x_n \in B_{1/n}(x) \cap S$ with $x_n \ne x$.  It is now clear that the sequence $\langle x_n \rangle$ in $S$ does converge to $x$.
  \end{enumerate}

  ($\Leftarrow$) Conversely, let us assume that there is a sequence $\langle x_n \rangle$ in $S$ that converges to $x$ but $x \notin S$; we must show that $x \in S'$.  For any $r > 0$, $B_r(x)$ is a neighborhood of $x$.  Since $x_n \to x$ as $n \to \infty$, there is an integer $N \in \mathbb{N}$ such that whenever $n \geqslant N$ we have $d(x_n,x) < r$.  Moreover, none of these $x_N, x_{N+1}, \dots$ can be the point $x$ because they belong to $S$ but $x$ is assumed not to be there.  Therefore, we have found (infinitely many) points $x_N, x_{N+1}, \dots$ in $S$ and in the neighborhood $B_r(x)$, none of which is $x$.  This shows that $x$ is indeed a limit point of $S$, i.e., $x \in S'$.
\end{proof}

Finally we talk about some properties of boundary.
Given a subset $S \subseteq M$, the space is partitioned into three subsets: $\mathring{S}$, the subset of exterior points of $S$, and the boundary $\partial S$.  Since the former two sets are open, the last $\partial S$ is always a closed set.
Furthermore, the following statement holds straight from Definition~1(j).

\begin{prop}
  Let $S$ be a subset of a metric space $M$.
  A point $p$ lies in the boundary $\partial S$ of $S$ if and only if every neighborhood of $p$ contains a point in $S$ and another point not in $S$.
\end{prop}

  \section{Continuous functions}
\label{sec:continuous}

With the language of sequences in hand, it is now time to state the definition of continuous functions or mappings.

\begin{defn}
  Let $M, N$ be two metric spaces, and $f : M \to N$ be a function.
  The function $f$ is said to be \textsf{continuous} if it preserves sequential convergence: $f$ sends convergent sequences in $M$ to convergent sequences in $N$, limits being sent to limits.
  That is, for each sequence $\langle p_n \rangle$ in $M$ which converges to a limit $p$ in $M$, the image sequence $\langle f(p_n) \rangle$ converges to $f(p)$ in $N$.
\end{defn}

There are two obvious continuous functions.  Namely, the identity function $\operatorname{id}_M: M \to M$ and any constant function $f: M \to N$.
We omit their proofs here.

It is easy to see that the composite of two continuous functions are still continuous.

\begin{thm}
  Let $f: M \to N$ and $g: N \to P$ be continuous functions among three metric spaces $M, N$, and $P$.  Then their composition $g \circ f : M \to P$ is also continuous.
\end{thm}

\begin{proof}
  Let $\langle p_n \rangle$ be a sequence in $M$ that converges to $p$.
  Then $\langle f(p_n) \rangle$ converges to $f(p)$ in $N$, since $f$ is continuous.
  This in turn implies that $\langle g(f(p_n)) \rangle$ converges to $g(f(p))$ in $P$ since $g$ is also continuous, as required.
\end{proof}

The following definition of continuity sounds more (un-)familiar: it uses Greek letters.
We state it as a theorem, but really it is equivalent to what we just defined.

\begin{thm}
  Let $(M, d_M), (N, d_N)$ be metric spaces.
  A function $f: M \to N$ is continuous if and only if it satisfies the \textbf{$(\varepsilon,\delta)$-condition}: For each $\varepsilon > 0$ and each $p \in M$ there exists a $\delta > 0$ such that if $x \in M$ and $d_M(x,p) < \delta$ then $d_N\left( f(x), f(p) \right) < \varepsilon$.
\end{thm}

\begin{proof}
  For the forward direction, let us assume that $f$ preserves sequential limits.
  If the conclusion is false, then there are $\varepsilon > 0$ and $p \in M$ such that, for each $\delta > 0$, there is a point that violates the condition.
  Let us consider $\delta = 1, \frac12, \frac13, \dots, \frac1n, \dots$.
  For each $n \in \mathbb{N}$, let $x_n$ be some point in $M$ such that $d_M(x_n, p) < 1/n$ but $d_N( f(x_n), f(p) ) \geqslant \varepsilon$. 
  It is clear that $x_n \to p$ in $M$ but $f(x_n)$ cannot converge to $f(p)$, which is a contradiction.
  Therefore $f$ must satisfy the $(\varepsilon, \delta)$-condition.

  On the other hand, let us assume that $f$ satisfies the $(\varepsilon, \delta)$-condition.
  Take a sequence $\langle p_n \rangle$ in $M$ that converges to $p$ in $M$.
  For any $\varepsilon > 0$, there is a $\delta > 0$ (we have a point $p$ already) such that $d_N(f(x), f(p)) < \varepsilon$ whenever $x \in M$ and $d_M(x,p) < \delta$.
  For this $\delta > 0$, there is an integer $N \in \mathbb{N}$ such that $d_M( x_n, p ) < \delta$ whenever $n \geqslant N$.
  Piecing these informations together, we see that whenever $n \geqslant N$,
  \[
    d_M(x_n, p) < \delta \implies d_N \left( f(x_n), f(p) \right) < \varepsilon.
  \]
  This is exactly the definition for $\langle f(x_n) \rangle$ converges to $f(p)$ in $N$, 
  and we are done.
\end{proof}
A property of a metric space or of a mapping between metric spaces that can be described solely in terms of open sets (or equivalently, in terms of closed sets) is called a \textit{topological property}.
Our next result describes continuity of mappings topologically.

Firstly we introduce a notion.
Let $f : M \to N$ be a function/mapping.
For any $n \in N$, the \textsf{preimage} of $n$ under $f$ is the subset
\[
  f^{-1}(n) = \{ m \in M \colon f(m) = n \}
\]
of $M$.\footnote{Our textbook uses $f^{\text{pre}}$ instead of $f^{-1}$.  I assume that Pugh wanted to preserve the notation $f^{-1}$ to mean the inverse function of $f$.  Nevertheless we could regard $f^{-1}$ as a \textit{set-valued} mapping.}
Note that $f^{-1}(n) = \varnothing$ when $n$ is not in the image of $f$; also that $f$ is an injection if and only if $|f^{-1}(n)| \leqslant 1$.
We also use the notation
\[
  f^{-1}(A) = \{ m \in M \colon f(m) \in A \}
\]
for any subset $A \subseteq N$.
Here is our statement.

\begin{thm}
  The following are equivalent for continuity of $f : M \to N$.
  \begin{enumerate}[(i)]
    \item The $(\varepsilon,\delta)$-condition.
    \item The sequential convergence preservation condition.
    \item The \textbf{closed set condition}: The preimage $f^{-1}(C)$ of each closed set $C$ in $N$ is closed in $M$.
    \item The \textbf{open set condition}: The preimage $f^{-1}(U)$ of each open set $U$ in $N$ is open in $M$.
  \end{enumerate}
\end{thm}

\begin{proof}
  We have shown the equivalence (i) $\iff$ (ii).
  We hereby deal with other implications.

  \smallskip
  \noindent\textbf{(ii) $\implies$ (iii)}.
  Let $C$ be any closed set in $N$, and $p$ be a limit point of $f^{-1}(C)$.
  By definition there is a sequence $\langle a_n \rangle$ in $M$ such that converges to $p \in M$ and $f(a_n) \in C$ for all $n$.
  By (ii) $f$ preserves sequential convergence, hence $\langle f(a_n) \rangle$ converges to $f(p) \in N$.
  Since $f(a_n) \in C$ for all $n$ and $C$ is a closed set, we get $f(p) \in C$.
  This is saying that $p \in f^{-1}(C)$.
  Since $p$ is arbitrary, we see that $f^{-1}(C)$ contains all of its limit points, hence $f^{-1}(C)$ is a closed set in $M$.

  \smallskip
  \noindent\textbf{(iii) $\iff$ (iv)}.  This follows from Theorem 4 in Topic 8 and taking the complement: $$\left( f^{-1}(X) \right)^c = f^{-1}(X^c)$$ for any subset $X \subseteq N$.

  \smallskip
  \noindent\textbf{(iv) $\implies$ (i)}.
  Let $p$ be a point in $M$ and a positive number $\varepsilon > 0$ be given.
  Then the open ball $B_\varepsilon(f(p))$ is an open ball in $N$.
  By (iv) the set $U = f^{-1}(B_\varepsilon(f(p)))$ is open in $M$, and $p$ belongs to $U$.
  Since $U$ is open, $p$ is an interior point of $U$, i.e., there is a $\delta > 0$ such that $B_\delta(p) \subseteq U = f^{-1}(B_\varepsilon(f(p))$.
  This is saying that $f(B_\delta(p)) \subseteq B_\varepsilon(f(p))$.
  Translating this relation, it means that for any $x \in M$,
  \[
    d_M(x,p) < \delta \implies d_N(f(x),f(p)) < \varepsilon,
  \]
  which is exactly the $(\varepsilon, \delta)$-condition!

  \smallskip
  The proof is now complete.
\end{proof}

Note that no explicit mention is made of the metric in the closed and open set chararacterization for continuous functions.
The open set condition is purely topological.
It would be perfectly valid to take the open set characterization as a \textit{definition} of continuity; in fact this is how it is done in general topology.
With this, it is natural to introduce the following definition.

\begin{defn}
  Two metric spaces (or topological spaces) $M$ and $N$ are said to be \textsf{homeomorphic} if and only if there is a bijection $f: M \to N$ such that $f$ is continuous from $M$ to $N$, while $f^{-1}$ is continuous from $N$ to $M$.  (We could also say that the bijection $f$ is \textit{bicontinuous}.)  In this case $f$ is called a \textsf{homeomorphism} between $M$ and $N$. 
\end{defn}

In the mathematical terms, any word that ends in ``morphism'' usually means ``preserving the structure.''
A homeomorphism between $M$ and $N$ preserves the topological structures on both sides; it establishes a one-to-one correspondence between open sets in $M$ and open sets in $N$.
Two spaces that are homeomorphic to each other are really the same thing, but maybe under different names.
We see this phenomenon in other areas of mathematics.
For example, the same algebraic structures on two sets are called \textit{isomorphic}.

\medskip
\noindent\textbf{\large Subspace topology}

If a set $S$ is contained in a metric subspace $N \subseteq M$ you need to be careful when you say that $S$ is open or closed.
For example,
\[
  S = [0,1) = \{ x \in \mathbb{R} \colon 0 \leqslant x < 1 \}
\]
is not an open set in $\mathbb{R}$, but it is an open set in $[0,1]$, as we shall see.
The way to see this clearly is through the notion of \textit{inheritance}, or the so-called \textit{subspace topology}.

\begin{defn}
  Let $M$ be a metric space and $N$ be a subset of $M$.
  We say that $N$ \textsf{inherits its topology} from $M$ in the sense that each subset $V \subseteq N$ which is open in $N$ is actually the intersection $V = N \cap U$ for some $U \subseteq M$ that is open in $M$, and vice versa.
  In this situation we say that $N$ is equipped with the \textsf{subspace topology} from $M$.
\end{defn}

Here, the metric on $N$ is \textit{inherited} from $M$: we simply use the metric on $M$ for points on $N$.
Let us see why this notion of open sets in a subspace makes sense.
Indeed, for $p \in N \subseteq M$ and $r > 0$,
\[
  B_N(p,r) = \{ x \in N \colon d(x,p) < r \} = B_M(p,r) \cap N.\footnote{Here we use $B_N(p,r)$ to denote the open ball centered at $p$ with radius $r > 0$ in the metric space $N$.  I hope that this notation is self-evident.}
\]
Let $V$ be an open set in $N$.
For each $p \in N$, there is an $r_p > 0$ such that $B_N(p, r_p) \subseteq V$.
Set
\[
  U = \bigcup_{p \in V} B_M(p, r_p).
\]
It is clear that $U$ is an open set in $M$ and $V = U \cap N$.
The opposite direction is also clear.
The same can be said for closed sets in a subspace.
We state it for future reference if needed.

\begin{prop}
  Let $N$ be a subspace of a metric space $M$, and $N$ be equipped with the subspace topology.
  Then $C$ is a closed subset of $N$ if and only if there is a closed subset $K$ of $M$ such that $C = K \cap N$.
\end{prop}

\noindent\textbf{\large Product metric}

We next define a metric on the Cartesian product $M = X \times Y$ of two metric spaces $X$ and $Y$.
There are three natural ways to do so:
\begin{align*}
  d_E(p,p') &= \sqrt{ d_X(x,x')^2 + d_Y(y,y')^2 } \\
  d_{\operatorname{max}}(p,p') &= \max \{ d_X(x,x'), d_Y(y,y') \} \\
  d_{\operatorname{sum}}(p,p') &= d_X(x,x') + d_Y(y,y')
\end{align*}
where $p=(x,y)$, $p'=(x',y')$ belong to $M = X \times Y$.
The following result is immediate.

\begin{prop}
  \label{prop:product-metric}
  We have $d_{\operatorname{max}} \leqslant d_E \leqslant d_{\operatorname{sum}} \leqslant 2 d_{\operatorname{max}}$.
\end{prop}

In fact, the topologies induced by these three metrics on the product space are identical in the following sense.

\begin{thm}
  Let $U$ be a subset of a product space $M = X \times Y$ of two metric spaces.
  If $U$ is open under anyone of the three metrics, it is also open under the other two metrics.
\end{thm}

\begin{proof}
  Let us show one of the implications; others follow in a similar fashion.
  Suppose that $U$ is an open set under $d_E$.
  For any $p \in U$, there is an $r > 0$ such that $B_{E,r}(p) \subseteq U$ (the notation $B_{E,r}(p)$ is self clear.)
  Since $d_E \leqslant d_{\operatorname{sum}}$, we see that $d_E(p,p') < r$ whenever $d_{\operatorname{sum}}(p,p') < r$ for any $p' \in M$, i.e., $B_{\operatorname{sum},r}(p) \subseteq B_{E,r}(p)$.
  Therefore $p \in B_{\operatorname{sum},r}(p) \subseteq B_{E,r}(p) \subseteq U$ for that $r > 0$.
  Since $p$ is arbitrary, we conclude that $U$ is also an open set under $d_{\operatorname{sum}}$.
\end{proof}

From Proposition~5, we see that the notion of convergence in a product space prevails under these three different metrics, as the following theorem states; its proof is omitted here.

\begin{thm}
  \label{thm:product-convergence}
  The following are equivalent for a sequence $\langle p_n = (p_{1n}, p_{2n}) \rangle$ in $M = X \times Y$:
  \begin{enumerate}[(a)]
    \item $\langle p_n \rangle$ converges with respect to the metric $d_{\operatorname{max}}$.
    \item $\langle p_n \rangle$ converges with respect to the metric $d_{E}$.
    \item $\langle p_n \rangle$ converges with respect to the metric $d_{\operatorname{sum}}$.
    \item $\langle p_{1n} \rangle$ and $\langle p_{2n} \rangle$ converge in $X$ and $Y$, respectively.
    
    
  \end{enumerate}
\end{thm}

By repeating the products, the following result is useful and sometimes is forgotten when used.

\begin{cor}
  A sequence of vectors $\langle v_n \rangle$ in $\mathbb{R}^m$ converges in $\mathbb{R}^m$ if and only if each of its component sequences $\langle v_{in} \rangle_n$ converges, $1 \leqslant i \leqslant m$.
  The limit of the vector sequence is the vector
  \[
    v = \lim_{n \to \infty} v_n = \left( \lim_{n \to \infty} v_{1n}, \lim_{n \to \infty} v_{2n}, \dots, \lim_{n \to \infty} v_{mn}  \right).
  \]
\end{cor}

The distance function $d: M \times M \to \mathbb{R}$ is a function from the product space $M \times M$ to the metric space $\mathbb{R}$, so the following assertion makes sense.

\begin{thm}
  The distance function $d: M \times M \to \mathbb{R}$ is continuous.
\end{thm}

\begin{proof}
  We check $(\varepsilon, \delta)$-continuity with respect to the metric $d_{\operatorname{sum}}$ on $M \times M$.
  Given $\varepsilon > 0$ we take $\delta = \varepsilon$.
  If $d_{\operatorname{sum}}( (p,q), (p',q') ) < \delta$ then the triangle inequality gives
  \begin{align*}
    d(p,q)   &\leqslant d(p,p') + d(p',q') + d(q',q) < d(p',q') + \varepsilon, \\
    d(p',q') &\leqslant d(p',p) + d(p,q) + d(q,q') < d(p,q) + \varepsilon,
  \end{align*}
  which gives
  \[
    d(p,q) - \varepsilon < d(p',q') < d(p,q) + \varepsilon.
  \]
  Thus $|d(p',q') - d(p,q)| < \varepsilon$ and we get continuity with respect to the metric $d_{\operatorname{sum}}$.
  By Theorem~\ref{thm:product-convergence} it does not matter which metric we use on $M \times M$.
\end{proof}

\medskip
\noindent\textbf{\large Completeness}

In an earlier class we discussed the Cauchy criterion for convergence of a sequence of real numbers.
There is a natural way to carry these ideas over to a metric space $M$.
A sequence $\langle p_N \rangle$ in $M$ satisfies a \textsf{Cauchy condition} provided that for each $\varepsilon > 0$ there is an integer $N \in \mathbb{N}$ such that for all $k, n \geqslant N$ we have $d(p_k, p_n) < \varepsilon$, and $\langle p_n \rangle$ is said to be a \textsf{Cauchy sequence}.  In symbols,
\[
  \forall \varepsilon > 0 \,\, \exists N \text{ such that } k, n \geqslant N \implies d(p_k, p_n) < \varepsilon.
\]

Under this notion, all properties of Cauchy sequences in $\mathbb{R}$ carry to general metric spaces, except that a Cauchy sequence in a metric space may or may not converge.
In fact, we have the following definition.

\begin{defn}
  A metric space $(M,d)$ is \textsf{complete} if every Cauchy sequence in $M$ converges to a limit in $M$.
\end{defn}

Under this definition, $\mathbb{R}$ and $\mathbb{R}^m$ are complete metric spaces, while $\mathbb{Q}$ is not.
Now there is a nice property about closed subsets of such spaces.

\begin{thm}
  Every closed subset of a complete metric space is a complete metric space.
\end{thm}

\begin{proof}
  Let $A$ be a closed subset of the complete metric space $M$ and let $\langle p_n \rangle$ be a Cauchy sequence in $A$ with respect to the inherited metric.
  It is of course a Cauchy sequence in $M$ so it converges to a limit $p$ in $M$.
  Since $A$ is closed we have $p \in A$, i.e., the Cauchy sequence $\langle p_n \rangle$ converges to a point $p \in A$.
  This shows that $A$ is also complete.
\end{proof}

We note that completeness is \textit{not} a topological property.
For example, consider $\mathbb{R}$ with its usual metric and $(-1,1)$ with the metric it inherited from $\mathbb{R}$.
Although they are homeomorphic metric spaces, $\mathbb{R}$ is complete but $(-1,1)$ is not.

There is a general construction to complete a general metric space.
Please consult Chapter 2, Section 10 of Pugh's textbook.

  \section{Seqential Compactness}
\label{sec:seq-comp}

Compactness is the single most important concept in real analysis.
It is what reduces the infinite to the finite.

\begin{defn}
  A subset $A$ of a metric space $M$ is \textsf{sequentially compact} if every sequence $\langle a_n \rangle$ in $A$ has a subsequence $\langle a_{n_k} \rangle$ that converges to a limit in $A$.
\end{defn}

The empty set and finite sets are trivial examples of sequentially compact sets.
For a sequence $\langle a_n \rangle$ contained in a finite set repeats a term infinitely often, and the corresponding constant subsequence converges.

Let us first derive some properties that every compact set enjoys.

\begin{thm}
  \label{thm:cpt-closed-bounded}
  Every sequentially compact set is closed and bounded.
\end{thm}

\begin{proof}
  Let $A$ be a sequentially compact set in a metric space $M$.
  Let us show that $A$ is a closed set first.
  Suppose that $p \in M$ is a limit point of $A$,
  that is, there is a sequence $\langle a_n \rangle$ in $A$ that converges to $p$.
  Since $A$ is sequentially compact, there is some subsequence of $\langle a_n \rangle$ that converges to $q \in A$.
  But the whole sequence $\langle a_n \rangle$ already converges, hence it converges to $q$ as well.
  By the uniqueness of limit, we see that $p = q \in A$.
  Thus $A$ is closed.

  To see why $A$ must be bounded, we suppose that $A$ is unbounded to begin with.
  Fix a point $m \in M$, there is a point $a_n \in A$ such that $d(a_n, m) \geqslant n$ for every $n \in \mathbb{N}$.
  Since $A$ is sequentially compact, $\langle a_n \rangle$ has a subsequence $\langle a_{n_k} \rangle$ that converges in $A$.
  Being a convergent sequence, $\langle a_{n_k} \rangle$ is a bounded sequence.  But this contradicts to the construction that $d(m,a_{n_k}) \geqslant n_k \geqslant k \to \infty$ as $k \to \infty$.
  Hence $A$ must be a bounded set.
\end{proof}

The converse of Theorem~\ref{thm:cpt-closed-bounded} in the Euclidean space $\mathbb{R}^n$ is also true, as we shall see.
But its proof is much harder.
We will break the proof into several steps.
The following set is the starting point.

\begin{thm}
  \label{thm:bounded-closed-interval-cpt}
  Any bounded closed interval $[a,b] \subseteq \mathbb{R}$ is sequentially compact.
\end{thm}

\begin{proof}
  Let $\langle x_n \rangle$ be a sequence in $[a,b]$ and set
  \[
    C = \{ x \in [a,b] \colon x_n < x \text{ only finitely often} \}.
  \]
  Equivalently, for all but finitely many $n$, $x_n \geqslant x$.
  Since $a \in C \ne \varnothing$ and $b$ is an upper bound for $C$, the supremum $c = \sup C \in [a,b]$ by the least upper bound property.
  We claim that there is a subsequence of $\langle x_n \rangle$ that converges to $c$.
  If there is no subsequence $\langle x_n \rangle$ that converges to $c$, then there is an $r > 0$ such that the interval $(c - r, c + r)$ only contains a finite number of terms in $\langle x_n \rangle$. 
  Since there is an $r' \in \mathbb{R}$ such that $0 \leqslant r' < r$ and $c - r' \in C$, we see that $c + r \in C$ as well, contradicting to $c = \sup C$.
  Therefore the claim holds and our proof is complete.
\end{proof}

By taking Cartesian products and the fact that a sub-subsequence of a sequence is a subsequence itself, we have the following results.

\begin{thm}
  \label{thm:cpt-product}
  The Cartesian product of $m$ sequentially compact sets is sequentially compact for every $m \in \mathbb{N}$, $m \geqslant 2$.
\end{thm}

\begin{proof}
  The base case $m = 2$ holds by the comment before the theorem.
  The rest is true by mathematical induction on $m$.
\end{proof}

\begin{cor}
  Every \textsf{box} $[a_1, b_1] \times [a_2, b_2] \times \cdots \times [a_m, b_m]$ in $\mathbb{R}^m$ is sequentially compact.
\end{cor}

Actually this corollary is equivalent to the following named theorem.

\begin{thm}[Bolzano-Weierstrass theorem]
  \label{thm:bw}
  Every bounded sequence in $\mathbb{R}^m$ has a convergent subsequence.
\end{thm}

\begin{proof}
  A bounded sequence is contained in a box, which is sequentially compact, and therefore the sequence must have a subsequence that converges to a limit in that box. 
\end{proof}

Here is a simple fact about sequential compacts.
\begin{thm}
  \label{thm:cpt-closed-subset}
  Every closed subset of a sequentially compact set is also sequentially compact.
\end{thm}

\begin{proof}
  Let $A$ be a closed subset of a sequentially compact set $K$.
  Pick a sequence $\langle a_n \rangle$ in $A$.
  Viewing it as a sequence in $K$, it has a subsequence $\langle a_{n_k} \rangle$ that converges to a point $p \in K$. 
  Since $A$ is closed, $p \in A$ as well.
  This shows that $A$ is sequentially compact.
\end{proof}

Now we come to the first partial converse to Theorem~\ref{thm:cpt-closed-bounded}.

\begin{thm}[Heine-Borel theorem]
  Every closed and bounded subset of $\mathbb{R}^m$ is sequentially compact.
\end{thm}

\begin{proof}
  Let $A \subseteq \mathbb{R}^m$ be closed and bounded.
  Boundedness implies that $A$ is contained in some box, which is sequentially compact.
  Since $A$ is closed, Theorem~\ref{thm:cpt-closed-subset} implies that $A$ is also sequentially compact.
\end{proof}

The Heine-Borel theorem states that closed and bounded subsets of Euclidean spaces are sequentially compact, but it is vital to remember that a closed and bounded subset of a general metric space may fail to be sequentially compact.
For example, the set $\mathbb{N}$ of natural numbers equipped with the discrete metric is a complete metric space, it is closed in itself, and it is bounded.
But it is not sequentially compact.
After all, what subsequence $\langle 1, 2, 3, \dots \rangle$ converges?

Next we discuss how sequentially compact sets behave under continuous transformations.

\begin{thm}
  \label{thm:cont-cpt}
  If $f : M \to N$ is continuous and $A$ is a sequentially compact subset of $M$ then $f(A)$ is a sequentially compact subset of $N$.
  That is, a continuous function sends sequentially compact sets to sequentially compact sets.
\end{thm}

\begin{proof}
  Suppose that $\langle b_n \rangle$ is a sequence in $f(A)$.
  For each $n \in \mathbb{N}$ choose a point $a_n \in A$ such that $f(a_n) = b_n$.
  Since $A$ is sequentially compact there is a subsequence $\langle a_{n_k} \rangle$ that converges to some point $a \in A$.
  By continuity of $f$ it follows that $\langle b_{n_k} \rangle = \langle f(a_{n_k}) \rangle \to f(p) \in f(A)$ as $k \to \infty$.
  Thus, every sequence $\langle b_n \rangle$ in $f(A)$ has a subsequence that converges to a limit in $f(A)$, which shows that $f(A)$ is sequentially compact.
\end{proof}

From Theorem~\ref{thm:cont-cpt} follows the natural generalization of the min/max theorem before which concerns continuous real-valued functions defined an interval $[a,b]$.

\begin{cor}
  A continuous real-valued function defined on a nonempty, sequentially compact set is bounded; it assumes maximum and minimum values.
\end{cor}

\begin{proof}
  Let $f: M \to \mathbb{R}$ be continuous and $A$ be a nonempty, sequentially compact subset of $M$.
  By Theorem~\ref{thm:cont-cpt} $f(A)$ is a sequentially compact subset of $\mathbb{R}$.
  By Theorem~\ref{thm:cpt-closed-bounded} $f(A)$ is bounded and closed.
  Being a nonempty and bounded set in $f(A)$, its supremum and infimum both exist.
  Since they are limits of $f(A)$, they belong to $f(A)$ as well because $f(A)$ is closed. 
\end{proof}

A homeomorphism is a bicontinuous bijection.
Thus sequentially compactness is a topological property, as stated below; it follows directly from Theorem~\ref{thm:cont-cpt}.

\begin{thm}
  If $M$ is sequentially compact and $M$ is homeomorphic to $N$ then $N$ is sequentially compact.
  Sequential compactness is a topological property.
\end{thm}

Interestingly, the assumption on bicontinuity between sequential compacts seems superfluous.

\begin{thm}
  If $f: M \to N$ is a continuous bijection and $M$ is sequentially compact, then $f$ is a homeomorphism between $M$ and $N$.
  That is, its inverse mapping $f^{-1}: N \to M$ is automatically continuous.
\end{thm}

\begin{proof}
  Suppose that $q_n \to q$ in $N$.
  Since $f$ is a bijection, $p_n = f^{-1}(q_n)$ and $p = f^{-1}(q)$ are well-defined points in $M$.
  To check continuity of $f^{-1}$ we must show that $p_n \to p$.

  If $\langle p_n \rangle$ refuses to converge to $p$, then there are a subsequence $\langle p_{n_k}\rangle$ and a $\delta > 0$ such that $d(p_{n_k}, p) \geqslant \delta$. 
  Sequential compactness of $M$ gives a further subsequence which we still write $\langle p_{n_\ell} \rangle$ that converges to a point $p^* \in M$ as $\ell \to \infty$.
  Necessarily, $d(p,p^*) \geqslant \delta$, which implies $p \ne p^*$.

Since $f$ is continuous we have $f(p_{n_\ell}) \to f(p^*)$ as $\ell \to \infty$.
The limit of a convergent sequence is unchanged by passing to a subsequence, and so $f(p_{n_\ell}) = q_{n_\ell} \to q$ as $\ell \to \infty$.
Thus, $f(p^*) = q = f(p)$, contrary to $f$ being a bijection.
It follows that $p_n \to p$ and therefore that $f^{-1}$ is continuous since it preserves sequential convergence.
\end{proof}

Lastly, we combine the concepts of uniform continuity and sequential compactness.
Here is the definition for uniformly continuous mappings between metric spaces.

\begin{defn}
  A mapping $f: M \to N$ between metric spaces is \textsf{uniformly continuous} if for each $\varepsilon > 0$ there exists a $\delta > 0$ such that
  \[
    p, q \in M \text{ and } d_M(p,q) < \delta \implies
    d_N(f(p), f(q)) < \varepsilon.
  \]
\end{defn}

Again one should compare this definition with that of ordinary continuous functions and note their difference.

\begin{thm}
  Every continuous function defined on a sequentially compact set is uniformly continuous.
\end{thm}

\begin{proof}
  Suppose not, that is, $f: M \to N$ is continuous, $M$ is sequentially compact, but $f$ is not uniformly continuous.
  Then there is some $\varepsilon > 0$ such that no matter how small $\delta$ is, there exist points $p, q \in M$ with $d_M(p,q) < \delta$ but $d_N( f(p), f(q) ) \geqslant \varepsilon$.
  Take $\delta = 1/n$ for each $n \in \mathbb{N}$ to product two sequences $\langle p_n \rangle, \langle q_n \rangle$ in $M$ such that $d_M(p_n,q_n) < 1/n$ but $d_N(f(p_n),f(q_n)) \geqslant \varepsilon$.
  Sequential compactness of $M$ implies that there is a subsequence $\langle p_{n_k}\rangle$ of $\langle p_n \rangle$ that converges to some $p \in M$ as $k \to \infty$.
  Since $d_M(p_{n_k}, q_{n_k}) < 1/n_k \to 0$ as $k \to \infty$, $\langle q_{n_k} \rangle$ converges to $p$ as well.
  Continuity of $f$ at $p$ implies that $f(p_{n_k}) \to f(p)$ and $f(q_{n_k}) \to f(p)$.
  When $k$ is sufficiently large we have
  \[
    d_N(f(p_{n_k}), f(q_{n_k})) \leqslant d_N( f(p_{n_k}), f(p) ) + d_N( f(p), f(q_{n_k}) ) < \frac{\varepsilon}{2} + \frac{\varepsilon}{2} = \varepsilon,
  \]
  which contradicts to our construction that $d_N(f(p_{n_k}),f(q_{n_k})) \geqslant \varepsilon$ for any $k$.
\end{proof}

  \section{Compactness}
\label{sec:covering-comp}

We now come to the definition of genuine compactness in terms of open sets.
Most people find this abstract.
Nevertheless it catches the most important feature of what analysis needs.
You may find that it is parallel to what we have developed in sequential compactness.

\begin{defn}
  A collection $\mathcal U$ of subsets of $M$ \textsf{covers} $A \subseteq M$ if $A$ is contained in the union of sets belongs to $\mathcal{U}$.
  The collection $\mathcal{U}$ is a \textsf{covering} of $A$.
  If $\mathcal{U}$ and $\mathcal{V}$ both cover $A$ and if $\mathcal{V} \subseteq \mathcal{U}$ in the sense that each set $V \in \mathcal{V}$ belongs also to $\mathcal{U}$ then we say that $\mathcal{U}$ \textsf{reduces} to $\mathcal{V}$, and that $\mathcal{V}$ is a \textsf{subcovering} of $\mathcal{U}$.
\end{defn}

\begin{defn}
  If all the sets in a covering $\mathcal{U}$ of $A$ are open then $\mathcal{U}$ is an \textsf{open covering} of $A$.
  If every open covering of $A$ reduces to a finite subcovering of $A$ then we say that $A$ is \textsf{(covering) compact}.\footnote{You will frequently find it said that an open covering of $A$ \textit{has} a finite subcovering.  ``Has'' means ``reduces to.''}
\end{defn}

The idea is that if $A$ is compact and $\mathcal{U}$ is an open covering of $A$ then just a finite number of the open sets are actually doing the work of covering $A$.
The rest are redundant.

The mere existence of a finite open covering of $A$ is trivial and utterly worthless.
{\em Every} set $A$ has such a covering, namely the single open set $M$.
Rather, for $A$ to be compact, each and every open covering of $A$ must reduce to a finite subcovering of $A$.
Deciding directly whether this is so is daunting. 
How could you hope to verify the finite reducibility of all open coverings of $A$? 
There are so many of them. 
For this reason we concentrated on sequential compactness; it is relatively easy to check by inspection whether every sequence in a set has a convergent subsequence.

To check that a set is not compact it suffices to find an open covering which fails to reduce to a finite subcovering. 
Occasionally this is simple.
For example, the set $(0,1)$ is not compact in $\mathbb{R}$ because its covering
\[
  \mathcal{U} = \{ (1/n, 1) \colon n \in \mathbb{N} \}
\]
fails to reduce to a finite subcovering.

Indeed, the Heine-Borel theorem can be proved directly from the topological properties of compact sets.
This treatment can be found in many books about real analysis and point set topology.
Here we prove the following equivalence and the Heine-Borel theorem follows immediately.

\begin{thm}
  \label{thm:cpt}
  For a subset $A$ of a metric space $M$ the following are equivalent.
  \begin{enumerate}[(i)]
    \item $A$ is (covering) compact.
    \item $A$ is sequentially compact.
  \end{enumerate}
\end{thm}

\begin{proof}[Proof that (i) $\Rightarrow$ (ii).]
  We assume that $A$ is compact.
  If $A$ is not sequentially compact, then there is a sequence $\langle p_n \rangle$ of $A$ that does not have any subsequence that converges in $A$.
  If this is so, then for any point $a \in A$ there is some neighborhood $U_a$ of $a$ which contains only a finite number of terms of $\langle p_n \rangle$.
  Clearly $\{ U_a \colon a \in A \}$ is an open covering of $A$.
  Because $A$ is compact, there are a finite number of points $a_1, a_2, \dots, a_N \in A$ such that the finite collection $\mathcal{V} = \{ U_{a_1}, \dots, U_{a_N} \}$ also covers $A$.
  However, the collection $\mathcal{V}$ can contain only a finite number of terms of $\langle p_n \rangle$, which is absurd.
  Hence such a sequence $\langle p_n \rangle$ cannot exist in $A$, and $A$ is sequentially compact.
\end{proof}

Before we present the reverse implication, we need a notion called a Lebesgue number which is defined as follows.

\begin{defn}
  Let $\mathcal{U}$ be a covering of a subset $A$ in a metric space $M$.
  A \textsf{Lebesgue number} for the covering $\mathcal{U}$ of $A$ is a positive real number $\lambda$ such that for each $a \in A$ there is some $U \in \mathcal{U}$ with $B_\lambda(a) \subseteq U$.
\end{defn}

How do we interpret the above definition?
When $\mathcal{U}$ is an open covering of $A$, of course for each $a \in A$ there is a positive number $\lambda(a) > 0$ such that $B_{\lambda(a)}(a)$ lies in some member of $\mathcal{U}$.
But $\lambda(a)$ depends on $a$.
A Lebesgue number $\lambda$ for a covering $\mathcal{U}$ of $A$ can be thought a \textit{uniform} radius for every point in $A$ that produces an open neighborhood that lies in a single member of $\mathcal{U}$.
If $\lambda > 0$ is a Lebesgue number for a covering $\mathcal{U}$ of $A$, obviously any number $\lambda'$ with $0 < \lambda' \leqslant \lambda$ is also a Lebesgue number for $\mathcal{U}$ of $A$.

\begin{lem}
  \label{lem:lebesgue}
  Every open covering of a sequentially compact set has a Lebesgue number.
\end{lem}

\begin{proof}
  Suppose not: $\mathcal{U}$ is an open covering of a sequentially compact set $A$, and yet for each $\lambda > 0$ there exists an $a \in A$ such that no $U \in \mathcal{U}$ contains $B_\lambda(a)$.
  Take $\lambda = 1/n$ and let $a_n \in A$ be a point such that no $U \in \mathcal{U}$ contains $B_{1/n}(a_n)$.
  By sequentially compactness, there is a subsequence $\langle a_{n_k} \rangle$ of $\langle a_n \rangle$ that converges to some point $p \in A$.
  Since $\mathcal{U}$ is an open covering of $A$, there are an $r > 0$ and a $U \in \mathcal{U}$ such that $B_r(p) \subseteq U$.
  When $k$ is large enough, we have $d(a_{n_k}, p) < r/2$ and $1/n_k < r/2$.
  So if $d(x, a_{n_k}) < 1/n_k$, then
  \[
    d(x,p) \leqslant d(x,a_{n_k}) + d(a_{n_k},p) < \frac{1}{n_k} + \frac{r}{2} < \frac{r}{2} + \frac{r}{2} = r,
  \]
  that is, $B_{1/n_k}(a_{n_k}) \subseteq B_r(p) \subseteq U$.
  This contradicts to the supposition that no $U \in \mathcal{U}$ contains $B_{1/n_k}(a_{n_k})$.
  We conclude that, after all, $\mathcal{U}$ does have a Lebesgue number.
\end{proof}

\begin{proof}[Proof that (ii) $\Rightarrow$ (i) in Theorem~\ref{thm:cpt}.]
  Let $\mathcal{U}$ be an open covering of the sequentially compact set $A$.
  By Lemma~\ref{lem:lebesgue}, $\mathcal{U}$ has a Lebesgue number $\lambda > 0$.

  Let us assume the contrary, i.e., $\mathcal{U}$ cannot reduce to any finite subcovering that covers $A$.
  Choose any $a_1 \in A$ and some $U_1 \in \mathcal{U}$ such that $B_\lambda(a_1) \subseteq U_1$.
  Since $U_1$ alone cannot cover $A$, there is a point $a_2 \in A \setminus U_1$.
  We can proceed this procedure inductively as follows.

  Let $n \in \mathbb{N}$ and $n \geqslant 2$, and suppose that $a_1, \dots, a_{n-1} \in A$ and $U_1, \dots, U_{n-1} \in \mathcal{U}$ have been constructed such that
  \[
    a_k \notin \bigcup_{i=1}^{k-1} U_i, \quad
    B_\lambda(a_k) \subseteq U_k, \qquad k = 1, 2, \dots, n-1.
  \]
  Since $\{ U_1, \dots, U_{n-1} \}$ cannot cover $A$, there is a point $a_n \in A$ outside this union.
  Therefore there is a $U_n \in \mathcal{U}$ such that $B_\lambda(a_n) \subseteq U_n$.
  Thus a sequence $\langle a_n \rangle$ in $A$ and a sequence $\langle U_n \rangle$ in $\mathcal{U}$ have been constructed.

  We now argue that they lead to a contradiction.
  Since $A$ is sequentially compact, there is a subsequence $\langle a_{n_k} \rangle$ that converges to some point $p \in A$.
  For large $k$ we have $d(a_{n_k},p) < \lambda$ and
  \[
    p \in B_\lambda(a_k) \subseteq U_{n_k}.
  \]
  However, $a_{n_\ell}$ all lie outside $U_{n_k}$ whenever $\ell > k$, which contradicts to $a_{n_k} \to p$ as $k \to \infty$.
  Therefore $\mathcal{U}$ can indeed be reduced to a finite subcovering, and $A$ is then compact.
\end{proof}

After Theorem~\ref{thm:cpt}, we can now use the word ``compact'' for both covering compact and sequentially compact.\footnote{Note that the equivalence between these two compactnesses requires a Lebesgue number, hence this equivalence holds only for metric spaces.
They might be inequivalent in topological spaces whose topology is not induced by any metric.}
In the end we present two results about compactness that can be proved without appealing to sequential compactness.

\begin{thm}
  Every compact subset in a metric space is bounded and closed.
\end{thm}

\begin{proof}
  Let $A$ be a compact subset of a metric space $M$.
  Fix a point $m \in M$.  Clearly $A$ is covered by the open covering
  \[
    \mathcal{U} = \{ B_n(m) \colon n \in \mathbb{N} \}.
  \]
  Since $A$ is compact, $\mathcal{U}$ reduces to a finite subcovering
  \[
    \{ B_{n_k}(m) \colon k = 1, 2, \dots, N \}, \quad \text{where $n_1, \dots, n_N \in \mathbb{N}$.}
  \]
  If we set $R = \max \{ n_1, n_2, \dots, n_k \}$, we see that $A \subseteq B_R(m)$, which shows that $A$ is bounded.

  Next we show that $A$ is closed; this is equivalent to that $A^c$ is open.
  Consider an arbitrary point $p \in A^c$.
  For each $a \in A$, we take $\varepsilon(a) = d(p,a)/2 > 0$ and observe that $B_{\varepsilon(a)}(p) \cap B_{\varepsilon(a)}(a) = \varnothing$.
  Now $\{ B_{\varepsilon(a)}(a) \colon a \in A \}$ is an open covering of $A$.
  Since $A$ is compact, that covering can be reduced to a finite subcovering, i.e., there are a finite number of points $a_1, a_2, \dots, a_N \in A$ such that
  \[
      A \subseteq \bigcup_{n=1}^N B_{\varepsilon_n}(a_n), \qquad \text{where } \varepsilon_n = \varepsilon(a_n).
  \]
  Then if we take $\varepsilon = \min \{ \varepsilon_1, \dots, \varepsilon_N \} > 0$, we see that
  \[
    B_{\varepsilon}(p) \cap A
    \subseteq \bigcap_{n=1}^N B_{\varepsilon_n}(p) \cap \bigcup_{n=1}^N B_{\varepsilon_n}(a_n) = \varnothing,
  \]
  i.e., $B_{\varepsilon}(p) \subseteq A^c$.
  This shows that $p$ is an interior point of $A^c$.
  Since $p$ is arbitrary, we conclude that $A^c$ is open, i.e., $A$ is closed.
\end{proof}

\begin{thm}
  Let $A$ be a closed subset of a compact space $M$.
  Then $A$ is also compact.
\end{thm}

\begin{proof}
  Let $\mathcal{U}$ be an open covering of $A$.
  Then $\mathcal{U}_M := \mathcal{U} \cup \{ M \setminus A\}$ becomes an open covering of $M$, since $M \setminus A$ is an open set.
  Since $M$ is compact, $\mathcal{U}_M$ reduces to a finite subcovering $\mathcal{V}_M$.
  Then $\mathcal{V}_M \setminus \{ M \setminus A \}$ is a finite subcovering of $A$ which can be reduced from $\mathcal{U}$.
  Therefore $A$ is compact.
\end{proof}

  \section{Connectness}
\label{sec:conn}

We now consider the general notion of connectedness.
Let $A$ be a subset of a metric space $M$.
If $A$ is neither the empty set nor $M$ then $A$ is a \textsf{proper} subset of $M$.
Recall that if $A$ is both open and closed in $M$ it is said to be \textit{clopen}.
The complement of a clopen set is clopen.
The complement of a proper subset is proper.

\begin{defn}
  A \textsf{separation} of a metric space $M$ is a pair of disjoint open proper subsets of $M$ whose union is $M$.
  $M$ is \textsf{disconnected} if $M$ has a separation.
  $M$ is \textsf{connected} if $M$ has no separation.
\end{defn}

Equivalently, $M$ is disconnected if it has a proper clopen subset.
Connectedness of $M$ does not mean that $M$ is connected to something, but rather $M$ is one unbroken thing.

Another common criterion for a connected subset $E$ of a metric space $M$ is expressed in the following way: $E$ is connected if and only if there are no two disjoint open sets $U$ and $V$ in $M$ such that $E \subseteq U \cup V$.

We can characterize connected subsets of the reals $\mathbb{R}$.
But we need a formal definition of interval.

\begin{defn}
  A subset $I$ of $\mathbb{R}$ is called an \textsf{interval} if the following holds: for any $x, y \in I$ with $x < y$, $z \in I$ whenever $x < z < y$.
\end{defn}

Intuitively, if an interval contains two points, it also contains every point in-between.
Therefore an interval in $\mathbb{R}$ must be of the forms $(a,b), [a,b], [a,b)$, or $(a,b]$, depending on whether its supremum and infimum (which can be $\pm\infty$) belong to it or not.
In fact, intervals are the only connected subsets in $\mathbb{R}$.

\begin{thm}
  \label{thm:conn-R-interval}
  Let $E$ be a subset of $\mathbb{R}$.
  $E$ is connected if and only if $E$ is an interval.
\end{thm}

\begin{proof}
  ($\Longrightarrow$).
  If $E$ is not an interval, then there are three points $x < z < y$ such that $x,y \in E$ but $z \notin E$.
  Then $E \cap (-\infty, z)$ and $E \cap (z, \infty)$ form a separation of $E$, and hence $E$ is disconnected.

 \medskip
 \noindent ($\Longleftarrow$).
 Now we need to show that every nonempty interval $E$ in $\mathbb{R}$ is connected.  The empty set is always connected by definition, since it admits no separation, obviously.

 Suppose $E$ is not connected, i.e., there are disjoint open sets $U, V$ in $\mathbb{R}$ such that $E \subseteq U \cup V$ and both $E \cap U$ and $E \cap V$ are nonempty.
 Pick $x \in E \cap U$ and $y \in E \cap V$; by symmetry we may assume that $x < y$.
 Since $E$ is an interval, we have $[x,y] \subseteq E$.
 Consider the set
 \[
   X = \{ u \in [x,y] \colon u \in U \}.
 \]
 Obviously $x \in X \ne \varnothing$ and $X$ is bounded above by $y$.
 Therefore $z = \sup X$ exists by the least upper bound property.
 Since $z \in [x,y] \subseteq E$ as well, either $z \in U$ or $z \in V$.

 If $z \in U$ then $z$ cannot be $y$ since $y \notin U$.
 As $U$ is open, there is a $\delta > 0$ such that $z + \delta \in [x,y] \cap U$.  This contradicts to the fact that $z$ is an upper bound for $X$.
 On the other hand, if $z \in V$ then $z$ cannot be $x$ since $x \notin V$.
 As $V$ is open, there is $\delta' > 0$ such that $(z-\delta', z] \subseteq [x,y] \cap V$.  This implies that $X$ is also bounded above by $z - \delta'$, which is less than $z$; this contradicts to the fact that $z = \sup X$.

 Therefore the assumption that $E$ is disconnected is incorrect, that is, $E$ is in fact connected.
\end{proof}

Next we discuss the interplay between continuous mappings and connected sets.
Here is the main theorem.

\begin{thm}
  \label{thm:cont-conn}
  Let $M$ be connected.  If a mapping $f: M \to N$ is continuous and surjective, then $N$ is connected.  That is, the continuous image of a connected set is connected.
\end{thm}

\begin{proof}
  Let $A$ be a nonempty clopen subset of $N$.
  Then by the topological properties of continuous mappings, $f^{-1}(A)$ is again a clopen subset of $M$.  Since $f$ is onto and $A$ is nonempty, $f^{-1}(A)$ is nonempty as well.
  Since $M$ itself is connected, $f^{-1}(A)$ must be the whole space $M$.
  Therefore $A$ must be $N$ since $f$ is onto.
  This shows that $N$ cannot have any proper clopen subset, hence $N$ is connected.
\end{proof}

\begin{cor}
  If $M$ is connected and $M$ is homeomorphic to $N$ then $N$ is also connected.
  Connectedness is a topological property.
\end{cor}

Combining Theorems~\ref{thm:conn-R-interval} and \ref{thm:cont-conn}, we obtain a generalization of the intermediate value theorem.

\begin{cor}[Generalized intermediate value theorem]
  Every continuous real-valued function defined on a connected domain has the intermediate value property.
\end{cor}

\begin{proof}
  Let $M$ be a connected set and $f: M \to \mathbb{R}$ be a continuous function.
  By Theorem~\ref{thm:cont-conn} $f(M)$ is a connected subset of $\mathbb{R}$.
  By Theorem~\ref{thm:conn-R-interval} $f(M)$ is an interval.
  Therefore $f$ enjoys the intermediate value property.
\end{proof}

Now we see the very reason why the intermediate value theorem holds---the interval $[a,b]$ is connected!
This is different from the extreme value theorem, whose validity is based on the fact that $[a,b]$ is sequentially compact.

Connectedness properties give a good way to distinguish nonhomeomorphic sets.

\noindent\textbf{Example.} The union of two disjoint closed intervals is not homeomorphic to a single interval.
The former is disconnected while the latter is connected.

\medskip
\noindent\textbf{Example.} The closed interval $[a,b]$ is not homeomorphic to the unit circle $S^1 \subseteq \mathbb{R}^2$.
The former becomes disconnected when an interior point of it is removed, but the latter remains connected if any one of its points is removed.

\bigskip
Now we return to more properties about connected sets.

\begin{thm}
  The closure of a connected set is connected.
  More generally, if $S \subseteq M$ is connected and $S \subseteq T \subseteq \overline{S}$ then $T$ is connected.
\end{thm}

\begin{proof}
  Suppose there are two disjoint open sets $U, V \subseteq M$ such that $T \subseteq U \cup V$.
  Since $S \subseteq U \cup V$ as well, either $S \subseteq U$ or $S \subseteq V$; without loss of generality let us assume that $S \subseteq U$.
  Any point $p \in T \setminus S$ is a limit point of $S$.
  Therefore any neighborhood of $p$ contains a point $q \in S$ and $q \ne p$.
  Thus $p$ cannot belong to $V$, otherwise there would be a neighborhood of $p$ that lies completely in $V$, thus it has an empty intersection with $U$, contradiction.  Therefore $T \subseteq U$ and $T \cap V = \varnothing$.
  This shows that $T$ cannot have nontrivial separation, hence $T$ is connected.
\end{proof}

\begin{thm}
  The union of arbitrarily many connected sets that share a common point is connected.
\end{thm}

\begin{proof}
  Let $\{ C_\alpha \colon \alpha \in A \}$ be an arbitrary collection of connected sets such that $p \in \cap_{\alpha \in A} C_\alpha \ne \varnothing$.
  Write $C = \cup_{\alpha \in A} C_\alpha$.
  Suppose that there are two disjoint open sets $U, V$ such that $C \subseteq U \cup V$.
  Since $p \in C \subseteq U \cup V$, by symmetry we may assume that $p \in U$.
  Because for each $\alpha \in A$ we have $p \in C_\alpha \subseteq U \cup V$, $C_\alpha$ must also be a subset of $U$ because $C_\alpha$ is connected.
  Therefore $C \subseteq U$ and $C \cap V = \varnothing$.
  This shows that $C$ has no nontrivial separation and $C$ is connected.
\end{proof}

Lastly there is a similar but different notion of connectivity, which we define below.

\begin{defn}
  A \textsf{path} joining points $p$ to $q$ in a metric space $M$ is a continuous mapping $f: [a,b] \to M$ such that $f(a) = p$ and $f(b) = q$.
  If each pair of points in $M$ can be joined by a path that lies completely in $M$, then $M$ is called \textsf{path-connected}.
\end{defn}

\begin{thm}
  Path-connected sets are connected.
\end{thm}

\begin{proof}
  Let $M$ be a path-connected set.
  If $M$ is empty it is connected.
  So below we assume that $M$ is nonempty.

  If $M$ is not connected, then there are two disjoint clopen proper subsets $U, V$ of $M$ such that $M = U \cup V$.
  Pick a point $p \in U$ and $q \in V$.
  Since $M$ is path-connected there is a continuous mapping $f: [a,b] \to M$ such that $f(a) = p$ and $f(b) = q$.
  Therefore $f^{-1}(U)$ and $f^{-1}(V)$ form a separation of $[a,b]$ by two disjoint clopen proper subsets, which is a contradiction since $[a,b]$ is connected.
  Hence $M$ must be connected.
\end{proof}

Although any connected subset of $\mathbb{R}$ is path-connected, there are connected sets in $\mathbb{R}^2$ which are not path-connected.
Nevertheless, if we add one more condition these notions are the same.

\begin{thm}
  Every open connected subset of $\mathbb{R}^m$ is path-connected.
\end{thm}

\begin{proof}
  We first note here that any convex set in $\mathbb{R}^m$ is path-connected since any pair of points can be joined by a segment that lies totally in that convex set.
  Because an open ball is convex, it is also path-connected.

  Let $D$ be a nonempty, open, and connected subset of $\mathbb{R}^m$ (the empty set is path-connected by definition).
  Fix a point $p \in D$ and consider the following subsets of $D$:
  \[
    U = \{ u \in D \colon \text{ $u$ and $p$ can be joined by a path in $D$} \}, \quad V = D \setminus U.
  \]

  We argue that both $U$ and $V$ are open subsets of $D$ as follows.
  Let $u \in U$.  Since $u \in D$ and $D$ is open, there is an $r > 0$ such that $B_r(u) \subseteq D$.  Then every point $x$ in $B_r(u)$ can be joined to $p$ by a path from $p$ to $u$ followed by the radius from $u$ to $x$.
  Hence $B_r(u) \subseteq U$ as well.  This shows that $U$ is an open subset of $D$.

  On the other hand, suppose $v \in V$.  Again there is an $s > 0$ such that $B_s(v) \subseteq D$.
  No point in $B_s(v)$ can be joined to $p$ by a path in $D$ because otherwise $v \in U$.  Hence $B_s(v) \subseteq V$.
  This shows that $V$ is also an open subset of $D$.

  Now $D$ is written as a union of two disjoint open subsets $U, V$, and $U \ne \varnothing$.
  Since $D$ is connected we must have $V = \varnothing$, i.e. $U = D$.
  This says any point in $D$ can be joined to $p$ by a path in $D$, hence $D$ is path-connected.
\end{proof}

  \section{Compactness and Continuous Mappings, Heine-Borel Theorem Again}
\label{sec:heine-borel}

\begin{thm}
  Let $f : M \to N$ be a continuous mapping between two metric spaces and $M$ be compact.
  Then the image $f(M)$ is a compact subset of $N$.
  That is, the continuous image of a compact set is compact.
\end{thm}

\begin{proof}
  Let us prove this result by using covering.
  Suppose that $\mathcal{U} = \{ U_\alpha \colon \alpha \in A \}$ is an open covering of $f(M)$.
  Since $f$ is continuous, $f^{-1}(U_\alpha)$ is open for each $\alpha \in A$.
  Therefore $\mathcal{U}' = \{ f^{-1}(U_\alpha) \colon \alpha \in A \}$ forms an open covering of $M$.
  Since $M$ is compact, $\mathcal{U}'$ reduces to a finite subcovering, i.e., there are a finite number of indices $\alpha_1, \dots, \alpha_k \in A$ such that
  \[
    M \subseteq \bigcup_{i=1}^k f^{-1}(U_{\alpha_i}).
  \]
  Then
  \[
    f(M) \subseteq \bigcup_{i=1}^k f(f^{-1}(U_{\alpha_i})) \subseteq \bigcup_{i=1}^k U_{\alpha_i},
  \]
  i.e. $\mathcal{U}$ indeed reduces to a finite subcovering of $f(M)$.
\end{proof}

\begin{cor}
  If $M$ and $N$ are homeomorphic and $M$ is compact, then $N$ is compact too.
  Compactness is a topological property.
\end{cor}

Towards the Heine-Borel theorem, we will use the following properties.

\begin{thm}
  A bounded, closed interval $[a,b]$ in $\mathbb{R}$ is compact.
\end{thm}

\begin{proof}
  Let $\mathcal{U}$ be an open covering of $[a,b]$.
  We need to reduce $\mathcal{U}$ to a finite subcovering.
  To achieve this, let us consider the following subset of $[a,b]$:
  \[
    X = \{ x \in [a,b] \colon \text{ the interval $[a,x]$ can be covered by a finite collection of sets in $\mathcal{U}$} \}.
  \]
  Obviously $a \in X \ne \varnothing$ and $b$ is an upper bound for $X$.
  Therefore $c = \sup X$ exists by the least upper bound property.
  Below we argue that $c \in X$ and $b = c$, which proves our goal.

  Since $c \in [a,b]$ and $\mathcal{U}$ is an open covering of $[a,b]$, there is an open set $U_0 \in \mathcal{U}$ with $c \in U_0$.
  Then there is a $\delta > 0$ such that $(c - \delta, c + \delta) \subseteq U_0$.
  Because $c = \sup X$, there is an $x \in X$ with $c - \delta < x \leqslant c$.
  From the definition of $X$, the interval $[a,x]$ can be covered by the union of some finite number of open sets $U_1, U_2, \dots, U_N$ in $\mathcal{U}$.
  Therefore $[a,c]$ can be covered by the union of $U_0, U_1, \dots, U_N$ in $\mathcal{U}$, totally $N+1$ members in $\mathcal{U}$, which is a finite number.  Thus $c \in X$.

  If we go through the argument in the last paragraph, then we see that $c = b$; for otherwise $c + \frac{\delta}{2}$ is also an element in $X$ that is larger than $c$, which is absurd.
  Hence $b = c \in X$ and the proof is completed.
\end{proof}

\begin{thm}[Tube lemma]
  \label{thm:tube}
  Let $M, N$ be compact metric spaces.
  Then the Cartesian product $M \times N$ is compact.
\end{thm}

\begin{proof}
  We had three metrics on the product of two metric spaces that induce the same product topology.
  Hence it does not matter which metric we use here.
  Here we use the $d = d_{\op{max}}$ metric on $M \times N$; that is,
  \[
    d( (x_1, x_2), (y_1, y_2) ) := \max \{ d_M(x_1, y_1), d_N(x_2, y_2) \}, \qquad (x_1,x_2), (y_1, y_2) \in M \times N.
  \]
  To show that $M \times N$ is compact, we need to reduce any given open covering $\mathcal{U} = \{ U_\alpha \colon \alpha \in A \}$ of $M \times N$ to a finite subcovering.

  Let us investigate the vertical segment $V_p = \{ p \} \times N$ for every $p \in M$.
  Since $V_p$ is homeomorphic to $N$, $V_p$ is compact.
  For each $q \in N$, there is an $\alpha_q \in A$ such that $(p,q) \in U_q$.
  Since $U_{\alpha_q}$ is open, there is an $r(q) > 0$ such that $B_{r(q)}( (p,q) ) \subseteq U_{\alpha_q}$ as well.
  As we see above, $V_p$ is compact and is covered by the open balls $\{ B_{r(q)}( (p,q) ) \colon q \in N \}$, therefore the covering reduces to a finite subcovering $\{ B_{r(q(1))}( (p,q(1)) ), \dots, B_{r(q(\ell))}( (p,q(\ell) )\}$ for some $q(1), \dots, q(\ell) \in N$.
  Let $\mathcal{V}(q) = \{ U_{\alpha_{q(1)}}, \dots, U_{\alpha_{q(\ell)}} \}$. 
  Note that the collection $\mathcal{V}_p$ covers not only $V_p$, but also $B_{\rho(p)}(p) \times N$, where $\rho(p) = \min \{ r(q(1)), \dots, r(q(\ell)) \} > 0$, as well. 

  Now $\{ B_{\rho(p)}(p) \colon p \in M \}$ becomes an open covering of the compact set $M$, therefore it reduces to a finite subcovering, i.e., there are a finite number of points $p_1, p_2, \dots, p_s \in M$ such that
  \[
    M \subseteq \bigcup_{n=1}^s B_{\rho(p_n)}(p_n).
  \]
  It is now clear that $\mathcal{U}$ reduces to the finite subcovering
  \[
    \bigcup_{n=1}^s \mathcal{V}_{p_n}
  \]
  that covers the product space $M \times N$.
  Now the proof is complete.
\end{proof}

Using induction, Theorem~\ref{thm:tube} implies that the Cartesian product of a \textit{finite} number of compact sets is still compact.
Therefore the following corollary is immediately implied.
Nevertheless the same conclusion holds for an arbitrary number of compact sets (\textit{Tychonoff theorem}), whose proof requires the axiom of choice.

\begin{cor}
  A box $[a_1, b_1] \times \cdots \times [a_m, b_m]$ is a compact subset of $\mathbb{R}^m$.
\end{cor}

Combining all the relevant results, we have proven the Heine-Borel theorem again.
\begin{thm}[Heine-Borel theorem]
  Let $A$ be a subset of $\mathbb{R}^m$.
  $A$ is compact if and only if $A$ is bounded and closed.
\end{thm}

The bounded value theorem and the extreme value theorem for real-valued functions defined over compact sets are still valid; their proofs are no different than those we did for sequentially compact sets.
However, we have a different proof for uniform continuity of continuous functions over compact sets.

\begin{thm}
  Let $f: M \to N$ be a continuous function where $M$ is a compact space.
  Then $f$ is uniformly continuous on $M$.
\end{thm}

\begin{proof}
  Let $\varepsilon > 0$ be given.
  For any $a \in M$, there is a $\delta(a) > 0$ such that
  \[
    f(B_{\delta(a)}(a)) \subseteq B_{\varepsilon/2}(f(a)).
  \]
  Clearly $\{ B_{\delta(a)}(a) \colon a \in M \}$ forms an open covering of $M$.    Since $M$ is also sequentially compact, this open covering has a Lebesgue number $\lambda > 0$, which means that for any $x \in M$, there is some $a \in M$ such that
  \[
    \tag{1}
    B_\lambda(x) \subseteq B_{\delta(a)}(a).
  \]
  
  We claim that the number $\lambda$ has the desired property.
  Applying $f$ to the both sides of (1) yields
  \[
    f(B_\lambda(x)) \subseteq f(B_{\delta}(a)) \subseteq B_{\varepsilon/2}(f(a)).
  \]
  Therefore whenever $p, x \in B_\lambda(x)$, we have $d_N(f(p), f(a)) < \varepsilon/2$ and $d_N(f(x), f(a)) < \varepsilon/2$ as well.
  Using the triangle inequality, we obtain
  \[
    d_N( f(p), f(x) ) \leqslant d_N(f(p), f(a)) + d_N( f(a), f(x) ) < \frac{\varepsilon}{2} + \frac{\varepsilon}{2} = \varepsilon.
  \]
  This means that
  \[
    f(B_\lambda(x)) \subseteq B_\varepsilon(f(x)).
  \]
  Since $\lambda$ does not depend on $x$, we have shown that $f$ is uniformly continuous on $M$.
\end{proof}

The Heine-Borel theorem holds in $\mathbb{R}^m$.
So what properties are equivalent to compactness in general metric spaces?
Sometimes this question might be hard to answer,
but we have a good answer here.
First we need a definition.

\begin{defn}
  A subset $A$ of a metric space $M$ is \textsf{totally bounded} if for each $\varepsilon > 0$ there exists a finite covering of $A$ by $\varepsilon$-balls.
\end{defn}

\begin{thm}[Generalized Heine-Borel theorem]
  A subset of a complete metric space is compact if and only if it is closed and totally bounded.
\end{thm}

\begin{proof}
  Any compact subset in a metric space is closed.
  Let $A$ be a compact subset of a metric space $M$.
  Also for any $\varepsilon > 0$, the collection
  \[
    \{ B_\varepsilon(a) \colon a \in A \}
  \]
  is an open covering of $A$.
  Since $A$ is compact, it reduces to a finite subcovering.
  Hence $A$ is totally bounded.

  Conversely, let $A$ be a closed and totally bounded subset of a complete metric space $M$.
  We claim that $A$ is sequentially compact.
  For this, we start with a sequence $\langle a_n \rangle$ in $A$.
  Let $\varepsilon_k = 1/k$, $k = 1, 2, \dots$.
  Since $A$ is totally bounded we can cover $A$ by finitely many $\varepsilon_1$-balls
  \[
    B_{\varepsilon_1}(q_1), \dots, B_{\varepsilon_1}(q_m).
  \]
  By the pigeonhole principle, terms of the sequence $\langle a_n \rangle$ lie in at least one of these neighborhoods infinitely often, say it is $B_{\varepsilon_1}(p_1)$.
  Choose
  \[
    a_{n_1} \in A_1 := A \cap B_{\varepsilon_1}(q_1).
  \]
  Every subset of a totally bounded set is totally bounded, so we can cover $A_1$ by finitely many $\varepsilon_2$-balls.
  For one of them, say $B_{\varepsilon_2}(p_2)$, $\langle a_n \rangle$ lie in $A_2 := A_1 \cap B_{\varepsilon_2}(p_2)$ infinitely often.
  Choose $a_{n_2} \in A_2$ with $n_2 > n_1$.

  Proceeding inductively, cover $A_{k-1}$ by finitely many $\varepsilon_k$-balls, one of which, say $B_{\varepsilon_k}(p_k)$, contains terms of the sequence $\langle a_n \rangle$ infinitely often.
  Then choose $a_{n_k} \in A_k := A_{k-1} \cap B_{\varepsilon_k}(p_k)$ with $n_k > n_{k-1}$.
  Then $\langle a_{n_k} \rangle$ is a subsequence of $\langle a_n \rangle$.
  Moreover $\langle a_{n_k} \rangle$ is a Cauchy sequence: for if $\varepsilon > 0$ is given we choose $N$ such that $2/N < \varepsilon$.  If $k, \ell \geqslant N$ then
  \[
    a_{n_k}, a_{n_\ell} \in A_N \quad \text{and} \quad
    \op{diam} A_N \leqslant 2 \varepsilon_N = \frac{2}{N} < \varepsilon,
  \]
  which shows that $\langle a_{n_k} \rangle$ satisfies the Cauchy condition.
  Completeness of $M$ implies that $\langle a_{n_k} \rangle$ converges to some $p \in M$.
  Since $A$ is closed, $p \in A$.  Hence $A$ is (sequentially) compact.
\end{proof}

\begin{cor}
  A metric space is compact if and only if it is complete and totally bounded.
\end{cor}

\begin{proof}
  Every compact metric space is complete.
  This is because, given a Cauchy sequence $\langle p_n \rangle$ in $M$, compactness implies that some subsequence converges in $M$, and if a Cauchy sequence has a convergent subsequence then the mother sequence converges too.
  As observed above, compactness immediately gives the total boundedness.

  Conversely, assume that $M$ is complete and totally bounded.
  Every metric space is closed in itself, hence $M$ is compact by the Generalized Heine-Borel theorem.
\end{proof}

  \section{Miscellany for Continuous Functions and Metric Spaces}
\label{sec:misc-cont}

\subsection{Monotone functions and convex functions}

In this topic we discuss properties on monotone functions and convex functions.
Most properties can be assigned as hard exercises.
Nevertheless we hereby indicate some of these since they can be useful later on.

\begin{defn}
  Let $I$ be an interval in $\mathbb{R}$, and $f$ be a real function defined on $I$.
  \begin{enumerate}[(i)]
    \item $f$ is said to be \textsf{increasing} (resp.\ \textsf{strictly increasing}) on $I$ if $f(x_1) \leqslant f(x_2)$ (resp.\ $f(x_1) < f(x_2)$) whenever $x_1, x_2 \in I$ and $x_1 < x_2$.
  
    \item $f$ is said to be \textsf{decreasing} (resp.\ \textsf{strictly decreasing}) on $I$ if $f(x_1) \geqslant f(x_2)$ (resp.\ $f(x_1) > f(x_2)$) whenever $x_1, x_2 \in I$ and $x_1 < x_2$.
      \item $f$ is said to be \textsf{monotone} on $I$ if $f$ is increasing or $f$ is decreasing on $I$.
  \end{enumerate}
\end{defn}

\noindent\textit{Remark.} Some books use {\em increasing} to mean our strictly increasing, and {\em weakly increasing} to mean our increasing.  Likewise for decreasing.
Make sure to check the definitions before discussion.

The highlight for monotone functions is that they do not have many points of discontinuity.  Here is why. 
\begin{prop}
  Let $f$ be a monotone real function on an interval $I \subseteq \mathbb{R}$.
  \begin{enumerate}[$(a)$]
    \item For each interior point $x$ of $I$, the left and right limits of $f$ exist at $x$, i.e.,
      \[
	f(x-) := \lim_{t \to x-} f(t), \qquad
	f(x+) := \lim_{t \to x+} f(t)
      \]
      both exist in $\mathbb{R}$.

    \item There are at most a countable number of points of discontinuity of $f$ in $I$.
  \end{enumerate}
\end{prop}

\begin{proof}
  \begin{enumerate}[$(a)$]
    \item We may assume that $f$ is increasing; otherwise replace $f$ by $-f$.
      In the following we show that the left limit $f(x-)$ exists when $x$ is not the minimum of $I$; the existence of the right limit $f(x+)$ for any $x$ that is not the maximum of $I$ follows analogously. 

      As indicated earlier, let us assume that $x$ is not the minimum of $I$; this means that the set $\{ t \in I \colon t < x \}$ is not empty.
      Consider the following subset of $\mathbb{R}$:
      \[
	A = \{ f(t) \colon t \in I, t < x \}.
      \]
      The set $A$ is nonempty and is bounded above by $f(x)$, since $f$ is increasing.
      Therefore $\alpha = \sup A$ exists by the least upper bound property.
      We claim that $f(x-)$ equals $\alpha$.

      For any $\varepsilon > 0$, $\alpha - \varepsilon$ is not an upper bound for $A$.
      Hence there exists a $t \in I$ with $t < x$ such that $\alpha - \varepsilon < f(t)$.
      Take $\delta = x - t > 0$.  Then whenever $s \in I$ with $t = x - \delta < s < x$, we have
      \[
	\alpha - \varepsilon < f(t) \leqslant f(s) \leqslant \alpha.
      \]
      Since $\varepsilon$ is arbitrary, we establish that $f(x-) = \alpha \leqslant f(x)$. 

    \item Again let us assume that $f$ is increasing on $I$.
      Since $f$ is a function defined over $I \subseteq \mathbb{R}$, $f$ is continuous at an interior point $x \in I$ if and only if
      \[
	f(x-) = f(x) = f(x+).
      \]
      Therefore any (interior) discontinuity $t$ of $f$ in $I$ corresponds to a nondegenerate open interval $I_t := \left( f(t-), f(t+) \right)$ in $\mathbb{R}$.
      Moreover, $I_{t_1} \cap I_{t_2} = \varnothing$ whenever $t_1 \ne t_2$.
      Now we may pick any rational number $q_t \in I_t$; we then have $q_{t_1} \ne q_{t_2}$ whenever $t_1 \ne t_2$.
      The mapping $t \mapsto q_t$ is then an injection from the points of discontinuity of $f$ in the interior of $I$ to the denumerable set $\mathbb{Q}$.
      From this we deduce that the set of those points is countable (i.e., finite or denumerable).
  \end{enumerate}
\end{proof}

Next is the definition of convex functions, which are very important in optimization problems.

\begin{defn}
  A function $f: (a,b) \to \mathbb{R}$ is called a \textsf{convex function} if for any $x,y \in (a,b)$ and any $s, t \in [0,1]$ with $s + t = 1$ we have
  \[
    f(sx + ty) \leqslant s \, f(x) + t \, f(y).
  \]
  
  If the inequality is reversed, the function $f$ is called a \textsf{concave function}.
\end{defn}

Geometrically speaking, the definition of convex functions says that the graph of a convex function stays \textit{below} any of its secant line, as shown in Figure~\ref{fig:convex}.

\begin{figure}
  \centering
  \begin{tikzpicture}
    \draw [domain=-1:4, very thick] plot (\x, {\x*\x/4 - \x/2 - 1});
    \draw [blue, thick] (0,-1) -- (3.5,0.3125);
  \end{tikzpicture}
  \caption{The graph of a convex function under its secant line (blue)}
  \label{fig:convex}
\end{figure}

Although the definition of convex functions only talks about inner division points, it is pleasant to see that this already implies continuity of such functions.
We start with the following lemma.

\begin{lem}
  Let $f : I \to \mathbb{R}$ be a convex function on an interval $I \subseteq \mathbb{R}$.
  Then
  \[
    \frac{ f(x) - f(y) }{ x - y } \leqslant \frac{ f(x) - f(z) }{ x - z } \leqslant \frac{ f(y) - f(z) }{ y - z }
  \]
  for any triplets $x, y, z \in I$ with $x < y < z$.
\end{lem}

The validity of this lemma is clear from Figure~\ref{fig:convex-secant}, when we regard all the fractions as {\em slopes} of the secant lines determined by points on the graph.
The algebraic demonstration of the above inequalities is left to the readers.

\begin{figure}
  \centering
  \begin{tikzpicture}
    \draw [domain=-1:4, very thick] plot (\x, {\x*\x/4 - \x/2 - 1});
    \draw [blue, thick] (0,-1) coordinate (X) -- (3.5,0.3125) coordinate (Z);
    \draw [green!20!black, thick] (X) -- (1.6, -1.16) coordinate (Y);
    \draw [red, thick] (Y) -- (Z);
    \node [below] at (X) {$x$};
    \node [below] at (Y) {$y$};
    \node [below] at (Z) {$z$};
  \end{tikzpicture}
  \caption{Comparison of slopes of secant lines on the graph of a convex function}
  \label{fig:convex-secant}
\end{figure}
\begin{thm}
  Let $I$ be a nondegenerate interval in $\mathbb{R}$, and $f : I \to \mathbb{R}$ be a convex function.
  \begin{enumerate}[$(a)$]
    \item For any $x,y$ in $I$ and $s, t \in \mathbb{R}$ with $s + t = 1$ but $s, t \notin [0,1]$, we have
      \[
	f(sx + ty) \geqslant s \, f(x) + t \, f(y), \qquad
	\text{if } sx + ty \in I.
      \]

    \item $f$ is continuous at any interior point of $I$.
  \end{enumerate}
\end{thm}

\begin{proof}
  \begin{enumerate}[$(a)$]
    \item Let $z = sx + ty$.
      By symmetry let us assume that $s < 0$ and $t > 1$.
      Then $y = \dfrac1t z - \dfrac{s}{t} x$ becomes an inner division point of $z$ and $x$ in $I$.
      From the definition of convex functions, we have
      \[
	f(y) \leqslant \frac{1}{t} f(z) - \frac{s}{t} f(x),
      \]
      which is equivalent to $f(z) = f(sx+ty) \geqslant s \, f(x) + t \, f(y)$.

    \item Let $z$ be any interior point of $I$, and fix two points $x,y \in I$ with $x < z < y$.

      Assume first that $t \in I$ and $x < t < z$.  By the definition of convex functions and (1), we have
      \begin{equation}
	\label{ineq:convex1}
	f(z) + \frac{f(x)-f(z)}{x-z} (t-z) \leqslant f(t) \leqslant
	f(z) + \frac{f(y)-f(z)}{y-z} (t-z),
      \end{equation}
      since $t$ is an inner division point of $x,z$ but an outer division point of $y,z$.  By squeeze theorem and (\ref{ineq:convex1}) we see that $f(t) \to f(z)$ as $t \to z-$.

      On the other hand, let us take $t \in I$ and $z < y < t$.
      This time the inequalities (\ref{ineq:convex1}) are reversed:
      \begin{equation}
	\label{ineq:convex2}
	f(z) + \frac{f(x)-f(z)}{x-z} (t-z) \geqslant f(t) \geqslant
	f(z) + \frac{f(y)-f(z)}{y-z} (t-z).
      \end{equation}
      Therefore $f(t) \to f(z)$ as $t \to z+$ as well,
      Since $f(z-) = f(z) = f(z+)$, we conclude that the convex function $f$ is continuous at the interior point $z$ of $I$.
  \end{enumerate}
\end{proof}

\noindent{\em Remark.}
There are a few more properties of convex functions that are related to differentiation.
They will be discussed in the later topics.

\subsection{Sup norm on continuous functions over compacts}

Let $K$ be a (nonempty) compact metric space.
Denote by $\mathcal{C}(K) = \mathcal{C}(K, \mathbb{R})$ the set of all continuous real-valued functions on $K$.

\begin{equation*}
  \| f \|_K := \sup \{ |f(x)| \colon x \in K \}.
\end{equation*}

\begin{thm}
  \label{thm:max-norm}
  \begin{enumerate}[(1)]
    \item For any $f \in \mathcal{C}(K)$, $\| f \|_K \in \mathbb{R}$.
      In fact, 
      \begin{equation}
	\label{eq:max-norm}
	\| f \|_K = \max \{ |f(x)| \colon x \in K \}.
      \end{equation}
    \item $\| \cdot \|_K$ defines a norm on $\mathcal{C}(K)$.
  \end{enumerate}
\end{thm}

\begin{proof}
  \begin{enumerate}[(1)]
    \item This follows immediately from the extreme value theorem.

    \item The first two conditions for norm are obvious.
      Let us check the triangle inequality.
      For $f, g \in \mathcal{C}(K)$, there is a point $x_0 \in K$ such that $\| f + g \|_K = |f(x_0) + g(x_0)|$.  Therefore,
      \begin{align*}
	\| f + g \|_K &= |f(x_0) + g(x_0)| \\
	&\leqslant |f(x_0)| + |g(x_0)| & \text{(This is the triangle inequality in $\mathbb{R}$)} \\
	&\leqslant \| f \|_K + \| g \|_K,
      \end{align*}
      which is needed to show.
  \end{enumerate}  
\end{proof}

This sup-norm then induces a metric, which in turn induces a topology, on $\mathcal{C}(K)$.
In general a bounded closed subset of $\mathcal{C}(K)$ is not compact.
For example, we take $K = [0,1] \subseteq \mathbb R$ and consider the sequence $\langle f_n(x) = x^n \rangle \subseteq M = \mathcal{C}(K)$.
Clearly $\| f_n \|_K = 1$ so they all belong to the unit closed ball $\overline{B_1(0)}$.
But $\langle f_n \rangle$ has no convergent subsequence under $\| \cdot \|_K$, as we shall see below.
Therefore $\overline{B_1(0)}$ in $\mathcal{C}([0,1])$ is not (sequentially) compact.

What does an open ball in $\mathcal{C}(K)$ look like?
Let $f \in \mathcal{C}(K)$ and $r > 0$.  We have
\[
  B_r(f) = \{ g \in M \colon \| g - f \|_K < r \}.
\]
The condition $\|g-f\|_K < r$ can be rephrased as
\[
  f(x) - r < g(x) < f(x) + r \qquad \forall\, x \in K.
\]
Hence $g \in B_r(f)$ means that the graph of $g$ over $K$ falls into an $r$-\textit{tubular neighborhood} of $f$, depicted as in Figure~\ref{fig:tubular}.

\begin{figure}
  \centering
  \begin{tikzpicture}
    \draw[-Stealth] (0,0) -- (5,0) node [right] {$K$};
    \draw[very thick, domain=0.5:4.5] plot (\x, {(\x-0.9)*(\x-2.5)*(\x-4)/5 + 1.5}) node [right, scale=0.7] {$f(x)$}; 
    \draw[dashed, domain=0.5:4.5] plot (\x, {(\x-0.9)*(\x-2.5)*(\x-4)/5 + 1.2}) node [right, scale=0.7] {$f(x)-r$}; 
    \draw[dashed, domain=0.5:4.5] plot (\x, {(\x-0.9)*(\x-2.5)*(\x-4)/5 + 1.8}) node [right, scale=0.7] {$f(x)+r$}; 
  \end{tikzpicture}
  \caption{A tubular neighborhood $B_r(f)$}
  \label{fig:tubular}
\end{figure}

Let us talk about convergence in $\mathcal{C}(K)$ now.
Suppose $\langle f_n \rangle$ is a Cauchy sequence in $\mathcal{C}(K)$.
Then for each $\varepsilon > 0$, there is an $N \in \mathbb{N}$ such that
\[
  \tag{2}
  |f_n(x) - f_m(x)| < \varepsilon \qquad \text{whenever $n,m \geqslant N$ and $x \in K$.}
\]
Hence the real sequence $\langle f_n(x) \rangle$ also satisfies a Cauchy condition.
By the completeness of $\mathbb{R}$, $\langle f_n(x) \rangle$ converges to some real number which we denote by $f(x)$.
We hereby write
\[
  \lim_{n \to \infty} f_n(x) = f(x), \qquad x \in K
\]
or $f_n \to f$, and call this a \textit{pointwise convergence}.
Moreover, 
\[
  \tag{3}
  \sup \{ |f_n(x) - f(x)| \colon x \in K \} \leqslant \varepsilon \qquad
  \text{whenever $n \geqslant N$},
\]
by passing $m \to \infty$.  Note that this $N$ does not depend on $x \in K$, and we call this \textit{uniform convergence} and write $f_n \rightrightarrows f$ on $K$.
We now characterize this limit function $f$ on $K$.

\begin{thm}
  \label{thm:CK-complete}
  Let $\mathcal{C}(K)$ denote the space of continuous real-valued functions defined on a compact metric space $K$.
  Let $\langle f_n \rangle \subseteq \mathcal{C}(K)$ be a Cauchy sequence under $\| \cdot \|_K$ whose pointwise limit is $f$.
  Then $f$ is also a continuous function on $K$.
\end{thm}

\begin{proof}
  Let $\varepsilon > 0$ be given.
  By (3) there is an $N \in \mathbb N$ such that
  \[
    \tag{4}
    |f_n(x) - f(x)| < \frac{\varepsilon}{3} \qquad \forall \, n \geqslant N, \, x \in K.
  \]
  Let us look at the function $f_N$.
  Since $f_N$ is continuous at any point $x \in K$, there is a $\delta > 0$ such that
  \[
    \tag{5}
    |f_N(t) - f_N(x)| < \frac{\varepsilon}{3}, \qquad \forall \, t \in K_\delta(x).
  \]
  So now whenever $t \in K_\delta(x)$, we have (by (4), (5), (4))
  \[
    |f(t) - f(x)| \leqslant |f(t) - f_N(t)| + |f_N(t) - f_N(x)| + |f_N(x) - f(x)| 
    < \frac{\varepsilon}{3} + \frac{\varepsilon}{3} + \frac{\varepsilon}{3} = \varepsilon.
  \]
  Therefore $f$ is continuous at $x \in K$.
  Since $x$ is arbitrary, we conclude that $f \in \mathcal{C}(K)$.
\end{proof}

With Theorem~\ref{thm:CK-complete}, we see that $(\mathcal{C}(K), \| \cdot \|_K)$ becomes a \textit{complete} metric space, since every Cauchy sequence in it converges to some member of it.
We will discuss more on this type of convergence towards the end of the semester.

\subsection{Normed spaces}

Let $M$ be a vector space over $F = \mathbb{R}$ or $\mathbb{C}$.
We can talk about the ``length'' of a vector in the following way.

\begin{defn}
  A \textsf{norm} on $M$ is a real-valued function $\| \cdot \|$ on $M$ that satisfies the following properties:
  \begin{enumerate}[(1)]
    \item (Positivity) $\| x \| \geqslant 0$ for any $x \in M$; moreover, $\| x \| = 0$ if and only if $x = 0$ (the additive identity in $M$).
    \item (Homogeneity) For every vector $x$ and $\alpha \in F$, we have $\| \alpha x \| = | \alpha | \, \| x \|$ (the absolute value of $\alpha$ times the norm of $x$).
    \item (Triangle inequality) For any $x, y \in M$, we have $\| x + y \| \leqslant \| x \| + \| y \|$.
  \end{enumerate}
\end{defn}

A norm $\| \cdot \|$ on $M$ can induce a metric on $M$ by setting $d(x,y) = \| x - y \|$.
The verification is immediate.
We have seen examples of normed spaces of finite dimensions.
They are: for $x = (x_1, x_2, \dots, x_m) \in \mathbb{R}^m$, define
\[
  \| x \|_p = \begin{cases}
    \left( \sum_{k=1}^m |x_k|^p \right)^{1/p}, & \text{if $1 \leqslant p < \infty$}; \\
    \max \{ |x_k| \colon k = 1, \dots, m \}, & \text{if $p = \infty$.}
  \end{cases}
\]
In fact, all of them are equivalent in the following sense.

\begin{thm}
  Let $\| \cdot \|_p$ denote the $p$-norm in $\mathbb{R}^m$, $1 \leqslant p \leqslant \infty$.
  Then for any $x \in \mathbb{R}^m$ and $1 < p < \infty$, we have
  \[
    \| x \|_\infty \leqslant \| x \|_p \leqslant \| x \|_1 \leqslant m \| x \|_\infty.
  \]
\end{thm}

The only non-trivial part of Theorem~2 is the inequality $\| x \|_p \leqslant \| x \|_1$, whose proof is left to the interested readers.
Theorem~2 implies that topology induced by $\| \cdot \|_p$ on $\mathbb{R}^m$ does not depend on $p$.
For example, the closed unit ball $\overline{B_1(0)}$ is always compact no matter which $p$-norm we choose for $M = \mathbb{R}^m$.

However, the situation is quite different for vector spaces of infinite dimensions.
Here we investigate an important instance.

\begin{defn}[$\ell^2$-space]
  Let $A$ denote the set of all infinite sequences of real numbers.
  Define
  \[
    \ell^2 = \{ \langle a_n \rangle \subseteq A \colon \sum_n |a_n|^2 < \infty \}.
  \]
  $\ell^2$ is called the space of \textsf{square-summable} sequences.\footnote{This number $2$ can be replaced by $p \in [1, \infty]$.
  We choose $p=2$ for another reason: this norm can be induced by an inner product.}
\end{defn}

In some cases $A$ might consist of infinite sequences of complex numbers, but we will not go into that.
Clearly $A$ is equipped with a structure of vector space over $\mathbb{R}$ with componentwise addition and scalar multiplication.
Here is the main result.

\begin{thm}
  The functional
  \[
    \| \langle a_n \rangle \|_2 := \left( \sum_{n=1}^\infty |a_n|^2 \right)^{1/2}
  \]
  defines a norm on $\ell^2$, and $\ell^2$ is a vector subspace of $A$.
\end{thm}

\begin{proof}
  The only non-trivial part is the triangle inequality, which we verifies as follows.  For any positive integer $m$, we have
  \[
    \left( \sum_{k=1}^m |a_k + b_k|^2 \right)^{1/2} \leqslant
    \left( \sum_{k=1}^m |a_k|^2 \right)^{1/2} +
    \left( \sum_{k=1}^m |b_k|^2 \right)^{1/2}
  \]
  (this is the triangle inequality for $2$-norm on $\mathbb{R}^m$).
  Let $m \to \infty$ and use the comparison theorem for limits, we see that
  \[
    \| \langle a_n \rangle + \langle b_n \rangle \|_2 = 
    \left( \sum_{k=1}^\infty |a_k + b_k|^2 \right)^{1/2} \leqslant
    \left( \sum_{k=1}^\infty |a_k|^2 \right)^{1/2} +
    \left( \sum_{k=1}^\infty |b_k|^2 \right)^{1/2}
    = \| \langle a_n \rangle \|_2 + \| \langle b_n \rangle \|_2,
  \]
  which is needed to show.
\end{proof}

With this theorem, $\ell^2$ is now a metric space.\footnote{Unfortunately, the topology induced by the $\ell^2$-norm is not the product topology on denumerable sets of $\mathbb{R}$'s.}
It should be noted that the closed unit ball $\overline{M_1(0)}$ is \textit{not} compact for $M = \ell^2$.
To see why, we now explain that $\overline{B_1(0)}$ is not sequentially compact.
For $i \in \mathbb{N}$, let $e_i \in \ell^2$ denote the vector with all zero entries except at the $i^\text{th}$ position, at which it is $1$.
Clearly $\langle e_i \rangle$ is a sequence in $\overline{B_1(0)}$.
But $d(e_i, e_j) = \| e_i - e_j \|_2 = \sqrt{2}$ for any $i \ne j$, hence it has no convergent subsequence.

On the other hand, $\ell^2$ has a countable subset $\ell^2(\mathbb{Q})$, consisting of all square-summable sequences with rational entries only.
It is clear that $\overline{\ell^2(\mathbb{Q})} = \ell^2$, which means that $\ell^2(\mathbb{Q})$ is a \textsf{dense} subset of $\ell^2$.
To summarize, $\ell^2$ is a \textsf{separable} space, in the sense that it has a countable dense subset.

  \chapter{Differentiation on $\mathbb{R}$} 
\label{chap:diff}

It is time to shift our focus from spaces and functions to calculus.
In this semester we deal with one-variable calculus mainly.
As usual, let us start with the definition of differentiation.

\section{Basics of differentiation}
\label{sec:diff-def}

\begin{defn}
  Let $I$ be an open interval in $\mathbb{R}$ and $a \in I$.
  A function $f: I \to \mathbb{R}$ is said to be \textsf{differentiable} at $a$ when the limit
  \begin{equation}
    \label{eq:diff1}
    \lim_{x \to a} \frac{ f(x) - f(a) }{ x - a } 
  \end{equation}
  exists.  In this case that limit is called the \textsf{derivative} of $f$ at $a$ and is denoted by $f'(a)$.

  A function $f : (a,b) \to \mathbb{R}$ is \textsf{differentiable} in $(a,b)$ if it is differentiable at every point of $(a,b)$.
\end{defn}

By a change of variable $x = a + h$, (\ref{eq:diff1}) can also be written as
\begin{equation}
  \label{eq:diff2}
  \lim_{h \to 0} \frac{f(a+h)-f(a)}{h}.
\end{equation}
Let us start with a useful lemma.

\begin{thm}[Carath\'eodory theorem]
  \label{thm:caratheodory}
  Let $I$ be an open interval in $\mathbb{R}$ and $a \in I$.
  A function $f : I \to \mathbb{R}$ is differentiable at $a$ if and only if there is a function $F : I \to \mathbb{R}$ that is continuous at $a$ and
  \[
    f(x) = f(a) + F(x) \cdot (x-a), \qquad \forall \, x \in I.
  \]
\end{thm}

\begin{proof}
  Define, for $x \in I$,
  \[
    F(x) = 
    \begin{cases}
      \dfrac{f(x) - f(a)}{x-a}, & \text{when $x \ne a$}, \\
      c, & \text{when $x = a$},
    \end{cases}
  \]
  where $c$ is some constant.
  If $f'(a)$ exists, we let $c = f'(a)$ and this implies that $F$ is continuous at $a$.
  Conversely, if $F$ is continuous at $a$, then
  \[
    c = F(a) = \lim_{x \to a} F(x) = \lim_{x \to a} \frac{ f(x) - f(a) }{ x - a },
  \]
  which shows that $f$ is differentiable at $a$ with $f'(a) = c$.
\end{proof}

\noindent \textit{Remark.} $F$ may be called the function of \textit{slopes of secant lines} with respect to $a$.

With Theorem~\ref{thm:caratheodory}, it is easier to prove some standard calculus facts.

\begin{thm}
  \begin{enumerate}[(1)]
    \item Differentiability implies continuity.

    \item If $f$ and $g$ are differentiable at $x$ and $\lambda$ is a scalar, then $f+g$ and $\lambda f$ are both differentiable at $x$, with the following derivative formulae:
      \begin{align*}
	(f + g)'(x) &= f'(x) + g'(x), \\
	(\lambda f)'(x) &= \lambda \, f'(x).
      \end{align*}
      With these formulae, the set of real-valued functions that are differentiable at $x$ has the structure of a vector space over $\mathbb{R}$.

    \item (Product rule, a.k.a.~Leibniz formula)
      If $f$ and $g$ are differentiable at $x$ then so is their product $f \cdot g$, the derivative being
      \[
	(f \cdot g)'(x) = f'(x) g(x) + f(x) g'(x).
      \]

    \item (Quotient rule) If $f$ and $g$ are differentiable at $x$ and $g(x) \ne 0$ then their ratio $f/g$ is differentiable at $x$, the derivative being
      \[
	\left( \frac{f}{g} \right)'(x) = \frac{f'(x) g(x) - f(x) g'(x)}{(g(x))^2}.
      \]
      
    \item The derivative of a constant is zero, i.e., $c' = 0$.

    \item (Chain rule) If $f$ is differentiable at $x$ and $g$ is differentiable at $y = f(x)$, then their composite $g \circ f$ is differentiable at $x$, the derivative being
      \[
	(g \circ f)'(x) = g'(f(x)) f'(x).
      \]
  \end{enumerate}
\end{thm}

Their proofs are standard.  Here we only prove a few among them using Theorem~\ref{thm:caratheodory}.

\begin{proof}
  \begin{enumerate}[(1)]
    \item Let $f$ be differentiable at $x$.  Then there exists a function $F$ continuous at $x$ such that 
      \begin{equation}
	\label{eq:cara1}
	f(t) = f(x) + F(t) (t-x).
      \end{equation}
      Taking the limit as $t \to x$ at the right-hand side of (\ref{eq:cara1}), we see that
      \[
	\lim_{t \to x} (f(x) + F(t) (t-x)) = f(x) + F(x) (x-x) = f(x).
      \]
      This shows that $f$ is continuous at $x$.

      \addtocounter{enumi}{1}
    \item By Theorem~\ref{thm:caratheodory} there are functions $F$ and $G$ that are continuous at $x$ and
      \begin{align*}
	f(t) &= f(x) + F(t) (t-x), \\
	g(t) &= g(x) + G(t) (t-x).
      \end{align*}
      Multiplying both equations above yields
      \[
	f(t) g(t) = f(x) g(x) + \bigl(F(t)g(x) + f(x)G(t) + F(t)G(t)(t-x)\bigr) (t-x).
      \]
      Clearly $F(t) g(x) + f(x) G(t) + F(t) G(t) (t-x)$ is a function that is continuous at $t = x$, with
      \begin{align*}
	F(x) g(x) + f(x) G(x) + F(x) G(x) (x-x) &= F(x) g(x) + f(x) G(x) \\
	&= f'(x) g(x) + f(x) g'(x).
      \end{align*}
      Hence by Theorem~\ref{thm:caratheodory}, $f \cdot g$ is differentiable at $x$ with the derivative $f'(x) g(x) + f(x) g'(x)$. 

      \addtocounter{enumi}{2}
    \item There are functions $F$ that is continuous at $x$ and $G$ that is continuous at $y = f(x)$ such that
      \begin{align*}
	f(t) &= f(x) + F(t) (t-x), \\
	g(s) &= g(y) + G(s) (s-y).
      \end{align*}
      Now use $s = f(t)$ and $y = f(x)$ to get
      \begin{align*}
	g(f(t)) &= g(f(x)) + G(f(t)) (f(t) - f(x)) \\
	&= g(f(x)) + G(f(t)) F(t) (t-x).
      \end{align*}
      By (1) $f$ is continuous at $x$, so are $G(f(t))$ and $G(f(t)) F(t)$.
      Therefore by Theorem~\ref{thm:caratheodory}, $g \circ f$ is differentiable at $x$, with the derivative
      \[
	(g \circ f)'(x) = G(f(x)) F(x) = g'(f(x)) f'(x).
      \]
  \end{enumerate}
\end{proof}

With these facts in hand, we have lots of functions whose derivatives are easily calculated.

\begin{cor}
  \begin{enumerate}[(1)]
    \item The derivative of a polynomial $a_0 + a_1 x + \cdots + a_n x^n$ exists at every $x \in \mathbb{R}$ and equals $a_1 + 2a_2 x + \cdots + n a_n x^{n-1}$.
    \item The quotient $\dfrac{P(x)}{Q(x)}$ between two polynomials $P(x)$ and $Q(x)$ iss differentiable at every $x = a \in \mathbb{R}$ except for those $a$ with $Q(a) = 0$.
  \end{enumerate}
\end{cor}

\section{Mean value theorem}
\label{sec:MVT}

Now we formulate one of the most useful tools about differentiation, which is called the mean value theorem.
But we will state a more general statement.

\begin{thm}[Cauchy's mean value theorem]
  \label{thm:cmvt}
  Let $f, g$ be two real-valued functions defined on a closed interval $[a,b]$.
  Suppose that both $f$ and $g$ are continuous on $[a,b]$ and differentiable in $(a,b)$.
  Then there exists a point $c \in (a,b)$ such that
  \[
    g'(c) \cdot (f(b) - f(a)) = f'(c) \cdot (g(b) - g(a)).
  \]
\end{thm}

If we take $g(x) = x$, then the above theorem returns to its original form.
\begin{cor}[(Ordinary) Mean value theorem]
  \label{thm:mvt}
  Let $f$ be a real-valued function defined on a closed interval $[a,b]$.
  If $f$ is continuous on $[a,b]$ and differentiable in $(a,b)$,
  then there exists a point $c \in (a,b)$ such that
  \[
    f(b) - f(a) = f'(c) \cdot (b-a).
  \]
\end{cor}

The proof of the mean value theorem is based upon a useful observation, whose implication seems to be a corollary of the mean value theorem.
However this view should be avoided because it creates a circular argument.

\begin{lem}[Rolle's theorem]
  \label{thm:rolle}
  Let $f$ be a real-valued function which is continuous on $[a,b]$ and differentiable in $(a,b)$.
  If $f(a) = f(b)$, then there is a point $c \in (a,b)$ such that $f'(c) = 0$.
\end{lem}

\begin{proof}
  If $f$ is a constant function on $[a,b]$, then $f'$ vanishes in $(a,b)$ and the conclusion holds.
  Hence we may assume that $f$ is not a constant function on $[a,b]$.

  By the extreme value theorem, $f$ achieves its maximum $M$ and minimum $m$ on $[a,b]$.  Since $f$ is not a constant function, we must have $m < M$.
  However it is assumed that $f(a) = f(b)$, therefore $f$ achieves a local extremum at some \textit{interior} point $c \in (a,b)$.
  Let us show that $f'(c) = 0$.
  Since $f$ is differentiable at $c$,
  \[
    f'(c) = \lim_{x \to c\pm} \frac{ f(x) - f(c) }{x - c}.
  \]
  As $x$ approaches to $c$ from the both sides, the numerator in the limit $f(x) - f(c)$ has the same sign while the denominator has the opposite signs.
  By the limit comparison theorem we must have $0 \leqslant f'(c) \leqslant 0$, that implies $f'(c) = 0$.  This is what is asked to show.
\end{proof}

\begin{proof}[Proof of Cauchy's mean value theorem]
  Consider the following auxiliary function:
  \[
    h(x) = f(x) \cdot (g(b) - g(a)) - g(x) \cdot (f(b) - f(a)), \quad x \in [a,b].
  \]
  The function $h$ is also continuous on $[a,b]$ and differentiable in $(a,b)$ since it is a linear combination of $f$ and $g$.
  Moreover,
  \[
    h(a) = f(a) g(b) - g(a) f(b) = h(b),
  \]
  therefore $h$ satisfies the hypothesis of the Rolle's theorem.
  From that we conclude that there is a point $c \in (a,b)$ such that $h'(c) = 0$.  If we write out $h'(c)$, we get
  \[
    0 = h'(c) = f'(c) (g(b) - g(a)) - g'(c) (f(b) - f(a)),
  \]
  and the proof is finished.
\end{proof}

\noindent\textbf{Example.}
Let us regard a plane curve parametrized by $(f(t), g(t))$, where $f, g$ are real-valued functions which are continuous on $[a,b]$ and differentiable in $(a,b)$.
If $(f'(t), g'(t)) \ne 0$ for any $t \in (a,b)$, then by the Cauchy's mean value theorem we see that there is a moment $c \in (a,b)$ such that
\[
  (f'(c), g'(c)) \parallel (f(b), g(b)) - (f(a), g(a)).
\]
This says that the tangent vector $(f'(c), g'(c))$ is parallel to the secant vector that connects both endpoints $(f(a),g(a))$ and $(f(b), g(b))$.
This can be specialized to an ordinary situation when we simply take $f(x) = x$ and $g$ is just some differentiable function on $[a,b]$.

\medskip
From the mean value theorem, signs of derivatives could tell the trend of a function.  The proof is straightforward and omitted here.

\begin{cor}[First derivative test]
  Let $f$ be a real-valued function which is continuous on $[a,b]$ and differentiable in $(a,b)$.
  \begin{enumerate}[(i)]
    \item If $f'(x) > 0$ (resp.\ $f'(x) \geqslant 0$) for any $x \in (a,b)$, then $f$ is strictly increasing (resp.\ increasing) on $[a,b]$.
    \item If $f'(x) < 0$ (resp.\ $f'(x) \leqslant 0$) for any $x \in (a,b)$, then $f$ is strictly decreasing (resp.\ decreasing) on $[a,b]$.

  \end{enumerate}
\end{cor}

\noindent\textit{Remark.} It should be noted that if a function $f: I \to \mathbb{R}$ is strictly increasing and differentiable on an open interval $I \subseteq \mathbb{R}$, it can only be inferred that $f'(x) \geqslant 0$ for any $x \in I$; the equality might hold at some point.
A typical example is $f(x) = x^3$ defined on $\mathbb{R}$, whose derivative vanishes at $x = 0$.

Another corollary is about Lipschitz continuity.

\begin{cor}
  If a function $f: I \to \mathbb{R}$ has bounded derivatives on an intervarl $I \subseteq \mathbb{R}$, then $f$ is \textsf{Lipschitz continuous} on $I$; that is, there is a constant $C \geqslant 0$ such that
  \[
    |f(x) - f(y)| \leqslant C |x-y|, \qquad \forall\, x, y \in I.
  \]
\end{cor}

\begin{proof}
  We simply take $C = \sup \{ |f'(t)| \colon t \in I \}$, and apply the mean value theorem then take the absolute values: there is some $\xi$ between $x$ and $y$ such that
  \[
    |f(x) - f(y)| = |f'(\xi)| \cdot |x-y| \leqslant C |x-y|.
  \]
\end{proof}

Finally we prove a useful tool when dealing with limits.

\begin{thm}[L'Hospital's rule for $\dfrac{0}{0}$\footnote{We only deal with L'Hospital's rule for $\frac00$ here. For the proof of L'Hospital's rule for $\frac\infty\infty$, see Walter Rudin, \textit{Principles of Mathematical Analysis}, $3^\text{rd}$ edition, pp.109-110.}]
  Let $f$ and $g$ be differentiable defined on an interval $(a,b)$.
  Suppose that
  \begin{itemize}
    \item $f(x) \to 0$, $g(x) \to 0$ as $x \to b-$;
    \item $g(x) \ne 0$ and $g'(x) \ne 0$ for any $x$ in some neighborhood of $b$; and
    \item $\displaystyle \lim_{x \to b-} \frac{ f'(x) }{ g'(x) } = L \in \mathbb{R}$.
  \end{itemize}
  Then $\displaystyle \lim_{x \to b-} \frac{f(x)}{g(x)} = L$ as well.
\end{thm}

\begin{proof}
  Choose a point $c \in (a,b)$ such that $g(x) \ne 0$ and $g'(x) \ne 0$ for any $x \in [c,b)$.
  Extend the domains of $f$ and $g$ to $[c,b]$ by defining $f(b) = g(b) = 0$.
  Under the hypothesis $f$ and $g$ are continuous on $[c,b]$ and differentiable in $(c,b)$.
  For any $\varepsilon > 0$, there is a $\delta > 0$ such that $\left| \dfrac{f'(t)}{g'(t)} - L \right| < \varepsilon$ whenever $t \in (b-\delta, b) \subseteq (c,b)$.
  Thus, whenever $x \in (b-\delta, b)$, there is a $\xi \in (x,b)$ such that
  \[
    \frac{f(x)}{g(x)} = \frac{ f(x) - f(b) }{ g(x) - g(b) } = \frac{ f'(\xi) }{ g'(\xi) }
  \]
  by Cauchy's mean value theorem.  Hence whenever $b - \delta < x < b$, we have
  \[
    \left| \frac{f(x)}{g(x)} - L \right| = \left| \frac{ f'(\xi) }{ g'(\xi) } - L \right| < \varepsilon.
  \]
  Since $\varepsilon$ is arbitrary, we conclude that $\displaystyle \lim_{x \to b-} \frac{f(x)}{g(x)} = L$ as well.

  For the case $b = +\infty$, we use a change of variable $y = \dfrac1x$; as $x \to \infty$, $y \to 0+$.
  Define two functions
  \[
    F(y) = 
    \begin{cases}
      f(1/y), & \text{if $y > 0$}; \\
      0     , & \text{if $y = 0$};
    \end{cases}
    \qquad
    G(y) = 
    \begin{cases}
      g(1/y), & \text{if $y > 0$}; \\
      0     , & \text{if $y = 0$};
    \end{cases}
  \]
  Then for $y > 0$, $F'(y) = -f'(y)/y^2$ and $G'(y) = -g'(y)/y^2$.
  It is not hard to realize that l'Hospital's rule applies to $F, G$ at $y \to 0+$ implies that to $f,g$ at $x \to +\infty$.
\end{proof}

\noindent\textit{Remark.} Although we prove the case for limits from the left, the same works for limits from the right.
Also noted is that the process of l'Hospital's rule usually goes \textit{backward}, as the following example shows.
Since
\[
  \lim_{x \to 0} \frac{ 2 \eu^{2x} }{1} = 2,
\]
hence by l'Hospital's rule we have
\[
  \lim_{x \to 0} \frac{ \eu^{2x} - 1 }{ x } \stackrel{\textrm{H}}{=} \lim_{x \to 0} \frac{ 2 \eu^{2x} }{ 1 } = 2.
\]

There are pathological examples about differentiation.
First we encounter a differentiable function whose derivative is not continuous.
Consider the function $f: \mathbb{R} \to \mathbb{R}$ defined as
\[
  f(x) = 
  \begin{cases}
    x^2 \sin \dfrac{1}{x}, & \text{if $x \ne 0$}; \\
    0,                     & \text{if $x = 0$}.
  \end{cases}
\]
The function $f$ is certainly differentiable at every $x \ne 0$, with derivative
\[
  f'(x) = 2x \sin \dfrac1x - \cos \dfrac1x, \qquad x \ne 0.
\]
Nevertheless, $f$ is also differentiable at $x = 0$, as we verify the difference-quotient:
\[
  \lim_{x\to 0} \frac{ f(x) - f(0) }{ x }
  = \lim_{x\to 0} \frac{x^2 \sin \frac1x}{x} 
  = \lim_{x\to 0} x \sin \frac1x = 0.
\]
(The last equality follows from the squeeze theorem.)

Note that $f'(x)$ does not converge to $0$ as $x \to 0$, as the term $\cos \dfrac1x$ oscillates between $\pm1$.
Hence its derivative $f'$ is discontinuous at $x = 0$. 
Analogously, the function $\displaystyle x^\alpha \sin^\beta \dfrac1x$ behaves differently if suitable constants $\alpha$ and $\beta$ are specified.

Although the derivative of a differentiable function may or may not be continuous, its discontinuity may never \textit{jump}, as the following theorem shows.

\begin{thm}
  If $f$ is differentiable on an open interval $I$ in $\mathbb{R}$ then its derivative function $f'$ has the intermediate value property. That is, suppose $[a,b]  \subseteq I$ with $f'(a) = \alpha$ and $f'(b) = \beta$; then for any $\gamma$ between $\alpha$ and $\beta$, there is a point $c \in (a,b)$ such that $f'(c) = \gamma$.
\end{thm}

\noindent\textit{Remark.} If $f'$ is already continuous, we may apply the intermediate value theorem for continuous functions directly.
However we now supply a proof that does not depend on the continuity of $f'$.

\begin{proof}
  Without loss of generality, we may assume that $\alpha > \gamma > \beta$; otherwise consider $-f$ instead of $f$.
  By considering $f(x) - \gamma x$, we may further assume that $\alpha > \gamma = 0 > \beta$. 

  Now we have
  \[
    0 < \alpha = f'(a) = \lim_{x \to a+} \frac{ f(x) - f(a) }{ x - a }.
  \]
  For $\varepsilon = \alpha/2 > 0$, there is a $\delta > 0$ such that whenever $x \in (a, a+\delta)$, we have
  \[
    \left| \frac{f(x) - f(a)}{x - a} - \alpha \right| < \varepsilon = \frac{\alpha}{2}.
  \]
  After moving terms around, we see that for any $x \in (a, a + \delta)$,
  \[
    \frac{f(x) - f(a)}{x-a} > \frac{\alpha}{2} \implies
    f(x) > f(a) + \frac{\alpha}{2} (x-a) > f(a).
  \]
  \textit{This shows that $f(a)$ cannot be the maximum for $f$ in $[a,b]$.}
  Similarly, $f(b)$ cannot be the maximum for $f$ in $[a,b]$ either.
  Since from the extreme value theorem the maximum of $f$ in $[a,b]$ must occur somewhere, it must occur at some \textit{interior} point $c \in (a,b)$.
  Since $f$ is differentiable at $c$, we must have $f'(c) = 0 = \gamma$.
\end{proof}

To state the following corollary, we need the following definitions.
Let $x$ be an interior point of an interval $I \subseteq \mathbb{R}$.
Then we define, for a function $f$ on $I$ whose following notations make sense,
\[
  f'(x-) = \lim_{t \to x-} f'(t); \qquad
  f'(x+) = \lim_{t \to x+} f'(t).
\]
That is, $f'(x\pm)$ are the one-sided limits of $f'$ at $x$.
One should not confuse them with \textit{one-sided derivatives}, whose definitions and notations are
\[
  f'_-(x) = \lim_{t \to x-} \frac{ f(t) - f(x) }{ t - x }; \qquad
  f'_+(x) = \lim_{t \to x+} \frac{ f(t) - f(x) }{ t - x }.
\]
\begin{cor}
  The derivative of a differentiable function never has a jump discontinuity in the sense that $f'(x-)$ and $f'(x+)$ both exist but $f'(x-) \ne f'(x+)$.
\end{cor}

\section{Higher derivatives, Taylor approximation}
\label{sec:higher-diff}

If a function $f$ is differentiable in an interval $I$, its derivative $f'$ may be viewed as a function in $I$, called the \textit{derivative function}.
Hence it is reasonable to ask whether the function $f'$ can be differentiated or not.
The derivative of $f'$, if it exists, is the \textsf{second derivative} of $f$,
\[
  f''(x) = (f')'(x) = \lim_{t \to x} \frac{ f'(t) - f'(x) }{ t - x }.
\]

Higher derivatives are defined inductively and written $f^{(r)} = ( f^{(r-1)} )'$ for $r \in \mathbb{N}$, where $f^{(0)} = f$ and $f^{(1)} = f'$.
If $f^{(r)}(x)$ exists then $f$ is \textsf{$r^\text{th}$-order differentiable} at $x$.
If $f^{(r)}(x)$ exists for each $x \in (a,b)$ then $f$ is \textsf{$r^\text{th}$-order differentiable}.
If $f^{(r)}(x)$ exists for all $r \in \mathbb{N} \cup \{0\}$ and all $x$ then $f$ is called \textsf{infinitely differentiable} or \textsf{smooth}.
The \textsf{zeroth derivative} of $f$ is $f$ itself, $f^{(0)}(x) = f(x)$.
Clearly, if $f$ is $r^{\text{th}}$-order differentiable with $r \geqslant 1$ then $f^{(r-1)}$ is continuous.

If $f$ is differentiable and its derivative function $f'(x)$ is a continuous function of $x$, then $f$ is called \textsf{continuously differentiable} and we say that $f$ is \textsf{of class $\mathcal{C}^1$}.
Likewise, if $f$ is $r^{\text{th}}$-order differentiable and $f^{(r)}(x)$ is a continuous function of $x$ then $f$ is \textsf{continuously $r^{\text{th}}$ differentiable} and we say that $f$ is \textsf{of class $\mathcal{C}^r$}.
If $f$ is smooth then by the proceeding discussion it is of class $\mathcal{C}^r$ for all finite $r$ and we say that $f$ is \textsf{of class $\mathcal{C}^\infty$}.
To round out the notation we say that a continuous function is of class $\mathcal{C}^0$.

Thinking of $\mathcal{C}^r$ as the set of functions of class $\mathcal{C}^r$, we have the regularity hierarchy
\[
  \mathcal{C}^0 \supseteq \mathcal{C}^1 \supseteq \cdots \supseteq \bigcap_{r \in \mathbb{N}} \mathcal{C}^r = \mathcal{C}^\infty.
\]
Each inclusion $\mathcal{C}^r \supseteq \mathcal{C}^{r+1}$ is proper, i.e. there are functions of class $\mathcal{C}^r$ which are not of class $\mathcal{C}^{r+1}$.
For example, the function $f(x) = |x|$ is of class $\mathcal{C}^0$ but not of class $\mathcal{C}^1$.

The $r^{\text{th}}$-order \textsf{Taylor polynomial} of an $r^{\text{th}}$-order differentiable function $f$ at $a$ is\footnote{In the theory of power series, we usually define $0^0 = 1$, which occurs at the constant term.}
\begin{equation}
  \label{eq:taylor1}
  P_r(x) = f(a) + f'(a) (x-a) + \frac{f''(a)}{2!} (x-a)^2 + \cdots + \frac{f^{(r)}(a)}{r!} (x-a)^r = \sum_{k=0}^r \frac{ f^{(k)}(a) }{k!} (x-a)^k.
\end{equation}

By a change of variable $x = a + h$, (\ref{eq:taylor1}) can also be written as
\begin{equation}
  \label{eq:taylor2}
  P_r(a+h) = f(a) + f'(a) h + \frac{f''(a)}{2!} h^2 + \cdots + \frac{f^{(r)}(a)}{r!} h^r = \sum_{k=0}^r \frac{ f^{(k)}(a) }{k!} h^k.
\end{equation}
The coefficients $\dfrac{f^{(k)}(a)}{k!}$ are constants, and the variable is $x$ (or $h$).
Differentiation of $P_r$ with respect to $x$ at $x = a$ gives
\[
  P_r^{(k)}(a) = f^{(k)}(a), \qquad k = 0, 1, \dots, r.
\]
In fact, $P_r$ is the unique polynomial of degree at most $r$ with these properties.

\begin{thm}[Taylor theorem]
  Let $I$ be an open interval in $\mathbb{R}$.
  Assume that $f: I \to \mathbb{R}$ is $r^{\text{th}}$-order differentiable at $a \in I$.  Then
  \begin{enumerate}[(i)]
    \item $P_r$ approximates $f$ to order $r$ at $a$ in the sense that the Taylor \textsf{$r^{\text{th}}$-order remainder}
      \[
        R_r(h) := f(a+h) - P_r(a+h)
      \]
      is $r^{\text{th}}$-order flat at $h = 0$; i.e., $R_r(h) / h^r \to 0$ as $h \to 0$.

      In particular, the Taylor polynomial $P_r(x)$ is the only polynomial of degree at most $r$ with the $r^{\text{th}}$-order flat property.

    \item If, in addition, $f$ is $(r+1)^{\text{st}}$-order differentiable in $I$ then for some $\theta$ between $a$ and $a+h$ such that
      \begin{equation}
	\label{eq:r-remainder}
	R_r(h) = \frac{ f^{(r+1)}(\theta) }{ (r+1)! } h^{r+1}.
      \end{equation}
  \end{enumerate}
\end{thm}

\noindent\textit{Remark.} (\ref{eq:r-remainder}) is the \textsf{Lagrange form} of the remainder.
If $|f^{(r+1)}(\theta)| \leqslant M$ for all $\theta \in I$ then
\[
  |R_r(h)| \leqslant \frac{ M h^{r+1} }{(r+1)!},
\]
an estimate that is valid uniformly with respect to $a$ and $a+h$ in $I$, whereas (i) is only an infinitesimal pointwise estimate.
Of course (ii) requires stronger hypotheses than (i).

\begin{proof}
  \begin{enumerate}[(i)]
    \item The first $r$ derivatives of $R_r(h)$ exist and equal $0$ at $h = 0$.
      If $h > 0$ then the repeated application of the mean value theorem give
      \begin{align*}
	R_r(h) &= R_r(h) - 0 = R'(\theta_1) h = (R'(\theta_1)-0) h = R''(\theta_2) \theta_1 h \\
	&= \cdots = R^{(r-1)}(\theta_{r-1}) \theta_{r-2} \cdots \theta_1 h,
      \end{align*}
      where $0 < \theta_{r-1} < \cdots < \theta_2 < \theta_1 < h$.
      Thus
      \[
	\left| \frac{R_r(h)}{h^r} \right|
	= \left| \frac{ R^{(r-1)}(\theta_{r-1}) \theta_{r-2} \cdots \theta_1 h }{ h^r } \right|
	\leqslant \left| \frac{ R^{(r-1)}(\theta_{r-1}) }{h} \right|
	\leqslant \left| \frac{ R^{(r-1)}(\theta_{r-1}) - 0 }{\theta_{r-1}}
	\right| \to 0,
      \]
      as $h \to 0$, with the last limit equals $0$ since $R^{(r-1)}$ is differentiable at $0$.

      On the other hand, if $h < 0$ the same is true for $h < \theta_1 < \cdots < \theta_{r-1} < 0$.
      
    \item Let $h \geqslant 0$ and consider two functions in $h$:
      \[
	F(h) = f(a+h) - \sum_{k=0}^r \frac{ f^{(k)}(a) }{k!} h^k, \quad
	\text{and} \quad
	G(h) = h^{r+1}.
      \]
      Both $F$ and $G$ are $\mathcal{C}^r$ on $[0,h]$ and have the $(r+1)^{\text{st}}$-order derivatives in $(0,h)$.
      Also note that for all $k=0,1,\dots,r$ we have $F^{(k)}(0) = 0 = G^{(k)}(0)$ and $G^{(k)}(h) \ne 0$ for $h \ne 0$. 
      By repeatedly applying the generalized mean value theorem, we see that
      \[
	\frac{F(h)}{G(h)} = \frac{F(h)-F(0)}{G(h)-G(0)} = \frac{F'(h_1)}{G'(h_1)} = \cdots = \frac{F^{(r)}(h_r)}{G^{(r)}(h_r)} = \frac{F^{(r+1)}(h_{r+1})}{G^{(r+1)}(h_{r+1})}
      \]
      for some $0 < h_{r+1} < h_r < \cdots < h_1 < h$.
      Finally we note that, with $\theta = a + h_{r+1}$,
      \[
	F^{(r+1)}(h_{r+1}) = f^{(r+1)}(\theta) \quad \text{and} \quad
	G^{(r+1)}(h_{r+1}) = (r+1)!,
      \]
      and conclude that
      \[
	R_r(h) = f(a+h) - P_r(h) = F(h) = \frac{F^{(r+1)}(h_{r+1})}{G^{(r+1)}(h_{r+1})} G(h) = \frac{f^{(r+1)}(\theta)}{(r+1)!} h^{r+1}.
      \]
      The same can be said for $h \leqslant 0$.
  \end{enumerate}
\end{proof}

Let us come back to the meaning of differentiation.
Recall that $f$ is differentiable at $a$ provided that the limit
\begin{equation}
  \label{eq:diff}
  f'(a) = \lim_{x \to a} \frac{ f(x) - f(a) }{ x - a }
\end{equation}
exists.  In fact, (\ref{eq:diff}) can be rewritten as 
\[
  \lim_{x \to a} \frac{ f(x) - (f(a) + f'(a)\cdot(x-a)) }{x - a} = 0.
\]
It is immediately recognized that $f(a) + f'(a)\cdot(x-a)$ is the first-order Taylor polynomial of $f$ at $a$.
Hence we may use the notation
\[
  f(x) \approx f(a) + f'(a) \cdot (x-a), \qquad \text{when $x$ is near $a$}.
\]
That is, the value $f(x)$ when $x$ is near $a$ can be approximated by the linear function $f(a) + f'(a) \cdot (x-a)$.
If we denote its error by $\varepsilon(h) = \varepsilon(x-a)$, then by (i) of Taylor theorem,
\[
  \lim_{h \to 0} \frac{ \varepsilon(h) }{ h } = 0.
\]
Another way to address this phenomenon is that the function $f(a) + f'(a) \cdot (x-a)$ is the \textit{best} linear function that passes through the point $(a, f(a))$ and approximates $f(x)$.

Of course when $f$ has higher-order derivatives, it can be better approximated by higher-order Taylor polynomials, with its error well-controlled.

\section{Inverse function theorem}
\label{sec:IFT}

A strictly monotone continuous function $f : (a,b) \to \mathbb{R}$ bijects $(a,b)$ onto some interval $(c,d)$ where $c = f(a+)$ and $d = f(b-)$ in the increasing case.
(We permit $c$ or $d$ to be infinite.)
It is a homeomorphism $(a,b) \to (c,d)$ and its inverse $f^{-1} : (c,d) \to (a,b)$ is also a homeomorphism.

Does differentiability of $f$ imply differentiability of $f^{-1}$?
If $f'$ never vanishes the answer is ``yes.''
Keep in mind, however, the function $f(x) = x^3$.
It shows that differentiability of $f^{-1}$ fails when $f'(x) = 0$.
For the inverse function is $y \mapsto y^{1/3}$, which is not differentiable at $y = 0$.

\begin{thm}[Inverse function theorem in $\mathbb{R}^1$]
  Let $I$ be an open interval in $\mathbb{R}$ and $f : I \to \mathbb{R}$ is one-to-one and continuous.
  If $b = f(a)$ for some $a \in I$ and if $f'(a)$ exists and is nonzero, then $f^{-1}$ is differentiable at $b$ and
  \[
    (f^{-1})'(b) = \frac{1}{f'(a)} = \frac{1}{f'(f^{-1}(b))}.
  \]
\end{thm}

\begin{proof}
  Since $f$ is one-to-one and continuous on an open interval $I \subseteq \mathbb{R}$,
  $J = f(I)$ is also an interval, $f$ is strictly monotone, and $f^{-1} : J \to I$ is a bijection.
  The continuity of $f^{-1}$ can be verified by checking the sequential convergence preservation condition.

  Now we prove that $f^{-1}$ is differentiable at $b = f(a)$, where $f'(a) \ne 0$.
  By the Carath\'eodory theorem, there is a function $F: I \to \mathbb{R}$ which is continuous at $a$, $F(a) = f'(a)$, and for any $x \in I$,
  \begin{equation}
    \label{eq:cara-diff}
    f(x) = f(a) + F(x) \cdot (x-a).
  \end{equation}
  Note that $F(x) \ne 0$ in some neighborhood of $a$ because $F$ is continuous at $a$ and $F(a) \ne 0$.
  Now write $g = f^{-1}$, $x = g(y)$, $a = g(b)$, and move terms around in (\ref{eq:cara-diff}) to get
  \begin{equation}
    \label{eq:cara-diff2}
    g(y) = g(b) + \frac{1}{F(g(y))} (y - b).
  \end{equation}
  Since $g$ is continuous in $y$ and $F$ is continuous and nonzero at $a = g(b)$, $\dfrac{1}{F \circ g}$ is also continuous at $b$.
  Therefore by the Carath\'eodory theorem again (but in the reverse direction,) $g = f^{-1}$ is differentiable at $b$ with
  \[
    (f^{-1})'(b) = g'(b) = \frac{1}{F(g(b))} = \frac{1}{F(a)} = \frac{1}{f'(a)} = \frac{1}{f'(f^{-1}(b))}.
  \]
\end{proof}

\noindent\textbf{Example.} Now we deal with fractional power of a variable.
Let $n$ be a positive integer, and consider the function $y = f(x) = x^n$ for $x > 0$.
The function $f$ is one-to-one and continuous in $(0, \infty)$, with derivative $f'(x) = n x^{n-1}$ that never vanishes.
Therefore the inverse function theorem applies, with
\[
\frac{\dd x}{\dd y} = \frac{1}{f'(x)} = \frac{1}{n x^{n-1}} = \frac{1}{n} \cdot y^{\frac{1}{n}-1}
\]
for $x = y^{1/n}$, $y > 0$.
Then by the chain rule,
\[
  (x^{m/n})' = ( (x^{1/n})^m )' = m x^{\frac{m-1}{n}} \cdot (x^{1/n})'
  = m x^{\frac{m-1}{n}} \cdot \frac{1}{n} x^{\frac{1}{n}-1} 
  = \frac{m}{n} x^{\frac{m}{n} - 1},
\]
for $x > 0$.
These differentiation formulae are consistent with those in integer powers.

\section{More on convex functions}
\label{sec:convex-diff}

Let $I$ be an interval in $\mathbb{R}$.
Recall that a function $f: I \to \mathbb{R}$ is called a \textsf{convex function} if for any $x, y \in I$ with $x < y$ and for any $0 \leqslant \lambda \leqslant 1$, the following inequality holds:
\begin{equation}
  \label{eq:convex}
  f \bigl( (1-\lambda) x + \lambda y \bigr) \leqslant (1-\lambda) f(x) + \lambda f(y).
\end{equation}
We have shown, as a corollary, that for any $a, x, b \in I$ with $a < x < b$, the following inequalities hold for a convex function $f: I \to \mathbb{R}$:
\begin{equation}
  \label{eq:convex-diff}
  \frac{ f(x) - f(a) }{ x - a } \leqslant \frac{ f(b) - f(a) }{ b - a } \leqslant \frac{ f(b) - f(x) }{ b - x }.
\end{equation}

Although it is not in the definition, we have shown that a convex function $f$ in $I$ is always continuous.
We can prove further properties for convex functions based on (\ref{eq:convex-diff}).

\begin{thm}
  \label{thm:convex-onesided}
  Let $f$ be a convex real function on an open interval $I$.
  Then for each $x \in I$ both one-sided derivatives $f'_{\pm}$ exist at $x$.
  Furthermore,
  \[
    f'_+(a) \leqslant f'_-(b)
  \]
  for any $a, b \in I$ with $a < b$.
\end{thm}

\begin{proof}
  Firstly we note that the function
  \[
    x \mapsto \frac{ f(b) - f(x) }{ b - x }, \qquad x < b \text{ in $I$},
  \]
  is an increasing function in $x$, by looking at the latter inequality of (\ref{eq:convex-diff}).
  Likewise, the function
  \[
    x \mapsto \frac{ f(x) - f(a) }{ x - a }, \qquad a < x \text{ in $I$},
  \]
  is also an increasing function in $x$, by looking at the former inequality of (\ref{eq:convex-diff}).

  Let us rewrite the inequalities (\ref{eq:convex-diff}) as
  \begin{align}
    \frac{ f(t) - f(x) }{ t - x } &\leqslant \frac{ f(b) - f(x) }{ b - x }, \qquad t < x < b \text{ in $I$}; \label{eq:cs1} \\
    \frac{ f(x) - f(a) }{ x - a } &\leqslant \frac{ f(t) - f(x) }{ t - x }, \qquad a < x < t \text{ in $I$}. \label{eq:cs2}
  \end{align}

  Let us take $t \to x-$ in (\ref{eq:cs1}).
  Since the fraction increases as $t$ increases and is bounded from above by the right-hand fraction, we see that
  \[
    f'_-(x) = \lim_{t \to x-} \frac{ f(t) - f(x) }{ t - x }
  \]
  exists by the least upper bound property.
  A similar argument applied to (\ref{eq:cs2}) as $t \to x+$ shows that
  \[
    f'_+(x) = \lim_{t \to x+} \frac{ f(t) - f(x) }{ t - x }
  \]
  exists as well.
  Moreover, for $a, b \in I$ and $a < b$, we have
  \[
    f'_+(a) \leqslant \frac{ f(b) - f(a) }{ b - a } \leqslant f'_-(b),
  \]
  which is established by the monotone processes above.
  The proof is completed.
\end{proof}

A related result on differentiable convex functions can also be obtained.
\begin{thm}
  Suppose that $f$ is differentiable on an open interval $I$.
  Then $f$ is convex in $I$ if and only if $f'$ is increasing on $I$.
\end{thm}

\begin{proof}
  If $f$ is differentiable and convex in $I$, then $f'$ is increasing by (ii) of Theorem~\ref{thm:convex-onesided}.

  Conversely, suppose that $f$ is differentiable in $I$ and $f'$ is increasing in $I$.  Let $a, b \in I$ and $0 \leqslant \lambda \leqslant 1$.  Our goal is to prove the inequalities (\ref{eq:convex}).
  The cases $\lambda \in \{ 0, 1 \}$ hold trivially.
  For $0 < \lambda < 1$, let $x = (1 - \lambda) a + \lambda b$.
  Then $a < x < b$.
  By the mean value theorem, there are points $\xi \in (a,x)$ and $\eta \in (x,b)$ such that
  \begin{equation}
    \label{eq:secant}
    \frac{ f(x) - f(a) }{ x - a } = f'(\xi) \leqslant f'(\eta) = \frac{ f(b) - f(x) }{ b - x }.
  \end{equation}
  Using $x = (1 - \lambda) a + \lambda b$, (\ref{eq:secant}) becomes
  \begin{align*}
    & & \frac{ f(x) - f(a) }{ \lambda (b-a) } &\leqslant \frac{ f(b) - f(x) }{ (1 - \lambda) (b-a) } \\
    & \iff & (1 - \lambda) (f(x) - f(a)) &\leqslant \lambda (f(b) - f(x)) \\
    &  \iff & f( (1-\lambda)a + \lambda b ) = f(x) & \leqslant (1-\lambda) f(a) + \lambda f(b).
  \end{align*}
  Since $a, b$, and $\lambda$ are arbitrary, we conclude that $f$ is convex in $I$.
\end{proof}

From Theorem 3 we arrive the second derivative test, which is very useful when drawing graphs of differentiable functions.

\begin{cor}[Second derivative test]
  Suppose that $f$ is twice differentiable on an open interval $I$.
  Then $f$ is convex in $I$ if and only if $f''(x) \geqslant 0$ for any $x \in I$.
\end{cor}

\begin{proof}
  A differentiable function is convex if and only if its derivative function is increasing.
  If it is also twice differentiable, then $f'$ is increasing if and only if $f'' \geqslant 0$.
\end{proof}

A function $f$ is said to have a \textsf{proper maximum} (resp.\ \textsf{proper minimum}) at $x_0$ if and only if there is a $\delta > 0$ such that $f(x) < f(x_0)$ (resp.\ $f(x) > f(x_0)$) for all $0 < |x - x_0| < \delta$.
As far as proper extrema are concerned, convex functions behave like strictly increasing functions.

\begin{thm}
  \begin{enumerate}[(i)]
    \item If $f$ is convex on a nonempty, open interval $(a,b)$, then $f$ has no proper maximum in $(a,b)$.
    \item If $f$ is convex on $[0,\infty)$ and has a proper minimum, then $f(x) \to \infty$ as $x \to \infty$.
  \end{enumerate}
\end{thm}

\begin{proof}
  \begin{enumerate}[(i)]
    \item Suppose $f$ achieves a proper maximum at $x_0$.
      Then there are points $y, z \in (a,b)$ such that $y < x_0 < z$ and $f(y) < f(x_0) > f(z)$.
      But in the case the point $(x_0, f(x_0))$ lies strictly above the secant segment joining the points $(y, f(y))$ and $(z, f(z))$ on the graph of the function $f$.  Hence $f$ cannot be convex in $(a,b)$.

    \item Let $f$ achieves a proper minimum at $x_1 \geqslant 0$.
      Then there is a point $t > x_1$ such that $f(t) > f(x_1)$.
      By convexity, for any $x > t$ the following inequality holds:
      \[
	f(x) \geqslant f(t) + \frac{ f(t) - f(x_1) }{ t - x_1 } (x - t) =: L(x).
      \]
      Since $L(x) \to \infty$ as $x \to \infty$, $f(x) \to \infty$ as $x \to \infty$ as well by the comparison theorem.
  \end{enumerate}
\end{proof}

\section{A peculiar smooth function}
\label{sec:smooth-not-analytic}

We assume familiarity of the exponential function $(\eu^x)' = \eu^x$ for this subsection.
Consider the following function:
\begin{equation}
  \label{eq:vanish-e}
  \mathfrak{e}(x) = 
  \begin{cases}
    \eu^{-1/x}, & \text{if $x > 0$}; \\
    0         , & \text{if $x \leqslant 0$}.
  \end{cases}
\end{equation}
Obviously $\mathfrak{e}$ is smooth in $\mathbb{R} \setminus \{ 0 \}$.
Below we are going to show that $\mathfrak{e}$ is smooth at $0$ as well.
The following lemma is needed.

\begin{lem}
  \label{lem:diff-e}
  Let $n$ be a positive integer.
  Then
  \[
    \lim_{x \to 0+} \frac{ \mathfrak{e}(x) }{ x^n } = 0.
  \]
\end{lem}

\begin{proof}
  We first make a change of variable $y = 1/x$.
  Then
  \[
    \lim_{x \to 0+} \frac{ \mathfrak{e}(x) }{ x^n } = \lim_{y \to +\infty} \frac{ \eu^{-y} }{ (1/y)^n } = \lim_{y \to +\infty} \frac{ y^n }{ \eu^y }.
  \]
  Applying l'Hospital's rule $n$ times yields
  \[
    \lim_{y \to +\infty} \frac{ y^n }{ \eu^y } = \lim_{y \to +\infty} \frac{n!}{ \eu^y } = 0.
  \]
  Therefore $\displaystyle \lim_{x \to 0+} \frac{ \mathfrak{e}(x) }{ x^n } = 0$ as well.
\end{proof}

Now we reach the ultimate goal.
\begin{thm}
  \label{thm:weird-e}
  The function $\mathfrak{e}$ defined by $(\ref{eq:vanish-e})$ is smooth at $0$; in fact, $\mathfrak{e}^{(n)}(0) = 0$ for any $n \in \mathbb{N}$.
\end{thm}

\begin{proof}
  Let us first work out the first derivative function of $\mathfrak{e}$ in $(0, \infty)$:
  \[
    \mathfrak{e}'(x) = \frac{ \eu^{-1/x} }{x^2}, \qquad x > 0.
  \]
  And it is differentiable at $x=0$ because
  \[
    \lim_{x \to 0+} \frac{ \mathfrak{e}(x) - \mathfrak{e}(0) }{ x - 0 }
    = \lim_{x \to 0+} \frac{ \eu^{-1/x} }{ x } = 0 = \mathfrak{e}'(0).
  \]
  by Lemma~\ref{lem:diff-e} (the other side $x \to 0-$ is trivial).
  
  By mathematical induction, we see that
  \[
    \mathfrak{e}^{(n)}(x) = 
    \begin{cases}
      \dfrac{ p_n(x) \eu^{-1/x} }{ x^{2n} }, & \text{if $x > 0$}, \\
      0                                    , & \text{if $x \leqslant 0$},
    \end{cases}
  \]
  for some polynomial $p_n(x)$ of degree at most $n$.
  Therefore by Lemma 6 $\mathfrak{e}^{(n)}$ is differentiable at $x = 0$ and we can process the next derivative.  Then mathematical induction on $n$ goes through.
\end{proof}

With Theorem~\ref{thm:weird-e}, we see that the $r^{\text{th}}$-order Taylor polynomial of $\mathfrak{e}$ at $0$ is $0$ for any $r \in \mathbb{N}$, but $\mathfrak{e}$ itself is not identically zero in any neighborhood of $0$.
We will return to this point in the section on power series.

  \chapter{Integration on $\mathbb{R}$}
\label{sec:integration}

\section{Riemann integrals}
\label{sec:Riemann-integral}

We now start developing an integration theory that covers most situations in elementary calculus class.
The motivation is that when $f : [a,b] \to \mathbb{R}$ is a non-negative, continuous function over $[a,b]$, the definite integral
\[
  \int_a^b f(x) \, \dd x
\]
should represent (the value of) the area of the region in $\mathbb{R}^2$ that is bounded by the lines $x=a$, $x=b$, $y=0$, and the graph $y = f(x)$.
When the graph $y = f(x)$ is ``straight'', i.e. composed of line segments, the area can be calculated by ancient geometry.
However, even in the ``easiest'' case, the computation of the area of a circle requires a limiting process.
So we need a rigorous foundation for integrals to make sense of what area means.
Here we deal with integral theory in $\mathbb{R}^1$.
That theory on $\mathbb{R}^n$ will be discussed in the analysis class in the following semester.

\begin{defn}
  Let $[a,b]$ be a bounded closed interval in $\mathbb{R}$.
  A \textsf{partition} $\mathcal{P}$ for $[a,b]$ is a \textit{finite} collection of points that includes the endpoints $a$ and $b$.
  Conventionally we write
  \begin{equation}
    \label{eq:part}
    \mathcal{P} = \{ a = x_0 < x_1 < \cdots < x_{n-1} < x_n = b \}.
  \end{equation}
  Given a partition $\mathcal{P}$ for $[a,b]$ as in (\ref{eq:part}), the interval $I_i = [x_{i-1}, x_i]$ will be called the $i^{\text{th}}$ subinterval determined by $\mathcal{P}$, $i = 1, 2, \dots, n$.
  The length of $I_i$ is denoted by $\Delta x_i = |I_i| = x_i - x_{i-1}$.
  The \textsf{mesh} of a partition $\mathcal{P}$ is the length of the longest subinterval deteremined by $\mathcal{P}$, and is denoted by $\| \Delta \mathcal{P} \|$.
\end{defn}

Now we define the following numbers, which are bounds for the area of the region under the graph $y = f(x)$ when the notion ``area'' is appropriate and fits into our intuition.

\begin{defn}
  Let $\mathcal{P} = \{ a = x_0 < \cdots < x_n = b \}$ be a partition for a bounded closed interval $[a,b]$.
  For a bounded function $f: [a,b] \to \mathbb{R}$, define the following sums.
  \begin{enumerate}[(i)]
    \item The \textsf{upper sum} of $f$ over $\mathcal{P}$ is
    \[
      U(f,\mathcal{P}) = \sum_{i=1}^n M_i \Delta x_i,
    \]
    where $M_i = \sup \{ f(x) \colon x \in I_i \}$, $i = 1, \dots, n$.

    \item The \textsf{lower sum} of $f$ over $\mathcal{P}$ is
    \[
      L(f,\mathcal{P}) = \sum_{i=1}^n m_i \Delta x_i,
    \]
    where $m_i = \inf \{ f(x) \colon x \in I_i \}$, $i = 1, \dots, n$.
  \end{enumerate}
\end{defn}

In the following we will keep the notations $M_i$ and $m_i$ with the above meanings.  In case there are multiple functions we will write $M_i(f)$ and $m_i(g)$ which are self-evident.

Note that we require that $[a,b]$ is a bounded interval, because when it is infinite there will always be at least one subinterval determined by any partition having infinite length.
Also note that we require that $f$ is a bounded function, for otherwise there will be some $M_i$ or $m_i$ being infinite to make the upper sum or the lower sum no sense.
But we do not ask $f$ to be continuous.

Before we proceed to pass to the limit process, we compare upper sums and lower sums under different partitions for $[a,b]$.

\begin{defn}
  Let $\mathcal{P}$, $\mathcal{Q}$ be two partitions for a bounded closed interval $[a,b]$.
  We say that $\mathcal{P}$ is a \textsf{refinement} of $\mathcal{Q}$, $\mathcal{P}$ is \textsf{finer} than $\mathcal{Q}$, or equivalently $\mathcal{Q}$ is \textsf{coarser} than $\mathcal{P}$, when $\mathcal{P} \supseteq \mathcal{Q}$.

  The \textsf{(least) common refinement} of partitions $\mathcal{P}$ and $\mathcal{Q}$ for $[a,b]$ is their union $\mathcal{P} \cup \mathcal{Q}$ (properly ordered).
\end{defn}

Intuitively, a partition for a bounded closed intervals is a finite collection of division points; of course it has to contain both endpoints.
Hence a finer partition means more division points.
A finer partition can be obtained by adding a finite number of division points, one at a time.
And a common refinement is the partition that contains both partitions.

\begin{thm}
  \label{thm:refine}
  Let $f : [a,b] \to \mathbb{R}$ be a bounded function.
  For two partitions $\mathcal{P}$ and $\mathcal{Q}$ for $[a,b]$ with $\mathcal{P}$ finer than $\mathcal{Q}$, we have the following comparisons between upper and lower sums:
  \[
    L(f,\mathcal{Q}) \leqslant L(f,\mathcal{P}) \leqslant U(f,\mathcal{P}) \leqslant U(f,\mathcal{Q}).
  \]
\end{thm}

\begin{proof}
  The middle inequality is obvious since infimum is no greater than supremum and the length of each subinterval determined by a parition is always positive.
  By symmetry it suffices to show the last inequality $U(f,\mathcal{P}) \leqslant U(f,\mathcal{Q})$ when $P \supseteq \mathcal{Q}$; the first inequality follows from an analogous argument.
  Furthermore, we may assume that $\mathcal{P} = \mathcal{Q} \cup \{ c \}$ for some $c \in (a,b)$ and $c \notin \mathcal{Q}$.
  The general case follows from mathematical induction (on the difference of cardinalities of $\mathcal{P}$ and $\mathcal{Q}$).
  
  Let $\mathcal{Q} = \{ a = x_0 < x_1 < \cdots < x_n = b \}$ and $c \in (x_{i-1}, x_i)$.
  Then $\mathcal{P} = \{ a = y_0 < y_1 < \cdots < y_{n+1} = b \}$ where
  \[
    y_j =
    \begin{cases}
      x_j, & \text{if $j < i$}, \\
      c,   & \text{if $j = i$}, \\
      x_{j-1}, & \text{if $j > i$}.
    \end{cases}
  \]

  Define 
  \[
    M = \sup \{ f(t) \colon t \in [x_{i-1}, x_i] \}
  \]
  and
  \[
    M_1 = \sup \{ f(t) \colon t \in [y_{i-1},c] \}, \qquad
    M_2 = \sup \{ f(t) \colon t \in [c, y_{i+1}] \}.
  \]
  Since $[y_{i-1},c] \cup [c, y_{i+1}] = [x_{i-1},x_i]$, we see that $\Delta y_i + \Delta y_{i+1} = \Delta x_i$ and $M \geqslant M_1$, $M \geqslant M_2$.
  Therefore
  \begin{align*}
    U(f,\mathcal{P}) - U(f,\mathcal{Q}) &= (M_1 \Delta y_i + M_2 \Delta y_{i+1}) - M \Delta x_i \\
    &= (M_1 - M) \Delta y_i + (M_2 - M) \Delta y_{i+1} \leqslant 0.
  \end{align*}
  Therefore $U(f,\mathcal{P}) \leqslant U(f,\mathcal{Q})$. 
\end{proof}

\begin{cor}
  Let $f : [a,b] \to \mathbb{R}$ be a bounded function, and $\mathcal{P}$, $\mathcal{Q}$ be two partitions for $[a,b]$.
  Then
  \[
    L(f,\mathcal{P}) \leqslant U(f,\mathcal{Q}).
  \]
\end{cor}

\begin{proof}
  Let $\mathcal{P} \cup \mathcal{Q}$ denote the common refinement of $\mathcal{P}$ and $\mathcal{Q}$.  Then by Theorem~4,
  \[
    L(f,\mathcal{P}) \leqslant L(f, \mathcal{P} \cup \mathcal{Q}) \leqslant U(f, \mathcal{P} \cup \mathcal{Q}) \leqslant U(f, \mathcal{Q}).
  \]
\end{proof}

From Theorem~4 we learn that the upper sums decrease but the lower sums increase as we keep adding more division points to partitions.
By Corollary~5 every lower sum is a lower bound for upper sums, while every upper sum is an upper bound for lower sums.
There is also a trivial partition $\{ a < b \}$ for $[a,b]$.
These inspire the following definitions.

\begin{defn}
  Let $f : [a,b] \to \mathbb{R}$ be a bounded function.

  \begin{enumerate}[(i)]
    \item The \textsf{upper integral} of $f$ over $[a,b]$ is
      \[
	\overline{I} = \overline{\int_a^b} f = (U) \int_a^b f := 
	\inf \{ U(f,\mathcal{P}) \colon \mathcal{P} \text{ is a parition for $[a,b]$} \}.
      \]

    \item The \textsf{lower integral} of $f$ over $[a,b]$ is
      \[
	\underline{I} = \underline{\int_a^b} f = (L) \int_a^b f := 
	\sup \{ L(f,\mathcal{P}) \colon \mathcal{P} \text{ is a parition for $[a,b]$} \}.
      \]
  \end{enumerate}
\end{defn}

Hence there are upper integrals and lower integrals for any bouned functions over any bounded closed intervals.
Of course we have $\displaystyle (U) \int_a^b f \geqslant (L) \int_a^b f$. 
The real challenge is when they coincide.
That is the definition for (Darboux/Riemann) integrability.

\begin{defn}
  Let $f : [a,b] \to \mathbb{R}$ be a bounded function.
  $f$ is said to be \textsf{Riemann integrable} over $[a,b]$ if its upper integral and lower integral coincide, that is,
  \[
    (U) \int_a^b f = (L) \int_a^b f.
  \]
  The common value is called the \textsf{(definite) integral} of $f$ over $[a,b]$ and is denoted by
  \[
    \int_a^b f = \int_a^b f(x) \, \dd x.
  \]

  The class of all Riemann integrable functions over $[a,b]$ is denoted by $\mathcal{R}([a,b])$, or simply $\mathcal{R}$ if the interval $[a,b]$ is clear from the content.
\end{defn}

It is not hard to see that constant functions are always integrable over any bounded closed intervals, and its integral equals the constant value of the function times the length of the bounded closed integral, which is exactly the (signed) area of the rectangular region under the graph of the constant function.
In order to prove that a larger class of functions are also integrable, we need an effective criterion to check the integrability.

\begin{thm}[Riemann's integrability criterion]
  \label{thm:riemann-integrable-cauchy-criterion}
  A bounded function $f : [a,b] \to \mathbb{R}$ is Riemann integrable if and only if for any $\varepsilon > 0$ there is a partition $\mathcal{P}$ for $[a,b]$ such that
  \begin{equation}
    \label{eq:riemann-cauchy-criterion}
    U(f,\mathcal{P}) - L(f,\mathcal{P}) < \varepsilon.
  \end{equation}
\end{thm}

\begin{proof}
  ($\Longrightarrow$) 
  Denote the integral by $I = \overline{\int_a^b} f = \underline{\int_a^b} f$.
  Let $\varepsilon > 0$ be given. 
  Since the upper integral is the infimum of the upper sums, there is a partition $\mathcal{P}_1$ for $[a,b]$ such that
  \[
    U(f,\mathcal{P}_1) < I + \frac{\varepsilon}{2}.
  \]
  On the other hand, since the lower integral is the supremum of the lower sums, there is a partition $\mathcal{P}_2$ for $[a,b]$ such that
  \[
    L(f,\mathcal{P}_2) > I - \frac{\varepsilon}{2}.
  \]
  Take $\mathcal{P} = \mathcal{P}_1 \cup \mathcal{P}_2$ to be their common refinement.  By Theorem~\ref{thm:refine} we see that
  \[
    U(f,\mathcal{P}) - L(f,\mathcal{P}) \leqslant U(f,\mathcal{P}_1) - L(f,\mathcal{P}_2) < \left( I + \frac{\varepsilon}{2} \right) - \left( I - \frac{\varepsilon}{2} \right) = \varepsilon,
  \]
  which is what is asked for.
  
  ($\Longleftarrow$) We know that for any $\varepsilon > 0$, there is a partition $\mathcal{P}$ for $[a,b]$ such that $U(f,\mathcal{P}) - L(f,\mathcal{P}) < \varepsilon$.
  Since $\overline{\int_a^b} f \leqslant U(f,\mathcal{P})$ and $\underline{\int_a^b} f \geqslant L(f,\mathcal{P})$, we have
  \[
    0 \leqslant \overline{\int_a^b} f - \underline{\int_a^b} f \leqslant U(f,\mathcal{P}) - L(f,\mathcal{P}) < \varepsilon.
  \]
  Since $\varepsilon$ is arbitrary, we conclude that $\overline{\int_a^b} f = \underline{\int_a^b} f$, i.e., $f$ is Riemann integrable over $[a,b]$ by definition.
\end{proof}

\noindent\textit{Remark.} Note that if (\ref{eq:riemann-cauchy-criterion}) holds for a partition $\mathcal{P}$ for $[a,b]$, then
\[
  U(f,\mathcal{P}') - L(f,\mathcal{P}') < \varepsilon
\]
for any partition $\mathcal{P}'$ for $[a,b]$ that is finer than $\mathcal{P}$.

Now we are able to prove the following theorem concerning integrability of continuous functions.

\begin{thm}
  Let $f: [a,b] \to \mathbb{R}$ be a continuous function.
  Then $f$ is Riemann integrable over $[a,b]$.
\end{thm}

\begin{proof}
  Since $f$ is continuous on $[a,b]$, it is bounded on $[a,b]$ (so it is eligible to talk about integrability on $[a,b]$ or not).
  Also this implies that $f$ is \textit{uniformly continuous} on $[a,b]$.
  Hence for any $\varepsilon > 0$ there is a $\delta > 0$ such that 
  \begin{equation}
    \label{eq:uni-cont}
    |f(x) - f(y)| < \frac{\varepsilon}{2(b-a)} \qquad \text{whenever $x,y \in [a,b]$ and $|x-y| < \delta$.}
  \end{equation}

  Let $\mathcal{P} = \{ a = x_0 < \cdots < x_n = b \}$ be a partition for $[a,b]$ such that $\| \Delta \mathcal{P} \| < \delta$. (For example, take an integer $N$ large enough so that $\dfrac{b-a}{N} < \delta$ and partition $[a,b]$ into $N$ subintervals of equal length.)
  Then by (\ref{eq:uni-cont}) we have $M_i - m_i \leqslant \varepsilon$ for any $i$.
  Hence we may estimate the difference of the upper and lower sums determined by $\mathcal{P}$ as
  \[
    U(f,\mathcal{P}) - L(f,\mathcal{P}) = \sum_{i=1}^n (M_i - m_i) \Delta x_i \leqslant \sum_{i=1}^n \frac{\varepsilon}{2(b-a)} \Delta x_i = \frac{\varepsilon}{2 (b-a)} \cdot (b-a) = \frac{\varepsilon}{2} < \varepsilon.
  \]
  Therefore $f$ is Riemann integrable over $[a,b]$ by Riemann's integrability criterion.
\end{proof}

With the definition of Riemann integrability, we can now prove a few properties for integrals.

\begin{thm}[Linearity for integrals]
  \label{thm:int-linearity}
  Let $f, g$ be two integrable functions over a bounded closed interval $[a,b]$, and $\lambda$ be a real constant.
  Then both $f+g$ and $\lambda f$ are integrable over $[a,b]$, with
  \begin{align*}
    \int_a^b (f+g) &= \int_a^b f + \int_a^b g, \\
    \int_a^b \lambda f &= \lambda \int_a^b f.
  \end{align*}
\end{thm}

\begin{proof}
  Let $\mathcal{P} = \{ a = x_0 < \cdots < x_n = b \}$ be a partition for $[a,b]$.  We use $M_i(f)$ and $m_i(f)$ to denote the supremum and the infimum of $f$ in the $i^{\text{th}}$ subinterval determined by $\mathcal{P}$, respectively.
  It is clear that
  \[
    M_i(f+g) \leqslant M_i(f) + M_i(g) \quad
    \text{and} \quad
    m_i(f+g) \geqslant m_i(f) + m_i(g)
  \]
  for any $i = 1, 2, \dots, n$.  Hence $U(f+g, \mathcal{P}) \leqslant U(f,\mathcal{P}) + U(g,\mathcal{P})$ and $L(f+g,\mathcal{P}) \geqslant L(f,\mathcal{P}) + L(g,\mathcal{P})$.
  
  Let us concentrate on the inequalities $U(f+g, \mathcal{P}) \leqslant U(f,\mathcal{P}) + U(g,\mathcal{P})$ now.
  Then $(U)\int_a^b (f+g) \leqslant U(f,\mathcal{P}) + U(g,\mathcal{P})$.
  For any $\varepsilon > 0$, there is a partition $\mathcal{P}_0$ for $[a,b]$ such that $U(f,\mathcal{P}_0) < (U)\int_a^b f + \frac{\varepsilon}{2}$ and $U(g,\mathcal{P}_0) < (U)\int_a^b g + \frac{\varepsilon}{2}$. Hence
  \[
    (U) \int_a^b (f + g) \leqslant U(f, \mathcal{P}_0) + U(g, \mathcal{P}_0) < (U) \int_a^b f + (U) \int_a^b g + \varepsilon.
  \]
  Since this holds for any $\varepsilon > 0$, we conclude that 
  \[
    (U) \int_a^b (f+g) \leqslant (U)\int_a^b f + (U)\int_a^b g
  \]
  by the $\varepsilon$-principle.
  An analogous inequality holds for the lower integrals.
  So we have
  \begin{equation}
    \label{eq:sum-integral}
  (L) \int_a^b f + (L) \int_a^b g \leqslant (L) \int_a^b (f+g) \leqslant 
(U) \int_a^b (f+g) \leqslant (U) \int_a^b f + (U) \int_a^b g.
  \end{equation}
  When both $f$ and $g$ are integrable over $[a,b]$, 
  the two ends of (\ref{eq:sum-integral}) meet.  
  Therefore all of the inequalities in (\ref{eq:sum-integral}) become equalities, which mean that $f+g$ is integrable over $[a,b]$ and $\int_a^b (f+g) = \int_a^b f + \int_a^b g$.

  We still use the same partition $\mathcal{P}$ as above.
  For $\lambda \geqslant 0$, we have
  \[
    M_i(\lambda f) = \lambda \cdot M_i(f) \quad
    \text{and} \quad
    m_i(\lambda f) = \lambda \cdot m_i(f);
  \]
  while for $\lambda < 0$,
  \[
    M_i(\lambda f) = \lambda \cdot m_i(f) \quad
    \text{and} \quad
    m_i(\lambda f) = \lambda \cdot M_i(f).
  \]
  It is then straightforward to conclude that in both cases $\lambda f$ is also integrable over $[a,b]$ with $\int_a^b \lambda f = \lambda \int_a^b f$.
\end{proof}

\begin{thm}
  Let $f : [a,b] \to \mathbb{R}$ be a bounded function.
  If $f$ is integrable over $[a,b]$, then $f$ is integrable over any closed subinterval $[c,d] \subseteq [a,b]$.
  Indeed, for any $c \in (a,b)$, we have
  \begin{equation}
    \label{eq:sum-interval}
    \int_a^c f + \int_c^b f = \int_a^b f.
  \end{equation}
\end{thm}

\begin{proof}
  Let $\varepsilon > 0$ be given.
  Since $f$ is integrable over $[a,b]$, there is a partition $\mathcal{P}$ for $[a,b]$ such that
  \[
    U(f,\mathcal{P}) - L(f,\mathcal{P}) < \varepsilon.
  \]
  Let $\mathcal{P}' = \mathcal{P} \cup \{c, d\}$.  Then
  \[
    U(f,\mathcal{P}') - L(f,\mathcal{P}') < \varepsilon
  \]
  as well.  Now define $\mathcal{P}_{[c,d]} := \mathcal{P}' \cap [c,d]$, i.e., taking those division points that lie in $[c,d]$.
  It is clear that
  \[
    U(f,\mathcal{P}_{[c,d]}) - L(f,\mathcal{P}_{[c,d]}) \leqslant U(f,\mathcal{P}') - L(f,\mathcal{P}') < \varepsilon.
  \]
  Since $\varepsilon$ is arbitrary, we conclude that $f$ is integrable over $[a,b]$ by Riemann's integrability criterion.

  To prove (\ref{eq:sum-interval}), we start with a partition $\mathcal{P} = \{ a = x_0 < \cdots < x_n = b \}$.  Let $\mathcal{Q} = \mathcal{P} \cup \{c\}$, and further define $\mathcal{Q}_{[a,c]} = \mathcal{Q} \cap [a,c]$ and $\mathcal{Q}_{[c,b]} = \mathcal{Q} \cap [c,b]$.
  Then we have
  \[
    U(f,\mathcal{Q}_{[a,c]}) + U(f,\mathcal{Q}_{[c,b]}) = U(f,\mathcal{Q}) \leqslant U(f,\mathcal{P}).
  \]
  Since the upper integral is the infimum for upper sums, we have
  \[
    (U) \int_a^c f + (U) \int_c^b f \leqslant U(f, \mathcal{P}).
  \]
  Since this is true for any partition $\mathcal{P}$ for $[a,b]$, we conclude that
  \[
    (U) \int_a^c f + (U) \int_c^b f \leqslant (U) \int_a^b f.
  \]
  Likewise, we have
  \[
    (L) \int_a^c f + (L) \int_c^b f \geqslant (L) \int_a^b f.
  \]
  
  When $f \in \mathcal{R}([a,b])$, the upper and lower integrals of $f$ over $[a,b]$ are the same.  Therefore (\ref{eq:sum-interval}) holds.
\end{proof}

Originally the identity (\ref{eq:sum-interval}) holds only for $a < c < b$ and $f \in \mathcal{R}([a,b])$. However, if we define
\[
\int_a^a f = 0 \qquad \text{and} \qquad \int_b^a f = - \int_a^b f
\]
for any integrable function $f$, the (\ref{eq:sum-interval}) holds regardless how $a,b,c$ are ordered, that is,
\[
  \tag{\ref{eq:sum-interval}}
    \int_a^c f + \int_c^b f = \int_a^b f
\]
for any integral function $f$ and real numbers $a,b$, and $c$.

The following comparison theorem is clear and we omit its proof.
\begin{thm}[Comparison theorem for integrals]
  Let $f, g$ be bounded functions on $[a,b]$.  If $f(x) \leqslant g(x)$ for each $x \in [a,b]$, then
  \[
    (L) \int_a^b f \leqslant (L) \int_a^b g \qquad \text{and} \qquad
    (U) \int_a^b f \leqslant (U) \int_a^b g.
  \]
  In particular, if $f \in \mathcal{R}([a,b])$ and $m \leqslant f(x) \leqslant M$ for each $x \in [a,b]$, then
  \[
    m(b-a) \leqslant \int_a^b f \leqslant M(b-a).
  \]
\end{thm}

\begin{thm}
  \label{thm:abs-integrable}
  If $f \in \mathcal{R}([a,b])$, so does $|f|$.
  Moreover, we have
  \[
    \left| \int_a^b f \right| \leqslant \int_a^b |f|.
  \]
\end{thm}

\begin{proof}
  Let $\mathcal{P} = \{ a = x_0 < \cdots < x_n = b \}$ be a partition for $[a,b]$.  Then
  \[
    M_i(|f|) - m_i(|f|) \leqslant M_i(f) - m_i(f)
  \]
  for each $i = 1, 2, \dots, n$ by considering the signs of $f$ in each subinterval.  This implies that
  \[
    U(|f|,\mathcal{P}) - L(|f|,\mathcal{P}) \leqslant U(f,\mathcal{P}) - L(f,\mathcal{P}).
  \]
  Therefore $|f|$ is integrable over $[a,b]$ whenever $f$ is.

  Since $-|f| \leqslant f \leqslant |f|$ and they are all integrable, hence by the comparison theorem we see that
  \begin{equation}
    \label{eq:int-abs}
    - \int_a^b |f| \leqslant \int_a^b f \leqslant \int_a^b |f|.
  \end{equation}
  Because the outer terms in (\ref{eq:int-abs}) are opposite to each other, we obtain
  \[
    \left| \int_a^b f \right| \leqslant \int_a^b |f|.
  \]
\end{proof}

\begin{thm}
  Let $f$ and $g$ be two integrable functions over $[a,b]$.
  \begin{enumerate}[(i)]
    \item The square $f^2$ is integrable over $[a,b]$.

    \item The product $fg$ is integrable over $[a,b]$.
  \end{enumerate}
\end{thm}

\begin{proof}
  \begin{enumerate}[(i)]
    \item Since it is assumed that $f \in \mathcal{R}([a,b])$, it is also bounded on $[a,b]$.  Say $|f| \leqslant M$ for some $M > 0$ on $[a,b]$.
    Let $\mathcal{P} = \{ a = x_0 < \cdots < x_n = b \}$ be a partition for $[a,b]$. 
    For any $x,y \in I_i$, we have
    \[
      |(f(x))^2 - (f(y))^2| = \bigl| |f(x)|^2 - |f(y)|^2 \bigr| = (|f(x)|+|f(y)|) \bigl||f(x)|-|f(y)|\bigr| \leqslant 2M \bigl| |f(x)| - |f(y)|\bigr|.
    \]
    Hence
    \[
      M_i(f^2) - m_i(f^2) = M_i(|f|^2) - m_i(|f|^2) \leqslant 2 M ( M_i(|f|) - m_i(|f|) ).
    \]
    Summing over all $i$, we obtain
    \[
      U(f^2,\mathcal{P}) - L(f^2,\mathcal{P}) \leqslant 2M (U(|f|,\mathcal{P}) - L(|f|,\mathcal{P})).
    \]
    Since $\mathcal{P}$ is arbitrary, by Theorem~\ref{thm:abs-integrable}, $|f|$ is integrable over $[a,b]$, we conclude that $f^2$ is also integrable over $[a,b]$.

  \item This statement follows immediately from the identity
    \[
      fg = \frac{1}{4} \left( (f+g)^2 - (f-g)^2 \right),
    \]
    Theorem~\ref{thm:int-linearity} and (i).
\end{enumerate}
\end{proof}

\begin{thm}[Mean value theorem for integrals]
  Suppose that $f$ and $g$ are integrable over $[a,b]$ with $g \geqslant 0$.
  Set
  \[
    m = \inf \{ f(x) \colon x \in [a,b] \} \quad \text{and} \quad M = \sup \{ f(x) \colon x \in [a,b] \}.
  \]
  Then there is a number $c \in [m, M]$ such that
  \[
    \int_a^b f g = c \int_a^b g.
  \]
  Furthermore, if $f$ is continuous on $[a,b]$, then there exists a point $x_0 \in [a,b]$ such that
  \[
    \int_a^b f g = f(x_0) \int_a^b g.
  \]
\end{thm}

\begin{proof}
  By Theorem~5~(ii) we know that $fg$ is integrable over $[a,b]$.
  Since $g \geqslant 0$, by the comparison theorem we have
  \[
    m \int_a^b g \leqslant \int_a^b fg \leqslant M \int_a^b g. 
  \]
  If $\int_a^b g = 0$, then $c$ can be any number from $[m,M]$.
  Otherwise, set
  \[
    c = \frac{ \int_a^b fg }{\int_a^b g } \in [m,M].
  \]

  If $f$ is continuous on $[a,b]$, then there is a point $x_0 \in [a,b]$ such that $f(x_0) = c$ by the intermediate value theorem.
\end{proof}

\section{Riemann sums}
\label{sec:Riemann-sum}

Now we turn to another approach to definite integrals.
Let $f : [a,b] \to \mathbb{R}$ be a function (not necessarily bounded), and $\mathcal{P} = \{ a = x_0 < \cdots < x_n = b \}$ be a partition for $[a,b]$.
From each subinterval $I_i = [ x_{i-1}, x_i ]$ a \textit{sample point} $t_i \in [x_{i-1},x_i]$ is chosen; the collection of these sample points will be denoted by $\mathcal{T} = \{ t_1, t_2, \dots, t_n \}$.
Then the \textsf{Riemann sum} corresponding to $f, \mathcal{P}$, and $\mathcal{T}$ is defined by
\begin{equation*}
  R(f,\mathcal{P},\mathcal{T}) := \sum_{i=1}^n f(t_i) \Delta x_i = f(t_1) \Delta x_1 + \cdots + f(t_n) \Delta x_n.
\end{equation*}

It is clear that when $f$ is bounded on $[a,b]$, we have
\begin{equation}
  \label{eq:dar-rie-dar}
  L(f,\mathcal{P}) \leqslant R(f,\mathcal{P},\mathcal{T}) \leqslant U(f,\mathcal{P}),
\end{equation}
since $m_i \leqslant f(t_i) \leqslant M_i$ and $\Delta x_i > 0$ for all $i$.
Now we give the definition when the Riemann sum converges.

\begin{defn}
  A real number $I$ is the \textsf{Riemann integral} of $f$ over $[a,b]$ if for any $\varepsilon > 0$ there is a partition $\mathcal{P}_0$ for $[a,b]$ such that
  \[
    | R(f,\mathcal{P},\mathcal{T}) - I | < \varepsilon
  \]
  for any partition $\mathcal{P}$ for $[a,b]$ that is \textit{finer} than $\mathcal{P}_0$ and any collection $\mathcal{T}$ of sample points chosen from (subintervals determined by) $\mathcal{P}$.

  If such an $I$ exists it is unique, we denote it as
  \[
    I = \lim_{\|\Delta\mathcal{P}\| \to 0} R(f,\mathcal{P},\mathcal{T}) = \lim_{\| \Delta\mathcal{P} \| \to 0} \sum_{i=1}^n f(t_i) \Delta x_i,
  \]
  and we say that $f$ is \textsf{Riemann integrable} over $[a,b]$.
\end{defn}

Wait\dots, we have two definitions for Riemann integrability, one from Riemann and another from Darboux.
In fact they are equivalent, as the following theorem shows.

\begin{thm}
  Let $f: [a,b] \to \mathbb{R}$ be a real-valued function.  Then $f$ is Darboux integrable over $[a,b]$ if and only if $f$ is Riemann integrable over $[a,b]$.  Moreover, the Riemann integral coincides with the Darboux integral in the sense that
  \[
    \int_a^b f = I = \lim_{\| \Delta\mathcal{P} \| \to 0} R(f,\mathcal{P},\mathcal{T}).
  \]
\end{thm}

\begin{proof}
  ($\Longrightarrow$) Suppose that $f$ is Darboux integrable over $[a,b]$.
  For any $\varepsilon > 0$ there is a partition $\mathcal{P}_0$ for $[a,b]$ such that
  \[
    U(f,\mathcal{P}_0) - L(f,\mathcal{P}_0) < \varepsilon.
  \]
  Then for any collection $\mathcal{T}$ of sample points chosen from any partition $\mathcal{P}$ for $[a,b]$ that is finer than $\mathcal{P}$, we have by (1)
  \[
    R(f,\mathcal{P},\mathcal{T}), \int_a^b f \in [ L(f,\mathcal{P}), U(f,\mathcal{P}) ] \subseteq [ L(f,\mathcal{P}_0), U(f,\mathcal{P}_0)].
  \]
  Therefore $|R(f,\mathcal{P},\mathcal{T}) - \int_a^b f| \leqslant U(f,\mathcal{P}_0) - L(f,\mathcal{P}_0) < \varepsilon$.
  Since $\varepsilon$ is arbitrary, we have shown that $f$ is Riemann integrable over $[a,b]$, and $\int_a^b f$ indeed is the Riemann integral of $f$ over $[a,b]$.

  ($\Longleftarrow$) Conversely, suppose that $f$ is Riemann integrable over $[a,b]$.
  Firstly we note that $f$ has to be bounded on $[a,b]$.
  Let $I$ be the Riemann integral of $f$ over $[a,b]$.
  Take $\varepsilon = 1$ and let $\mathcal{P}_0$ be a partititon for $[a,b]$ such that $|R(f,\mathcal{P},\mathcal{T}) - I| < 1$ for any paritition $\mathcal{P}$ for $[a,b]$ that is finer than $\mathcal{P}_0$.
  If $f$ is not bounded on $[a,b]$, there is a subinterval $I_i$ determined by $\mathcal{P}_0$ such that $f$ is not bounded on $I_i$.  Then there are two points $t_i, t'_i \in I_i$ such that
  \[
    | f(t_i) \Delta x_i - f(t'_i) \Delta x_i | > 2.
  \]
  Let $\mathcal{T}$ be a collection of sample points from $\mathcal{P}_0$ that contains $t_i$ and $\mathcal{T}' = (\mathcal{T} \setminus \{ t_i \}) \cup \{ t'_i \}$.  Then
  \begin{align*}
    2 = 1 + 1 &> |R(f,\mathcal{P}_0,\mathcal{T}) - I| + |R(f,\mathcal{P}_0,\mathcal{T}') - I| \\
    &\geqslant |R(f,\mathcal{P}_0,\mathcal{T}) - R(f,\mathcal{P}_0,\mathcal{T}')| \\
    &= | f(t_i) \Delta x_i - f(t'_i) \Delta x_i | > 2,
  \end{align*}
  a contradiction!  Hence $f$ is bounded on $[a,b]$.

  To continue to show that $f$ is Darboux integrable over $[a,b]$, for any $\varepsilon > 0$ we fix a partition $\mathcal{P}$ for $[a,b]$ such that
  \[
    | R(f,\mathcal{P},\mathcal{T}) - I | < \frac{\varepsilon}{3}
  \]
  for any collection $\mathcal{T}$ of sample points from $\mathcal{P}$.
  Using the approximation property for supremum and infimum, we can choose sample points $t_i$ and $t'_i$ from each subinterval $I_i$ determied by $\mathcal{P}$ such that
  \[
    f(t_i) - f(t'_i) > M_i - m_i - \frac{\varepsilon}{3 (b-a)}.
  \]
  Therefore, if we denote $\mathcal{T} = \{ t_1, \dots, t_n \}$ and $\mathcal{T}' = \{ t'_1, \dots, t'_n \}$, we get
  \begin{align*}
    U(f,\mathcal{P}) - L(f,\mathcal{P}) &= \sum_{i=1}^n (M_i - m_i) \Delta x_i \\
    &< \sum_{i=1}^n \left( f(t_i) - f(t'_i) + \frac{\varepsilon}{3(b-a)} \right) \Delta x_i \\
    &= R(f,\mathcal{P},\mathcal{T}) - R(f,\mathcal{P},\mathcal{T}') + \frac{\varepsilon}{3(b-a)} \sum_{i=1}^n \Delta x_i \\
    &\leqslant | R(f,\mathcal{P},\mathcal{T}) - I | + | I - R(f,\mathcal{P},\mathcal{T}') | + \frac{\varepsilon}{3(b-a)} (b-a) \\ &< \frac{\varepsilon}{3} + \frac{\varepsilon}{3} + \frac{\varepsilon}{3} = \varepsilon.
  \end{align*}
  Since $\varepsilon$ is arbitrary, we have shown that $f$ is Darboux integrable over $[a,b]$.
\end{proof}

\section{Change of variable}
\label{sec:change-of-variable}

From now on Riemann integrability can be used exchangably with Darboux integrability.
We also note here that the approach from sample points and Riemann sums is more widely used in freshman calculus.

\begin{thm}[Change of variable]
  Let $\alpha$ be a differentiable, strictly increasing function on $[a,b]$ such that $\alpha' \in \mathcal{R}([a,b])$.  Let $c = \alpha(a)$ and $d = \alpha(b)$, and $f$ be an integrable function on $[c,d]$.
  Then
  \[
    \int_c^d f(u) \, \dd u = \int_a^b f(\alpha(x)) \alpha'(x) \, \dd x.
  \]
\end{thm}

\begin{proof}
  We note first that $\alpha$ is a bijection from $[a,b]$ onto $[c,d]$.
  Moreover, if $\mathcal{P} = \{ a = x_0 < \cdots < x_n = b \}$ is a partition for $[a,b]$, then $\mathcal{Q} = \{ c = y_0 < \cdots < y_n = d \}$ is a partition for $[c,d]$ and vice versa, where $y_i = \alpha(x_i)$ for each $i$.
  Also if $\mathcal{T} = \{ t_1, \dots, t_n \}$ is a collection of sample points from $\mathcal{P}$, then $\mathcal{U} = \{ u_1, \dots, u_n \}$ is a collection of sample points from $\mathcal{Q} = \alpha(\mathcal{P})$, where $u_i = \alpha(t_i)$ for each $i$.

  Suppose that $|f| \leqslant M$ for some $M > 0$.  Since $\alpha' \in \mathcal{R}([a,b])$, for any given $\varepsilon > 0$ there is partition $\mathcal{P}_0$ for $[a,b]$ such that for any $\mathcal{P} = \{ a = x_0 < \cdots < x_n = b \}$ for $[a,b]$ that is finer than $\mathcal{P}_0$ and any two collection of sample points $\mathcal{S}$ and $\mathcal{T}$ from $\mathcal{P}$ we have
  \[
    \sum_{i=1}^n |\alpha'(s_i) - \alpha'(t_i)| \Delta x_i < \frac{\varepsilon}{2M}.
  \]

  Let $\mathcal{P}, \mathcal{Q}, \mathcal{T}, \mathcal{U}$ be as above.
  Then we have the following Riemann sum
  \[
    R(f,\mathcal{Q},\mathcal{U}) = \sum_{i=1}^n f(u_i) \Delta y_i = \sum_{i=1}^n f(u_i) \alpha'(s_i) \Delta x_i
  \]
  for some points $s_i \in [x_{i-1},x_i]$, by the mean value theorem.
  On the other hand
  \[
    R( (f \circ \alpha) \cdot \alpha', \mathcal{P}, \mathcal{T}) = 
    \sum_{i=1}^n f(\alpha(t_i)) \alpha'(t_i) \Delta x_i = \sum_{i=1}^n f(u_i) \alpha'(t_i) \Delta x_i.
  \]
  Hence
  \[
    | R(f,\mathcal{Q},\mathcal{U}) - R( (f\circ\alpha)\cdot \alpha', \mathcal{P}, \mathcal{T}) | \leqslant \sum_{i=1}^n |f(u_i)| \, |\alpha'(s_i) - \alpha'(t_i)| \Delta x_i < M \cdot \frac{\varepsilon}{2M} = \frac{\varepsilon}{2}.
  \]

  Since $f$ is integrable over $[c,d]$, there is a partition $\mathcal{Q}_1$ for $[c,d]$ (which can be chosen finer than $\mathcal{Q}_0 = f(\mathcal{P}_0)$) such that
  \[
    | R( f, \mathcal{Q}, \mathcal{U} ) - \int_c^d f | < \frac{\varepsilon}{2}
  \]
  for any partition $\mathcal{Q}$ finer than $\mathcal{Q}_1$ and any collection $\mathcal{U}$ of sample points from $\mathcal{Q}$.  Therefore for any partition $\mathcal{P}$ for $[a,b]$ that is finer than $\mathcal{P}_1 = \alpha^{-1}(\mathcal{Q}_1)$ and any collection $\mathcal{T}$ of sample points from $\mathcal{P}$, we have
  \begin{align*}
    &| R( (f\circ \alpha) \cdot \alpha', \mathcal{P}, \mathcal{T} ) - \int_c^d f | \\
    \leqslant \,\, &| R( (f\circ \alpha) \cdot \alpha', \mathcal{P}, \mathcal{T}) - R( f, \mathcal{Q}, \mathcal{U}) | + | R(f,\mathcal{Q},\mathcal{U}) - \int_c^d f| \\ < \,\, &\frac{\varepsilon}{2} + \frac{\varepsilon}{2} =  \varepsilon.
  \end{align*}
  Since $\varepsilon$ is arbitrary, we conclude that $(f \circ \alpha) \cdot \alpha'$ is integrable over $[a,b]$ and
  \[
    \int_c^d f(u) \, \dd u = \int_a^b f(\alpha(x)) \alpha'(x) \, \dd x.
  \]
\end{proof}

\section{Fundamental theorem of calculus}
\label{sec:FTC}

It is impractical to compute definite integrals from Riemann/upper/lower sums every time.
So some computational techniques for integrals are called for.
Fortunately the fundamental theorem of calculus tells us that definite integrals can be evaluated from antiderivatives.
We start with the notion of indefinite integral.

\begin{defn}
  Let $f : [a,b] \to \mathbb{R}$ be an integrable function over $[a,b]$.
  For any $x \in [a,b]$, define
  \[
    F(x) = \int_a^x f.
  \]
  Then the function $F : [a,b] \to \mathbb{R}$ is called an \textsf{indefinite integral} of $f$.
\end{defn}

Unlike in the freshman calculus, an indefinite integral may not be differentiable.
But we can precisely tell when it is differentiable, as the following theorem says.

\begin{thm}[Fundamental theorem of calculus, part one]
  \label{thm:FTC-1}
  Let $f : [a,b] \to \mathbb{R}$ be a Riemann integrable function over $[a,b]$, and define
  \[
    F(x) = \int_a^x f, \qquad x \in [a,b],
  \]
  to be an indefinite integral of $f$.  Then $F$ is a continuous function on $[a,b]$.

  Furthermore, if $f$ is continuous at $x_0 \in [a,b]$, then $F$ is differentiable at $x_0 \in [a,b]$, with derivative $F'(x_0) = f(x_0)$.
\end{thm}

\begin{proof}
  Let us prove the continuity of $F$ first.
  Since $f$ is integrable over $[a,b]$, $f$ is bounded, say $|f| \leqslant M$ for some $M > 0$.
  Then for any $x,t \in [a,b]$, we have
  \[
    | F(t) - F(x) | = \left| \int_a^t f - \int_a^x f \right| = \left| \int_x^t f \right| \leqslant \left| \int_x^t M \right| = M |t-x|.
  \]
  Hence $F$ is (Lipschitz) continuous on $[a,b]$ (with Lipschitz constant $M$).

  Now suppose that $f$ is continuous at $x_0 \in [a,b]$.
  For any $\varepsilon > 0$, there is a $\delta > 0$ such that whenever $t \in [a,b]$ and $|t-x_0| < \delta$, we have $|f(t) - f(x_0)| < \varepsilon$. Then for any $t \in [a,b]$,
  \begin{align*}
    F(t) - F(x_0) - f(x_0) (t-x_0) &= \int_{x_0}^t f(u) \, \dd u - \int_{x_0}^t f(x_0) \, \dd u \\
    &= \int_{x_0}^t ( f(u) - f(x_0) ) \, \dd u.
  \end{align*}
  Therefore whenever $0 < |t-x_0| < \delta$,
  \begin{align*}
    \left| \frac{ F(t) - F(x_0) }{ t - x_0 } - f(x_0) \right| &\leqslant \frac{1}{|t-x_0|} \left| \int_{x_0}^t |f(u) - f(x_0)| \, \dd u \right| \\
    &\leqslant \frac{1}{|t-x_0|} \cdot \varepsilon \cdot |t-x_0| = \varepsilon.
  \end{align*}
  Since $\varepsilon$ is arbitrary, we conclude that
  \[
    \lim_{t \to x_0} \frac{ F(t) - F(x_0) }{ t - x_0 } = f(x_0),
  \]
  that is, $F$ is differentiable at $x_0 \in [a,b]$ with $F'(x_0) = f(x_0)$.
\end{proof}

\noindent\textit{Remark.} If $f$ is integrable over $[a,b]$ but not continuous at $x_0 \in [a,b]$, then an indefinite integral $F$ of $f$ may not be differentiable at $x_0$.
For example, consider $f: [0,2] \to \mathbb{R}$ defined as
\[
  f(x) = 
  \begin{cases}
    0, & \text{ if $x \in [0,1]$}; \\
    1, & \text{ if $x \in (1,2]$}.
  \end{cases}
\]
Then an indefinite integral $F(x) = \int_0^x f$ of $f$ over $[0,2]$ is given by
\[
  F(x) = 
  \begin{cases}
    0, & \text{ if $x \in [0,1]$}; \\
    x-1, & \text{ if $x \in (1,2]$}.
  \end{cases}
\]
It is clear that this $F$ is not differentiable at $1$.

\medskip
Another part of the fundamental theorem of calculus concerns the procedure of differentiation followed by integration.
We state this procedure as in the next theorem.

\begin{thm}[Fundamental theorem of calculus, part two]
  \label{thm:FTC-2}
  Let $f$ be a differentiable real function on $[a,b]$ whose derivative $f'$ is integrable over $[a,b]$.
  Then
  \[
    \int_a^b f' = f(b) - f(a).
  \]
\end{thm}

\begin{proof}
  Let $\mathcal{P} = \{ a = x_0 < \cdots < x_n = b \}$ be a partition for $[a,b]$.
  By the mean value theorem, in each subinterval $I_i$ determined by $\mathcal{P}$ there is a point $t_i$ such that
  \[
    f(x_i) - f(x_{i-1}) = f'(t_i) \Delta x_i.
  \]
  Let $\mathcal{T} = \{ t_1, \dots, t_n \}$ be the collection of these sample points from $\mathcal{P}$.
  Then the Riemann sum obtained from $f, \mathcal{P}$, and $\mathcal{T}$ is
  \begin{align*}
    R(f', \mathcal{P}, \mathcal{T}) &= \sum_{i=1}^n f'(t_i) \Delta x_i \\
    &= \sum_{i=1}^n \left( f(x_i) - f(x_{i-1}) \right) \\
    &= f(x_n) - f(x_0) = f(b) - f(a).
  \end{align*}
  That is, the value $f(b) - f(a)$ appears as a Riemann sum in {\em every} partition for $[a,b]$.
  Since $f'$ is integrable over $[a,b]$, $f(b) - f(a)$ must be the Riemann integral $\int_a^b f$ by uniqueness.
\end{proof}

Let $f$ be a real function on an interval.  An \textsf{antiderivative} of $f$ is a differentiable function $F$ on $I$ whose derivative is $f$, i.e., $F' = f$.
Theorem~\ref{thm:FTC-2} is the reason why we look for an antiderivative of the integrand when evaluating a definite integral; in this way no limiting process is needed.
On the other hand, it might not be possible to write down such an antiderivative, even for a simple-looking function such as $\sqrt{x^3 + x + 1}$.

Let us use the fundamental theorem of calculus to prove two useful tools in integration.
The first application is the formula of integration by parts, whose proof is straightforward after the hypothesis is met.

\begin{thm}[Integration by parts]
  Let $f$ and $g$ be two differentiable functions on $[a,b]$ such that both $f'$ and $g'$ are integrable on $[a,b]$.  Then
  \begin{equation}
    \label{eq:by-parts}
    \int_a^b f'g + \int_a^b fg' = f(b) g(b) - f(a) g(a).
  \end{equation}
\end{thm}

\noindent\textit{Remark.} Equation~(\ref{eq:by-parts}) is usually written as
\[
  \int_a^b f g' = f(b)g(b) - f(a)g(a) - \int_a^b f' g.
\]

\begin{proof}
  Let $h = f \cdot g$. $h$ is differentiable because it is a product of two differentiable functions.
  And $h' = f'g + fg'$ is integrable over $[a,b]$ since all of $f, f', g, g'$ are integrable.
  Hence (\ref{eq:by-parts}) follows from Theorem~\ref{thm:FTC-2}:
  \[
    \int_a^b (f'g + fg') = \int_a^b h' 
    = h(b) - h(a) 
    = f(b) g(b) - f(a) g(a).
  \]
\end{proof}

The second application is the formula for change of variable.
We have proven a general case when the integrand is only assumed to be integrable.
However, when the integrand is continuous, there is an easier proof using the fundamental theorem of calculus.

\begin{thm}[Change of variable, continuous case]
  If $\alpha$ is continuously differentiable on $[a,b]$, and $f$ is continuous on $\alpha([a,b])$, then
  \begin{equation}
    \label{eq:change-of-variable}
    \int_{\alpha(a)}^{\alpha(b)} f = \int_a^b f(\alpha(x)) \alpha'(x) \, \dd x.
  \end{equation}
\end{thm}

\begin{proof}
  Define the following indefinite integrals:
  \begin{align*}
    F(y) &= \int_{\alpha(a)}^y f, & & y \in \alpha([a,b]); \\
    G(x) &= \int_a^x f(\alpha(t)) \alpha'(t) \, \dd t, & & x \in [a,b].
  \end{align*}

  It follows from Theorem~\ref{thm:FTC-1} that for $x \in [a,b]$,
  \begin{align*}
    \frac{\dd}{\dd x} F(\alpha(x)) &= F'(\alpha(x)) \alpha'(x) = f(\alpha(x)) \alpha'(x); \\
    \frac{\dd}{\dd x} G(x) &= f(\alpha(x)) \alpha'(x).
  \end{align*}
  That is,
  \[
    \frac{\dd}{\dd x} (F(\alpha(x)) - G(x)) = 0, \qquad x \in [a,b].
  \]
  By the mean value theorem, $F(\alpha(x)) - G(x)$ takes a constant value $C$ on $[a,b]$.  By plugging in $x = a$ we see that 
  \[
    C = F(\alpha(a)) - G(a) = 0 - 0 = 0.
  \]
  Therefore $G(x) = F(\alpha(x))$ for all $x \in [a,b]$, which is Equation~(\ref{eq:change-of-variable}).
\end{proof}

\noindent\textit{Remark.} Equation~(\ref{eq:change-of-variable}) might take another form: when $I = [a,b]$ and $J = \alpha(I)$, Equation~(\ref{eq:change-of-variable}) may be written as
\[
  \int_J f(u) \, \dd u = \int_I f(\alpha(x)) |\alpha'(x)| \, \dd x.
\]
The absolute sign for $\alpha'$ takes care of both the cases $\alpha' \geqslant 0$ and $\alpha' \leqslant 0$.

\section{Integral form for Taylor remainder}
\label{sec:taylor-integral}

Let us briefly review the notion of Taylor polynomials.
Suppose a function $f : I \to \mathbb{R}$ has derivatives of sufficiently high orders in an open interval $I$.
Then near a point $a \in I$ $f$ can be approximated by a polynomial in the following form:
\begin{gather*}
  f(x) = \sum_{k=0}^r \frac{ f^{(k)}(a) }{ k! } (x - a)^k + R_r(x-a), \\
  \lim_{x \to a} \frac{ R_r(x-a) }{ (x-a)^r }  = 0.
\end{gather*}
That is, the remainder term $R_r(x-a)$ is $r^{\text{th}}$-order flat at $x = a$. 
The polynomial
\[
  P_r(x) = \sum_{k=0}^r \frac{ f^{(k)}(a) }{k!} (x-a)^k = f(a) + f'(a) (x-a) + \cdots + \frac{f^{(r)}(a)}{r!} (x-a)^r
\]
is called the $r^{\text{th}}$-order Taylor polynomial of $f$ at $a$.
We have seen that if $f$ has $(r+1)^{\text{st}}$-order derivative in $I$, then the remainder $R_r(h)$ assumes the {\em Lagrange form}:
\[
  R_r(h) = \frac{ f^{(r+1)}(\theta) }{(r+1)!} h^{r+1}
\]
for some $\theta$ between $a$ and $a+h$.

With integration by parts, we can prove an integral form for the remainder.
\begin{thm}
  Let $f \in \mathcal{C}^{r+1}(I)$ where $I$ is an open interval and $a \in I$.
  Then the $r^{\text{th}}$-order Taylor remainder of $f$ at $a$ can be written as
  \begin{equation}
    \label{eq:remainder-integral}
    R_r(h) = R_r(x-a) = \frac{1}{r!} \int_a^x (x-t)^{r} f^{(r+1)}(t) \, \dd t.
  \end{equation}
\end{thm}

\begin{proof}
  The proof is by induction on $r$.
  For $r = 0$, Equation~(\ref{eq:remainder-integral}) holds by the fundamental theorem of calculus:
  \[
    R_0(x-a) = f(x) - f(a) = \int_a^x f'(t) \, \dd t.
  \]
  
  Suppose Equation~(\ref{eq:remainder-integral}) holds for some $r \in \mathbb{N} \cup \{ 0 \}$.  Note that 
  \begin{equation}
    \label{eq:rr1}
    R_r(h) = \frac{ f^{(r+1)}(a) }{ (r+1)! } (x-a)^{r+1} + R_{r+1}(h), \qquad x = a + h.
  \end{equation}
  By the formula of integration by parts,
  \begin{align}
    R_r(h) &= \frac{1}{r!} \int_a^x (x-t)^r f^{(r+1)}(t) \, \dd t \notag \\
    &= \frac{-1}{(r+1)!} \int_a^x f^{(r+1)}(t) \, \dd (x-t)^{r+1} \notag \\
    &= \left. \frac{-f^{(r+1)}(t)}{(r+1)!} (x-t)^{r+1} \right|_a^x + \frac{1}{(r+1)!} \int_a^x (x-t)^{r+1} f^{(r+2)}(t) \, \dd t \notag \\
      &= \frac{ f^{(r+1)}(a) }{(r+1)!} (x-a)^{r+1} + \frac{1}{(r+1)!} \int_a^x (x-t)^{r+1} f^{(r+2)}(t) \, \dd t. \label{eq:rr2}
  \end{align}
  Comparison between (\ref{eq:rr1}) and (\ref{eq:rr2}) yields
  \[
    R_{r+1}(h) = \frac{1}{(r+1)!} \int_a^x (x-t)^{r+1} f^{(r+2)}(t) \, \dd t,
  \]
  that is, the formula holds for $r+1$ as well.
  Hence the proof is completed by mathematical induction.
\end{proof}

The integral form of the $r^{\text{th}}$-order remainder of the Taylor polynomial can also be used to give error estimate if we have bounds on $|f^{(r+1)}|$.
Working examples can be found somewhere else for interested readers.

  \section{Improper integrals}
\label{sec:improper-integrals}

To extend the Riemann integral to unbounded intervals or unbounded functions, we begin with an elementary observation.

If $f$ is Riemann integrable over $[a,b]$, then its indefinite integral
\[
  F(x) = \int_a^x f
\]
is continuous on $[a,b]$.  Therefore
\[
  \int_a^b f = F(b) = \lim_{x \to b-} F(x) = \lim_{x \to b-} \int_a^x f.
\]
The same can be said for the left-end point $a$.
This leads to the following generalization of the Riemann integral.

\begin{defn}
  Let $(a,b)$ be a nonempty, open (possibly unbounded) interval and $f: (a,b) \to \mathbb{R}$.
  \begin{enumerate}[(i)]
    \item $f$ is said to be \textsf{locally integrable} over $(a,b)$ if and only if $f$ is integrable over each bounded closed subinterval $[c,d] \subseteq (a,b)$.

    \item $f$ is said to be \textsf{improperly integrable} over $(a,b)$ if and only if $f$ is locally integrable over $(a,b)$ and 
      \begin{equation}
	\label{eq:improper-integral}
	\int_a^b f := \left( \lim_{c \to a+} \int_c^m f \right) + \left( \lim_{d \to b-} \int_m^d f \right)
      \end{equation}
      exists and is finite for any $m \in (a,b)$.  (When it does converge, the limit will not depend on $m$.)
      This limit is called the \textsf{improper (Riemann) integral} of $f$ over $(a,b)$; or equivalently we say that the improper integral $\int_a^b f$ converges.  We say that the improper integral $\int_a^b f$ diverges when the limit (\ref{eq:improper-integral}) diverges.
  \end{enumerate}
\end{defn}

\medskip
\noindent\textbf{Example.} Let $p$ be a positive real number.
The improper integral
\[
  \int_1^\infty \frac{1}{x^p} \, \dd x
\]
converges if and only if $p > 1$, while the improper integral
\[
  \int_0^1 \frac{1}{x^p} \, \dd x
\]
converges if and only if $0 < p < 1$. \hfill$\heartsuit$

Note from the definition that in order to talk about the convergence for two-sided improper integral such as $\int_{-\infty}^\infty f$, it is necessary to break it into two improper integrals like
\[
  \int_{-\infty}^\infty f = \int_{-\infty}^m f + \int_m^\infty f
\]
for any $m \in (-\infty, \infty)$.
The improper integral $\int_{-\infty}^\infty f$ converges only when both $\int_{-\infty}^m f$ and $\int_m^\infty f$ converge.
The verdict does not depend on the choice of $m \in (-\infty, \infty)$.

Because an improper integral is a limit  of Riemann integrals, many of the results we proved earlier have analogues for the improper integral,
such as the linearity property.
Next we state a useful theorem whose proof is clear and omitted here.

\begin{thm}[Comparison theorem for improper integrals]
  Suppose that $f$ and $g$ are locally integrable over $(a,b)$.
  If $0 \leqslant f(x) \leqslant g(x)$ for $x \in (a,b)$, and $g$ is improperly integrable over $(a,b)$, then $f$ is also improperly integrable over $(a,b)$, with
  \[
    \int_a^b f \leqslant \int_a^b g.
  \]
\end{thm}

\medskip
\noindent\textbf{Example.} Let $\alpha > 0$.  Since $\log x < x^{\alpha/2}$ for sufficiently large real $x$ (we will return to this statement when we formally define the logarithmic function,) hence
\[
  \frac{\log x}{x^{1+\alpha}} < \frac{1}{x^{1+(\alpha/2)}}
\]
for sufficiently large $x$ too.  This implies that $\dfrac{\log x}{x^{1+\alpha}}$ is integrable over $[1,\infty)$.

There are two more definitions relating to improper integrability that are similar to the cases for series.
\begin{defn}
  Let $(a,b)$ be a nonempty, open interval in $\mathbb{R}$ and $f: (a,b) \to \mathbb{R}$.
  \begin{enumerate}[(i)]
    \item $f$ is said to be \textsf{absolutely integrable} over $(a,b)$ if $f$ is locally integrable and $|f|$ is improperly integrable over $(a,b)$.
    \item $f$ is said to be \textsf{conditionally integrable} over $(a,b)$ if $f$ is improperly integrable over $(a,b)$ but $|f|$ is not improperly integrable over $(a,b)$.
  \end{enumerate}
\end{defn}

There is a similar result to the integrability about the function of absolute values of an integrable function.
Let us state the result here.

\begin{thm}
  If $f$ is absolutely integrable over $(a,b)$, then $f$ is improperly integrable over $(a,b)$ and
  \[
    \left| \int_a^b f \right| \leqslant \int_a^b |f|,
  \]
  where both sides are taken in the sense of improper integrals.
\end{thm}

\begin{proof}
  Use the inequalities
  \[
    0 \leqslant |f(x)| + f(x) \leqslant 2 |f(x)|
  \]
  over any bounded closed subinterval $[c,d]$ of $(a,b)$ and the usual comparison theorem to get
  \[
    0 \leqslant \int_c^d |f| + \int_c^d f \leqslant 2 \int_c^d |f|.
  \]
  Our result follows by passing to the limits $c \to a+$ and $d \to b-$.
\end{proof}

\noindent\textbf{Example.} The function $\dfrac{\sin x}{x}$ is conditionally integrable over $[1,\infty)$.

Let us first prove that $\dfrac{\sin x}{x}$ is improperly integrable over $[1,\infty)$.  Indeed, for any finite $d > 1$, by integration by parts we get
\[
  \int_1^d \frac{\sin x}{x} \, \dd x = \cos 1 - \frac{\cos d}{d} + \int_1^d \frac{\cos x}{x^2} \, \dd x.
\]
Since $\left| \dfrac{\cos x}{x^2} \right| \leqslant \dfrac{1}{x^2}$ for any $x \geqslant 1$ and the function $\dfrac{1}{x^2}$ is improperly integrable over $[1,\infty)$, we see that $\dfrac{\sin x}{x}$ is improperly integrable over $[1,\infty)$ as well.

On the other hand, 
\begin{align*}
  \int_1^\infty \left| \frac{\sin x}{x} \right| \, \dd x &\geqslant \lim_{n \to \infty} \int_\pi^{n\pi} \left| \frac{\sin x}{x} \right| \, \dd x \\
  &\geqslant \lim_{n \to \infty} \sum_{k=2}^n \frac{1}{k\pi} \int_{(k-1)\pi}^{k\pi} |\sin x| \, \dd x \\
  &= \sum_{k=2}^\infty \frac{ 2 }{ k \pi }.
\end{align*}
The last term is a harmonic series, which can be shown to diverge in the following classes.
Hence $\left| \dfrac{\sin x}{x} \right|$ is not improperly integrable over $[1,\infty)$.

  \section{Logarithmic function and exponential function}
\label{sec:log-exp}

We now come to rigorous definitions for logarithmic and exponential functions.
The reason that we adapt these definitions is that many properties in analysis of them will follow naturally and almost immediately.
The motivation is that the integral formula
\begin{equation}
  \label{eq:power-int}
  \int x^n \, \dd x = \frac{x^{n+1}}{n+1} + C
\end{equation}
holds for any integer $n \in \mathbb{Z}$ except for $n = -1$.  So what happens there?
It turns out that it gives rise a new function that has wider applications than expected.

\begin{defn}
  The \textsf{logarithmic function} for $x > 0$ is defined via the following definite integral:
\[
  L(x) = \int_1^x \frac{1}{t} \, \dd t, \qquad x > 0.
\]
\end{defn}

This definition surely fills in the gap occurred in (\ref{eq:power-int}).
The reason why $L$ is called a logarithmic function is that it carries key properties such as logarithmic laws, as we see in the next result.

\begin{thm}
  The logarithmic function $L$ defined on $(0,\infty)$ enjoys the following properties.
  \begin{enumerate}[(i)]
    \item $L(1) = 0$.
    \item For $x, y > 0$, we have $L(xy) = L(x) + L(y)$.
    \item $L(x^q) = q L(x)$ for any $x > 0$ and $q \in \mathbb{Q}$.
      Moreover, the range of $L$ is $\mathbb{R}$.
    \item $L$ is differentiable with $L'(x) = \dfrac{1}{x}$; hence $L$ is strictly increasing and concave in $(0,\infty)$.
    \item For any $\varepsilon > 0$, $\displaystyle \lim_{x \to \infty} \frac{L(x)}{x^{\varepsilon}} = 0$.
  \end{enumerate}
\end{thm}

\begin{proof}
  \begin{enumerate}[(i)]
    \item Trivial.

    \item Using change of variable $t = xu$, we see that
      \begin{align*}
	L(xy) &= \int_1^{xy} \frac{1}{t} \, \dd t = \int_1^x \frac{1}{t} \, \dd t + \int_x^{xy} \frac{1}{t} \, \dd t \\
	&= L(x) + \int_1^y \frac{1}{xu} \cdot x \, \dd u\\
	&= L(x) + \int_1^y \frac{1}{u} \, \dd u = L(x) + L(y).
      \end{align*}

    \item By (ii) and mathematical induction one sees that $L(x^q) = q L(x)$ for any $x > 0$ and $q \in \mathbb{Q}$.

      It is clear that $L(\alpha) > 0$ for any $\alpha > 1$.
      Because we have established that $L(\alpha^n) = n L(\alpha)$ for all $n \in \mathbb{Z}$, we see that
      \[
	\lim_{n\to\infty} L(\alpha^n) = +\infty \quad \text{and} \quad
	\lim_{n\to-\infty} L(\alpha^n) = -\infty.
      \]
      Since $L$ is continuous in $(0,\infty)$, the range of $L$ is the whole $\mathbb{R}$.
      
    \item Again by the fundamental theorem of calculus we have $L'(x) = \dfrac1x$.
      Hence by the mean value theorem and the second derivative test we get that $L$ is strictly increasing and concave in $(0,\infty)$.

    \item By (iii), $L(x) \to \infty$ as $x \to \infty$.  We may assume that $\varepsilon \in \mathbb{Q}$ by taking a smaller positive {\em rational} exponent $\varepsilon'$.
      Hence by l'Hospital's rule we obtain
      \begin{align*}
	\lim_{x\to\infty} \frac{L(x)}{x^\varepsilon} &\stackrel{\text{H}}{=} \lim_{x\to\infty} \frac{1/x}{\varepsilon x^{\varepsilon - 1}} = \lim_{x\to\infty} \frac{1}{\varepsilon x^{\varepsilon}} = 0.
      \end{align*}
  \end{enumerate}
\end{proof}

Statement (ii) above is the one that resembles the \textit{logarithmic law}: $\log_a xy = \log_a x + \log_a y$.  Now we have to make sure what the base is.  There is a natural candidate for that.

\begin{defn}
  \begin{enumerate}[(i)]
    \item The \textsf{base of natural logarithm} $\eu$ is the (unique) positive real number that satisfies $L(\eu) = 1$.
    \item The inverse function of $L$ is denoted by $E: \mathbb{R} \to (0,\infty)$, and is called the \textsf{exponential function}.
  \end{enumerate}
\end{defn}

Now we apply various theorems to get properties about the exponential function $E$.

\begin{thm}
  The exponential function $E$ defined on $\mathbb{R}$ enjoys the following properties.
  \begin{enumerate}[(i)]
    \item $E(0) = 1$.

    \item For $a, b \in \mathbb{R}$, we have $E(a+b) = E(a) \cdot E(b)$.

    \item $E(qa) = (E(a))^q$ for any $a \in \mathbb{R}$ and $q \in \mathbb{Q}$.  Moreover, the range of $E$ is $(0,\infty)$.

    \item $E$ is differentiable with $E'(x) = E(x)$; hence $E$ is strictly increasing and convex.

    \item For any $r > 0$, $\displaystyle \lim_{x \to \infty} \frac{x^r}{E(x)} = 0$.
  \end{enumerate}
\end{thm}

\begin{proof}
  These properties for the exponential function more or less correspond to those for the logarithmic function.
  All proofs are omitted except for (iv).
  Since $L$ is continuously differentiable with $L'(x) = 1/x \ne 0$ for any $x > 0$, we conclude that its inverse function $E = L^{-1}$ is also differentiable with
  \[
    E'(x) = \frac{1}{L'(E(x))} = \frac{1}{1/E(x)} = E(x), \qquad x \in \mathbb{R}.
  \]
\end{proof}

Let us tabulate the properties for the logarithmic function and the exponential function below\footnote{%
A logarithmic function with base $a$, $0 < a \ne 1$, is written as $\log_a$.
Here $\ln$ is exactly $\log_{\eu}$.
The notion ``$\ln$'' occurs more often in engineering.
``$\log$'' (without specifying its base) usually refers to $\log_{\eu}$ in higher mathematics, but refers to $\log_{10}$ in high schools (or in daily life).
}.

\begin{center}
\begin{tabular}{c||c|c}
  & Logarithmic fuction & Exponential function \\ \hline
  Usual notation & $L(x) = \ln x$ & $E(x) = \exp(x)$ \\ \hline
  & $L(1) = 0$ & $E(0) = 1$ \\ \hline
  & $L(xy) = L(x) + L(y)$ & $E(a+b) = E(a) \cdot E(b)$ \\ \hline
  Domain & $(0,\infty)$ & $\mathbb{R}$ \\ \hline
  Range  & $\mathbb{R}$ & $(0,\infty)$ \\ \hline
  Derivative & $L'(x) = 1/x$ & $E'(x) = E(x)$
\end{tabular}
\end{center}

With the exponential function defined, any exponential function with a general base and a power function with arbitrary exponent can be defined as follows.

\begin{defn}
  \begin{enumerate}[(i)]
    \item Let $a > 0$.  The exponential function with base $a$ is defined by
      \[
	a^x = E(x L(a)), \qquad x \in \mathbb{R}.
      \]

    \item A power function with real exponent $r \in \mathbb{R}$ is defined by
      \[
	x^r = E(r L(x)), \qquad x > 0.
      \]
  \end{enumerate}
\end{defn}

These functions are both differentiable.
Their derivatives can be computed via the chain rule.

\medskip
\noindent\textbf{Example.} The derivative of the exponent function $a^x$ with base $a$ is
\begin{equation}
  \label{eq:arb-exp}
  \frac{\dd}{\dd x} a^x = \frac{\dd}{\dd x} E(x L(a)) = E'(x L(a)) (x L(a))' = E(x L(a)) L(a) = a^x \, \ln x.
\end{equation}

While the derivative of the power function $x^r$ with real exponent $r$ is
\[
  \frac{\dd}{\dd x} x^r = \frac{\dd}{\dd x} E(r L(x)) = E'(r L(x)) \, r L'(x) = x^r \cdot \frac{r}{x} = r x^{r-1}.
\]
Note this formula is consistent with the derivative formula for power functions with rational exponents.

\medskip
Finally let us highlight the special properties for the base $\eu$ of natural logarithm.  Since $L(\eu) = 1$, we have $E(1) = \eu$.  Therefore
\[
  \exp(x) = E(x) = E(1 \cdot x) = (E(1))^x = \eu^x.
\] 
That is, the exponential function $E(x)$ is the exponential function with base $\eu$.  Plugging in $a = \eu$ in (\ref{eq:arb-exp}), we see that
\[
  (\eu^x)' = \eu^x \, L(\eu) = \eu^x \cdot 1 = \eu^x.
\]

Nevertheless, let us not confuse two sources of $\eu$.  They are
\begin{enumerate}[(i)]
  \item $\displaystyle \eu_1 = \lim_{n \to \infty} \left( 1 + \frac{1}{n} \right)^n$.

  \item $\eu_2$ is the unique positive number satisfying $\displaystyle \int_1^{\eu_2} \frac{1}{t} \, \dd t$.
\end{enumerate}
It is a nice result to show that $\eu_1 = \eu_2$.
We will prove this result after discussion on series.

\end{document}
