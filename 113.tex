\documentclass[11pt]{article}
\newcommand{\ddd}{May 9, 2024}
\input{24aac-macro}

\begin{document}
\begin{center}
  \textbf{Topic 13: Outer measure}
\end{center}

Throught this topic and beyond, definitions and theorems are stated in $\mathbb{R}^n$ but proved in $\mathbb{R}^2$.  Multidimensionality can complicate notations of proofs but never its logic.

How should we measure the length of a subset of a line?  If the set to be measured is simple, so is the answer.  For example, the length of the interval $(a,b)$ is $b-a$.
But what is the length of the set of rational numbers?  As is often the case in analysis we proceed by inequalities and limits.
In fact one might distinguish the fields of algebra and analysis solely according to their use of equalities versus inequalities.

\begin{defn}
  The \textsf{length} of an interval $I = (a,b) \subseteq \mathbb{R}$ is $b-a$.
  It is denoted by $|I|$.  The \textsf{Lebesgue outer measure} of a set $A \subseteq \mathbb{R}$ is
  \[
    m^* A = \inf \left\{ \sum_k |I_k| : \text{$\{ I_k \}$ is a covering of $A$ by open intervals} \right\}.
  \]
\end{defn}
Implicitly we assume that the covering is at most countable; the series $\sum_k |I_k|$ is its total length.  The outer measure of $A$ is the infimum of the total lengths of all possible coverings $\{I_k\}$ of $A$ by open intervals.  If every series $\sum_k |I_k|$ diverges then by definition $m^*A = \infty$.

Outer measure is defined for every $A \subseteq \mathbb{R}$.  It measures $A$ from outside.  A dual approach measures $A$ from inside, called \textsf{inner measure}, which we will discuss later.

Three properties of outer measures (the ``axioms of outer measure'') are easy to check.

\begin{thm}
  \begin{enumerate}[(1)]
    \item The outer measure of the empty set is zero, i.e., $m^* \varnothing = 0$.

    \item If $A \subseteq B$ then $m^*A \leqslant m^*B$ (monotonicity).

    \item If $A = \cup_{n=1}^\infty A_n$ then $m^*A \leqslant \sum_{n=1}^\infty m^*A_n$ (countable additivity).
  \end{enumerate} 
\end{thm}

\begin{proof}
  Only (3) needs explanation.  Given $\varepsilon > 0$, for each $n$ there exists a covering $\{I_{n,k}\}_k$ of $A_n$ such that
  \[
    \sum_{k=1}^\infty |I_{n,k}| < m^*A_n + \frac{\varepsilon}{2^n}.
  \]
  Clearly the total collection $\{ I_{n,k} \}_{n,k}$ covers $A$, and
  \[
    \sum_{n,k} |I_{n,k}| = \sum_{n=1}^\infty \sum_{k=1}^\infty |I_{n,k}| \leqslant \sum_{n=1}^\infty \left( m^*A_n + \frac{\varepsilon}{2^n} \right) \leqslant \sum_{n=1}^\infty m^*A_n + \varepsilon.
  \]
  Thus $m^*A$ cannot exceed $\sum_{n=1}^\infty m^*A_n + \varepsilon$.  Since $\varepsilon$ is arbitrary, the infimum $m^*A$ is at most $\sum_{n=1}^\infty m^*A_n$, as required.
\end{proof}

Next, suppose you have a set $A$ is the plane and you want to measure its area.  Here is the natural way to do it.

\begin{defn}
  \label{defn:area}
  The \textsf{area} of a rectangle $R = (a,b) \times (c,d)$ is $|R| := (b-a)(d-c)$ and the (planar) outer area of $A \subseteq \mathbb{R}^2$ is the infimum of the total area of countable coverings of $A$ by open rectangles $R_k$:
  \[
    m^* A = \inf \left\{ \sum_k |R_k| \colon \{ R_k \} \text{ covers $A$} \right\}.
  \]
\end{defn}
Note that Definition~\ref{defn:area} is so natural that it can easily be generalized to higher dimensions.  We omit the technicality here.

Now is the notion when the size of a set is ``small''.  Note that we have discussed this notion in the previous topic, and here is just a paraphrase.

\begin{defn}
  If $Z \subseteq \mathbb{R}^n$ has outer measure zero then it is called a \textsf{zero set}.
\end{defn}

\begin{prop}
  Every subset of a zero set is a zero set.  A countable union of zero sets is a zero set.  Each hyperplane $P_i(a) = \{ (x_1, \dots, x_n) \in \mathbb{R}^n \colon x_i = a \}$ is a zero set in $\mathbb{R}^n$.
\end{prop}

\begin{proof}
  The first statement follows from the monotonicity of outer measure.  The second statement can be proved using the $\varepsilon/2^n$-trick.

  We prove the last statement when $n = 2$.  The ``hyperplane'' $P_i(a)$ is the line $\{ x = a \}$ when $i=1$ and $\{ y = a \}$ when $i=2$.  Given $\varepsilon > 0$ we can cover the line $P_1(a)$ with rectangle $R_k = I_k \times J_k$ where
  \[
    I_k = \left( a - \frac{\varepsilon}{k 2^{k+2}}, a + \frac{\varepsilon}{k 2^{k+2}} \right), \qquad J_k = (-k, k).
  \]
  The total area of these rectangles is $\varepsilon$ so $P_1(a)$ is a zero set.
\end{proof}

The next theorem states a property of outer measure that seems obvious.

\begin{thm}
  The linear outer measure of $[a,b]$ is $b-a$; the planar outer measure $[a,b] \times [c,d]$ is $(b-a)(d-c)$; the $n$-dimensional outer measure of a closed cell is the product of its edge lengths.
\end{thm}

\begin{proof}[Proof for $n=1$]
  For each $\varepsilon > 0$ the open interval $(a-\varepsilon, b+\varepsilon)$ covers $[a,b]$.  Thus $m^*([a,b]) \leqslant (b+\varepsilon) - (a-\varepsilon) = b-a+2\varepsilon$.  By the $\varepsilon$-principle we get $m^*([a,b]) \leqslant b-a$.

  The harder part is to get the reverse inequality.  We must show that if $\{ I_i \}$ is a countable open covering of $[a,b]$ then $\sum_i |I_i| \geqslant b-a$.  Since $[a,b]$ is compact it suffices to prove this for finite open coverings $\{ I_1, \dots, I_N \}$, where $I_i = (a_i, b_i)$.  If $N=1$ then $(a_1, b_1) \supseteq [a,b]$ implies $a_1 < a \leqslant b < b_1$ so $b-a < b_1 - a_1 = |I_1|$.  This is the base case for induction.  

  Let us assume that for each open covering of a compact interval $[c,d]$ by $N$ open intervals $\{ J_j \}$ we have $d-c < \sum_{j=1}^N |J_j|$, and let $\{ I_i \}$ be a covering of $[a,b]$ by $N+1$ open intervals $I_i = (a_i, b_i)$.  We claim that $\sum_{i=1}^{N+1}|I_i| \geqslant b-a$.  One of the intervals contains $a$, say $I_1 = (a_1, b_1)$.  If $b_1 \geqslant b$ then $I_1 \supseteq [a,b)$ and again $a_1 < a \leqslant b \leqslant b_1$ implies that $\sum_{i=1}^{N+1} |I_i| \geqslant |I_1| = b_1 - a_1 > b-a$.  On the other hand, if $b_1 < b$ then
  \[
    [a,b] = [a, b_1) \cup [b_1, b]
  \]
  and $|I_1| > b_1 - a$.  The compact interval $[b_1, b]$ is now covered by the $N$ open intervals $I_2, \dots, I_{N+1}$.  By induction we have $\sum_{i=2}^{N+1} |I_i| > b - b_1$.  Thus
  \[
    \sum_{i=1}^{N+1} |I_i| = |I_1| + \sum_{i=2}^{N+1} |I_i| > (b_1-a) + (b-b_1) = b-a,
  \]
  which completes the induction and the proof.
\end{proof}

The preceding inductive proof does not carry over to rectangles, for a rectangle has no left to right order.  However, the following grid proof works for all $n$.

\begin{proof}[Grid proof for $n=2$]
  Let $R = [a,b] \times [c,d] \subseteq \mathbb{R}^2$.  It is simple to see that $m^* R \leqslant (b-a)(d-c)$.  To check the reverse inequality, consider any countable covering of $R$ by open rectangles $R_k$.  We must show that $\sum_k |R_k| \geqslant (b-a)(d-c)$.  Since $R$ is compact the covering has a positive Lebesgue number $\lambda$.  Partition $[a,b]$ and $[c,d]$ into subintervals $I_i = (a_i, b_i)$ and $J_j = (c_j, d_j)$ such that each rectangle $S_{ij} = I_i \times J_j$ has diameter less than $\lambda$.  Then $\sum_{i,j} |S_{ij}| = |R|$.  Also, the union of the rectangles $S_{ij}$ contained in any $R_k$ forms a smaller rectangle $R_k' \subseteq R_k$.  (Strictly speaking $R_k'$ includes edges of the $S_{ij}$ but they form a zero set.)  Thus
  \[
    \sum_{i, j: S_{ij} \subseteq R_k} |S_{ij}| \leqslant |R_k|.
  \]
  It follows that
  \[
    (b-a)(d-c) = |R| = \sum_{i,j} |S_{ij}| \leqslant \sum_k \sum_{i,j:S_{ij} \subseteq R_k} |S_{ij}| \leqslant \sum_k |R_k|,
  \]
  which completes the proof.
\end{proof}

\begin{cor}
  The formula $m^* I = b-a$, $m^* R = (b-a)(d-c)$, and $m^* B = \prod_k m^*(I_k)$ hold also for intervals, rectangles, and boxes that are open or partly open.  In particular, $m*I=|I|$, $m^*R = |R|$, and $m^*B=|B|$ for open intervals, rectangles, and boxes.
\end{cor}
\end{document}
