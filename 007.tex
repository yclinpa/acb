\chapter{The Metric Topology of Euclidean Spaces}
\label{chap:metric_of_rn}

\section{Metric space}
\label{sec:metric}

It was an advance in geometry and analysis in the early 20th century to extract the essential properties needed for theorems to hold.
That was the birth of \emph{topology}, given by the German mathematician Felix Hausdorff's work \emph{Elements of set theory} in 1914.
We will need one of its branches, namely the metric space structure, in order to study analysis in Euclidean spaces.
A metric space is a proper generalization for Euclidean spaces, as we shall see now.

\begin{defn}
  A \textsf{metric space} $M$ is a set $M$, whose elements are referred as \emph{points}, equipped with a binary function $d: M \times M \to \mathbb{R}$ satisfying the following three properties:
  \begin{enumerate}[(1)]
    \item (Positivity) $d(x,y) \geqslant 0$ for any $x,y \in M$; $d(x,y) = 0$ if and only if $x=y$.
    \item (Symmetry) $d(x,y) = d(y,x)$ for any $x,y\in M$.
    \item (Triangle inequality) $d(x,y) \leqslant d(x,z) + d(z,y)$ for any $x, y, z \in M$.
  \end{enumerate}
  The function $d$ is usually referred as a \textsf{distance function} or \textsf{metric} on $M$.
  A metric space is denoted by $(M,d)$, sometimes simply by $M$ when the metric $d$ is understood.
\end{defn}

The following are examples of metric spaces.
\begin{enumerate}[(a)]
  \item Euclidean spaces $\mathbb{R}^n$.  When we say \emph{Euclidean spaces}, it is understood that $\mathbb{R}$ is endowed with the Euclidean metric:
    \[
      d(x,y) = \sqrt{ \sum_{k=1}^n |x_k - y_k|^2 }, \qquad
      x = (x_1,\dots, x_n), y = (y_1, \dots, y_n) \in \mathbb{R}^n.
    \]

  \item The trivial metric.  Every nonempty set $X$ can be given a trivial (or discrete) metric, defined as
    \[
      d(x,y) = 
      \begin{cases}
	0, & \text{if $x=y$} \\
	1, & \text{if $x\ne y$}
      \end{cases}, \qquad x, y \in X.
    \]
    This metric only tells whether two points coincide or not.
  \item Let $p > 0$.  Define a binary function $d_p : \mathbb{R}^n \times \mathbb{R}^n \to \mathbb{R}$ by
    \[
      d_p(x,y) = \left( \sum_{k=1}^n |x_k-y_k|^p \right)^{1/p}.
    \]
    Note that $p=2$ gives the usual Euclidean metric.
    
    It can be shown that
    \begin{enumerate}[(i)]
      \item $d_p$ is a metric on $\mathbb{R}^n$ if and only if $p \geqslant 1$.
      \item As $p \to \infty$, $d_p$ goes to
	\[
	  d_\infty(x,y) = \max \{ |x_1-y_1|, |x_2-y_2|, \dots, |x_n-y_n| \}.
	\]
	$d_\infty$ is also a metric on $\mathbb{R}^n$.
    \end{enumerate}
\end{enumerate}

We have defined the concept of limit for sequences in $\mathbb{R}$; its analog is easy for any metric space: we simply replace the absolute value in $\mathbb{R}$ by the metric on the metric space $M$.
So let us rephrase the definition for limit again.

\begin{defn}
  Let $M$ be a metric space.  A sequence $\langle a_n \rangle$ in $M$ is said to \textsf{converge} to a limit $p \in M$ if for any $\varepsilon > 0$ there is an integer $N \in \mathbb{N}$ such that
  \[
    n \geqslant N \implies d(a_n, p) < \varepsilon.
  \]
  In this case we write $a_n \to p$ (as $n \to \infty$.)

  If there is no $p \in M$ with the above property, we say that the sequence $\langle a_n \rangle$ \textsf{diverges} in $M$.
\end{defn}

It is \emph{mutatis mutandis} to show the uniqueness for limit of sequences in metric spaces from that in $\mathbb{R}$.
Also the concept of Cauchy sequences can also be introduced in metric spaces.
We now discuss subsequences of a sequence.

\begin{defn}
  Let $\langle a_n \rangle$ be a sequence in a metric space $M$.
  A \textsf{subsequence} of $\langle a_n \rangle$ is a sequence of the form $\langle a_{\varphi(k)} \rangle_k$, where $\varphi$ is a strictly increasing function from $\mathbb{N}$ to itself.
\end{defn}

Roughly speaking, a subsequence of a sequence is a sequence obtained by picking out infinitely many terms from the original sequence, while preserving their orders of appearance without repeating.

\begin{thm}
  Every subsequence of a convergent sequence in a metric $M$ converges and it converges to the same limit as does the mother sequence.
\end{thm}

The proof of this theorem is easy and we leave it to the readers as an exercise.
Nevertheless, this theorem is usually applied in the contrapositive way.
Namely, for a sequence $\langle a_n \rangle$, if it has a divergent subsequence, or it has two convergent subsequences that converge to different limits, then the original sequence diverges.
Another common way to state this theorem is that limits are unaffected when we pass to a subsequence.

\section{The Topology on Euclidean Space}
\label{sec:topology}

In real analysis we extract the essential properties on which others develop.
Although there are more general notions about (point set) topology, here we mainly deal with those that arise from metric spaces.
We start with a few definitions (cf.\ Rudin, Chapter 2, Definition 2.18).

\begin{defn}
  Let $(M, d)$ be a metric space.
  All points and sets mentioned below are understood to be elements and subsets of $M$.
  \begin{enumerate}[(a)]
    \item A \textsf{neighborhood} of a point $p$ is a set $B_r(p)$ consisting of all points $q$ such that $d(p,q) < r$ ($r > 0$).
      $B_r(p)$ is also called an \textsf{open ball} with center $p$ and radius $r$.

    \item A point $p$ is a \textsf{limit point} of $S$ if \textit{every} neighborhood of $p$ contains a point $q \ne p$ such that $q \in S$.

    \item If $p \in S$ but $p$ is not a limit point of $S$, then $p$ is called an \textsf{isolated point} of $S$.

    \item $S$ is \textsf{closed} if every limit point of $S$ belongs to $S$.

    \item A point $p$ is an \textsf{interior point} of $S$ if there is a neighborhood $N$ of $p$ such that $N \subseteq S$.  The collection of all interior points of $S$ is called the \textsf{interior} of $S$, and is denoted by $\mathring{S}$ or $\operatorname{int} S$.

    \item $S$ is \textsf{open} if every point of $S$ is an interior point of $S$.

    \item $S$ is \textsf{clopen} if $S$ is both closed and open.

    \item The \textsf{complement} of $S$ is the subset $S^c$ of all points that do not belong to $S$.

    \item A point $q$ is an \textsf{exterior point} of $S$ if there is a neighborhood $U$ of $q$ such that $U \subseteq S^c$, i.e., $U \cap S = \varnothing$.

    \item A point $b$ is called a \textsf{boundary point} of $S$ if it is neither an interior point nor an exterior point of $S$.
      The collection of all boundary points of $S$ is called the \textsf{boundary} of $S$, and is denoted by $\partial S$.
  \end{enumerate}
\end{defn}

With all these definitions, the most prominent ones are probably neighborhoods and limit points.
Especially, a point $p$ is {\em not} a limit point of a subset $S \subseteq M$ if and only if there is an $r > 0$ such that $B_r(p) \cap S \subseteq S$.
Two points are made here.  Firstly some books regard a neighborhood of a point $p$ is simply an open set $U$ that contains the point $p$.  But in this case we can find a smaller open ball $B_r(p)$ that is contained in $U$. 
Secondly our textbook (authored by Pugh) define a limit of a set to be a limit point or an isolated point of that set.
Since most literature does not agree with Professor Pugh, we simply leave him alone.  Nevertheless you should be careful when reading Pugh's texts.  (This is not an attack against him.)

Now we come to several properties that follow from these definitions.
First we need to explain the notion of an ``open ball.''

\begin{prop}
  An open ball in a metric space $M$ is always an open set.
\end{prop}

\begin{proof}
  Let $B_r(p)$ be an open ball, where $p \in M$ and $r > 0$.
  We need to show that every point $x \in B_r(p)$ is an interior point of $B_r(p)$.
  Let $x \in B_r(p)$; by definition we have $d(x,p) < r$.
  Take $\rho = r - d(x,p) > 0$.  We claim that $B_\rho(x) \subseteq B_r(p)$.

  Let $y$ be an arbitrary point of $B_\rho(x)$; by definition we have $d(x,y) < \rho$.
  Then by the triangle inequality,
  \[
    d(y,p) \leqslant d(y,x) + d(x,p) < \rho + d(x,p) = (r - d(x,p)) + d(x,p) = r,
  \]
  that is, $y \in B_r(p)$.
  Since $y$ is arbitrary, we conclude that $B_\rho(x) \subseteq B_r(p)$.
\end{proof}

Since every point $x$ in an open set is its interior point, there is an open ball centered at $x$ that falls in the open set completely.
We use this property to show statements about unions and intersection of open sets.

\begin{prop}
  \label{open-union-intersection}
  Let $M$ be a metric space.
  \begin{enumerate}[$(a)$]
    \item Let $\{ U_\alpha \colon \alpha \in A \}$ be an arbitrary collection of open sets in $M$ ($A$ is just some index set.)
      Then their union $U = \displaystyle \bigcup_{\alpha \in A} U_\alpha$ is open in $M$.

    \item Let $U_1, U_2, \dots, U_N$ be a {\em finite} collection of open sets in $M$.
      Then their intersection $V = \displaystyle \bigcap_{i=1}^N U_i$ is open in $M$.
  \end{enumerate}
\end{prop}

\begin{proof}
  \begin{enumerate}[(a)]
    \item Let $p$ be an arbitrary point of $U$.  Then there is some index $\alpha \in A$ such that $p \in U_\alpha$.
      Since $U_\alpha$ is open, there is an $r > 0$ such that $B_r(p) \subseteq U_\alpha \subseteq U$.
      This shows that $p$ is also an interior point of $U$.
      Since $p$ is arbitrary, we conclude that $U$ is open in $M$.

    \item Let $q$ be an arbitrary point of $V$.  Then $q \in U_i$ for each $i = 1, 2, \dots, N$.
      For each $i$, there is an $r(i) > 0$ such that $B_{r(i)}(q) \subseteq U_i$.
      Taking $r = \min \{ r(1), r(2), \dots, r(N) \} > 0$, we see that $B_r(q) \subseteq B_{r(i)}(q) \subseteq U_i$ for any $i$.
      Hence $B_r(q) \subseteq V$, i.e., $q$ is an interior point of $V$.
      Since $q$ is arbitrary, we see that $V$ is open in $M$.
  \end{enumerate}
\end{proof}

Proposition~\ref{open-union-intersection} says that the collection of all open sets in $M$ is closed under arbitrary union and finite intersection.\footnote{The word {\em closed} has been abused.  There are so many occasions in mathematics in which we use this word to mean something like closure.  But really it takes a while to make necessary distinction about this word that appears in different places.  In this sentence the word ``closed'' already has a different meaning from a set being closed in a metric space.}
The situation will be totally different when we talk about closed sets.
But first let us introduce the following result.

\begin{thm}
  \label{thm:open-closed}
  Let $M$ be a metric space.
  A subset $U$ of $M$ is open if and only its complement $C = U^c$ is closed.
\end{thm}

\begin{proof}
  Let us assume first that $U$ is open.
  If a point $x \in U$ then there is a neighborhood $B_r(x)$ of $x$ that falls completely in $U$.  But this implies that $x$ cannot be a limit point of $C = U^c$, since there is no sequence in $C$ that can converge to $x$.  Hence any limit point of $C$ must belong to $C$, by definition $C$ is closed.

  Now suppose $C = U^c$ is closed and $x \in U$.
  Since $x \notin C$, $x$ is not a limit point of $C$.
  Therefore there is an $r > 0$ such that $B_r(x) \cap C = \varnothing$.
  But this is exactly $B_r(x) \subseteq U$.
  Since $x$ is arbitrary, $U$ must be an open set.
\end{proof}

The choice of the words ``open'' and ``closed'' was unfortunate.
Unlike a door which must be either open or closed, a subset of a metric space may be open, closed, neither, or both.
For example an interval of the form $(a,b]$ in $\mathbb{R}$ is neither open nor closed.
It is important to remember that, to establish a set being open, one should prove the closedness of its complement rather than itself.

Combining Proposition~\ref{open-union-intersection}, Theorem~\ref{thm:open-closed} and De Morgan's Law, we immediately reach the following result.

\begin{prop}
  \label{closed-union-intersection}
  Let $M$ be a metric space.
  \begin{enumerate}[$(a)$]
    \item Let $\{ C_\alpha \colon \alpha \in A \}$ be an arbitrary collection of closed sets in $M$ ($A$ is just some index set.)
      Then their intersection $C = \displaystyle \bigcap_{\alpha \in A} C_\alpha$ is closed in $M$.

    \item Let $C_1, C_2, \dots, C_N$ be a {\em finite} collection of closed sets in $M$.
      Then their union $D = \displaystyle \bigcup_{i=1}^N C_i$ is closed in $M$.
  \end{enumerate}
\end{prop}

The following proposition holds trivially.

\begin{prop}
  \label{empty-universal-clopen}
  In any metric space $M$, the empty set $\varnothing$ and the whole space $M$ are both clopen.
\end{prop}

Indeed there is a more general notion of {\em topology.}\footnote{A \textsf{topology} on a set $M$ is a collection $\mathcal T$ of subsets of $M$, whose elements are called {\em open sets}, that is closed under arbitrary union, finite intersection, and contains $\varnothing$ and $M$.  In this scenario ``open set'' is introduced earlier than ``open ball'', which is only available for metric space.  Then a {\em closed set} is a subset of $M$ whose complement is open, that is, belongs to $\mathcal T$.  Theorem~\ref{thm:open-closed} becomes a definition!}
Please refer to standard textbooks such as Munkres' {\em Topology, A First Course}.

From Definition~1(e), it is clear that the interior $\mathring{S}$ is the largest open subset that is contained in $S$, in the sense that whenever $U$ is an open set and $U \subseteq S$, $U \subseteq \mathring{S}$.
Also from Definition~1(i) the collection of all exterior points of $S$ forms an open set, by a similar argument. 
On the other hand, the union of a subset and the set of all its limit points also has some significant properties.

\begin{defn}
  Let $S$ be a subset of a metric space $M$.
  Denote by $S'$ the set of all limit points of $S$.
  The \textsf{closure} of $S$ is defined to be the union $S \cup S'$, and is denoted by $\overline{S}$.
\end{defn}

\begin{thm}
  Let $M$ be a metric space, and $S$ be a subset of $M$.
  The closure $\overline{S}$ of $S$ is a closed set.
  In fact, it is the smallest closed set in $M$ that contains $S$, in the sense that whenever $K$ is a closed set and $K \supseteq S$, $K \supseteq \overline{S}$ as well.
\end{thm}

\begin{proof}
  Let $p \notin \overline{S} = S \cup S'$.
  Since $p$ is neither a point in $S$ nor a limit point of $S$, there is an $r > 0$ such that $B_r(p) \cap S = \varnothing$, i.e., $p$ is an exterior point of $S$.
  Conversely, if a point $p$ is an exterior point of $S$, $p$ is neither a point in $S$ nor a limit point of $S$.
  Therefore $\overline{S}$ is a closed set since its complement is the open set consisting of all exterior points of $S$.

  Now assume that $K$ is a closed set and $K \supseteq S$; we need to show that $K \supseteq S'$ as well.
  Let $p$ be a limit point of $S$.
  By definition $p$ is also a limit point of $K$, since every neighborhood of $p$ contains a point $q \ne p$ and $q \in S \subseteq K$.
  Since $K$ is closed, $p \in K$, and the proof is completed.
\end{proof}

Let us state an important property about the closure of a set.

\begin{thm}
  Let $M$ be a metric space, and $S$ be a subset of $M$.
  A point $x \in \overline{S}$ if and only if there is a sequence $\langle x_n \rangle$ in $S$ that converges to $x$.
\end{thm}

\begin{proof}
  ($\Rightarrow$): Since $\overline{S} = S \cup S'$, there are two possible cases.
  \begin{enumerate}
    \item If $x \in S$, then we can simply take the constant sequence $\langle x \rangle$, which converges to $x$ trivially.

    \item Suppose $x \in S'$.  For each $n \in \mathbb{N}$, $B_{1/n}(x)$ is a neighborhood of $x$.  Since $x$ is a limit point of $S$, there must be a point $x_n \in B_{1/n}(x) \cap S$ with $x_n \ne x$.  It is now clear that the sequence $\langle x_n \rangle$ in $S$ does converge to $x$.
  \end{enumerate}

  ($\Leftarrow$) Conversely, let us assume that there is a sequence $\langle x_n \rangle$ in $S$ that converges to $x$ but $x \notin S$; we must show that $x \in S'$.  For any $r > 0$, $B_r(x)$ is a neighborhood of $x$.  Since $x_n \to x$ as $n \to \infty$, there is an integer $N \in \mathbb{N}$ such that whenever $n \geqslant N$ we have $d(x_n,x) < r$.  Moreover, none of these $x_N, x_{N+1}, \dots$ can be the point $x$ because they belong to $S$ but $x$ is assumed not to be there.  Therefore, we have found (infinitely many) points $x_N, x_{N+1}, \dots$ in $S$ and in the neighborhood $B_r(x)$, none of which is $x$.  This shows that $x$ is indeed a limit point of $S$, i.e., $x \in S'$.
\end{proof}

Finally we talk about some properties of boundary.
Given a subset $S \subseteq M$, the space is partitioned into three subsets: $\mathring{S}$, the subset of exterior points of $S$, and the boundary $\partial S$.  Since the former two sets are open, the last $\partial S$ is always a closed set.
Furthermore, the following statement holds straight from Definition~1(j).

\begin{prop}
  Let $S$ be a subset of a metric space $M$.
  A point $p$ lies in the boundary $\partial S$ of $S$ if and only if every neighborhood of $p$ contains a point in $S$ and another point not in $S$.
\end{prop}
