\section{Logarithmic function and exponential function}
\label{sec:log-exp}

We now come to rigorous definitions for logarithmic and exponential functions.
The reason that we adapt these definitions is that many properties in analysis of them will follow naturally and almost immediately.
The motivation is that the integral formula
\begin{equation}
  \label{eq:power-int}
  \int x^n \, \dd x = \frac{x^{n+1}}{n+1} + C
\end{equation}
holds for any integer $n \in \mathbb{Z}$ except for $n = -1$.  So what happens there?
It turns out that it gives rise a new function that has wider applications than expected.

\begin{defn}
  The \textsf{logarithmic function} for $x > 0$ is defined via the following definite integral:
\[
  L(x) = \int_1^x \frac{1}{t} \, \dd t, \qquad x > 0.
\]
\end{defn}

This definition surely fills in the gap occurred in (\ref{eq:power-int}).
The reason why $L$ is called a logarithmic function is that it carries key properties such as logarithmic laws, as we see in the next result.

\begin{thm}
  The logarithmic function $L$ defined on $(0,\infty)$ enjoys the following properties.
  \begin{enumerate}[(i)]
    \item $L(1) = 0$.
    \item For $x, y > 0$, we have $L(xy) = L(x) + L(y)$.
    \item $L(x^q) = q L(x)$ for any $x > 0$ and $q \in \mathbb{Q}$.
      Moreover, the range of $L$ is $\mathbb{R}$.
    \item $L$ is differentiable with $L'(x) = \dfrac{1}{x}$; hence $L$ is strictly increasing and concave in $(0,\infty)$.
    \item For any $\varepsilon > 0$, $\displaystyle \lim_{x \to \infty} \frac{L(x)}{x^{\varepsilon}} = 0$.
  \end{enumerate}
\end{thm}

\begin{proof}
  \begin{enumerate}[(i)]
    \item Trivial.

    \item Using change of variable $t = xu$, we see that
      \begin{align*}
	L(xy) &= \int_1^{xy} \frac{1}{t} \, \dd t = \int_1^x \frac{1}{t} \, \dd t + \int_x^{xy} \frac{1}{t} \, \dd t \\
	&= L(x) + \int_1^y \frac{1}{xu} \cdot x \, \dd u\\
	&= L(x) + \int_1^y \frac{1}{u} \, \dd u = L(x) + L(y).
      \end{align*}

    \item By (ii) and mathematical induction one sees that $L(x^q) = q L(x)$ for any $x > 0$ and $q \in \mathbb{Q}$.

      It is clear that $L(\alpha) > 0$ for any $\alpha > 1$.
      Because we have established that $L(\alpha^n) = n L(\alpha)$ for all $n \in \mathbb{Z}$, we see that
      \[
	\lim_{n\to\infty} L(\alpha^n) = +\infty \quad \text{and} \quad
	\lim_{n\to-\infty} L(\alpha^n) = -\infty.
      \]
      Since $L$ is continuous in $(0,\infty)$, the range of $L$ is the whole $\mathbb{R}$.
      
    \item Again by the fundamental theorem of calculus we have $L'(x) = \dfrac1x$.
      Hence by the mean value theorem and the second derivative test we get that $L$ is strictly increasing and concave in $(0,\infty)$.

    \item By (iii), $L(x) \to \infty$ as $x \to \infty$.  We may assume that $\varepsilon \in \mathbb{Q}$ by taking a smaller positive {\em rational} exponent $\varepsilon'$.
      Hence by l'Hospital's rule we obtain
      \begin{align*}
	\lim_{x\to\infty} \frac{L(x)}{x^\varepsilon} &\stackrel{\text{H}}{=} \lim_{x\to\infty} \frac{1/x}{\varepsilon x^{\varepsilon - 1}} = \lim_{x\to\infty} \frac{1}{\varepsilon x^{\varepsilon}} = 0.
      \end{align*}
  \end{enumerate}
\end{proof}

Statement (ii) above is the one that resembles the \textit{logarithmic law}: $\log_a xy = \log_a x + \log_a y$.  Now we have to make sure what the base is.  There is a natural candidate for that.

\begin{defn}
  \begin{enumerate}[(i)]
    \item The \textsf{base of natural logarithm} $\eu$ is the (unique) positive real number that satisfies $L(\eu) = 1$.
    \item The inverse function of $L$ is denoted by $E: \mathbb{R} \to (0,\infty)$, and is called the \textsf{exponential function}.
  \end{enumerate}
\end{defn}

Now we apply various theorems to get properties about the exponential function $E$.

\begin{thm}
  The exponential function $E$ defined on $\mathbb{R}$ enjoys the following properties.
  \begin{enumerate}[(i)]
    \item $E(0) = 1$.

    \item For $a, b \in \mathbb{R}$, we have $E(a+b) = E(a) \cdot E(b)$.

    \item $E(qa) = (E(a))^q$ for any $a \in \mathbb{R}$ and $q \in \mathbb{Q}$.  Moreover, the range of $E$ is $(0,\infty)$.

    \item $E$ is differentiable with $E'(x) = E(x)$; hence $E$ is strictly increasing and convex.

    \item For any $r > 0$, $\displaystyle \lim_{x \to \infty} \frac{x^r}{E(x)} = 0$.
  \end{enumerate}
\end{thm}

\begin{proof}
  These properties for the exponential function more or less correspond to those for the logarithmic function.
  All proofs are omitted except for (iv).
  Since $L$ is continuously differentiable with $L'(x) = 1/x \ne 0$ for any $x > 0$, we conclude that its inverse function $E = L^{-1}$ is also differentiable with
  \[
    E'(x) = \frac{1}{L'(E(x))} = \frac{1}{1/E(x)} = E(x), \qquad x \in \mathbb{R}.
  \]
\end{proof}

Let us tabulate the properties for the logarithmic function and the exponential function below\footnote{%
A logarithmic function with base $a$, $0 < a \ne 1$, is written as $\log_a$.
Here $\ln$ is exactly $\log_{\eu}$.
The notion ``$\ln$'' occurs more often in engineering.
``$\log$'' (without specifying its base) usually refers to $\log_{\eu}$ in higher mathematics, but refers to $\log_{10}$ in high schools (or in daily life).
}.

\begin{center}
\begin{tabular}{c||c|c}
  & Logarithmic fuction & Exponential function \\ \hline
  Usual notation & $L(x) = \ln x$ & $E(x) = \exp(x)$ \\ \hline
  & $L(1) = 0$ & $E(0) = 1$ \\ \hline
  & $L(xy) = L(x) + L(y)$ & $E(a+b) = E(a) \cdot E(b)$ \\ \hline
  Domain & $(0,\infty)$ & $\mathbb{R}$ \\ \hline
  Range  & $\mathbb{R}$ & $(0,\infty)$ \\ \hline
  Derivative & $L'(x) = 1/x$ & $E'(x) = E(x)$
\end{tabular}
\end{center}

With the exponential function defined, any exponential function with a general base and a power function with arbitrary exponent can be defined as follows.

\begin{defn}
  \begin{enumerate}[(i)]
    \item Let $a > 0$.  The exponential function with base $a$ is defined by
      \[
	a^x = E(x L(a)), \qquad x \in \mathbb{R}.
      \]

    \item A power function with real exponent $r \in \mathbb{R}$ is defined by
      \[
	x^r = E(r L(x)), \qquad x > 0.
      \]
  \end{enumerate}
\end{defn}

These functions are both differentiable.
Their derivatives can be computed via the chain rule.

\medskip
\noindent\textbf{Example.} The derivative of the exponent function $a^x$ with base $a$ is
\begin{equation}
  \label{eq:arb-exp}
  \frac{\dd}{\dd x} a^x = \frac{\dd}{\dd x} E(x L(a)) = E'(x L(a)) (x L(a))' = E(x L(a)) L(a) = a^x \, \ln x.
\end{equation}

While the derivative of the power function $x^r$ with real exponent $r$ is
\[
  \frac{\dd}{\dd x} x^r = \frac{\dd}{\dd x} E(r L(x)) = E'(r L(x)) \, r L'(x) = x^r \cdot \frac{r}{x} = r x^{r-1}.
\]
Note this formula is consistent with the derivative formula for power functions with rational exponents.

\medskip
Finally let us highlight the special properties for the base $\eu$ of natural logarithm.  Since $L(\eu) = 1$, we have $E(1) = \eu$.  Therefore
\[
  \exp(x) = E(x) = E(1 \cdot x) = (E(1))^x = \eu^x.
\] 
That is, the exponential function $E(x)$ is the exponential function with base $\eu$.  Plugging in $a = \eu$ in (\ref{eq:arb-exp}), we see that
\[
  (\eu^x)' = \eu^x \, L(\eu) = \eu^x \cdot 1 = \eu^x.
\]

Nevertheless, let us not confuse two sources of $\eu$.  They are
\begin{enumerate}[(i)]
  \item $\displaystyle \eu_1 = \lim_{n \to \infty} \left( 1 + \frac{1}{n} \right)^n$.

  \item $\eu_2$ is the unique positive number satisfying $\displaystyle \int_1^{\eu_2} \frac{1}{t} \, \dd t$.
\end{enumerate}
It is a nice result to show that $\eu_1 = \eu_2$.
We will prove this result after discussion on series.
