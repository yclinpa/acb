\section{Compactness}
\label{sec:covering-comp}

We now come to the definition of genuine compactness in terms of open sets.
Most people find this abstract.
Nevertheless it catches the most important feature of what analysis needs.
You may find that it is parallel to what we have developed in sequential compactness.

\begin{defn}
  A collection $\mathcal U$ of subsets of $M$ \textsf{covers} $A \subseteq M$ if $A$ is contained in the union of sets belongs to $\mathcal{U}$.
  The collection $\mathcal{U}$ is a \textsf{covering} of $A$.
  If $\mathcal{U}$ and $\mathcal{V}$ both cover $A$ and if $\mathcal{V} \subseteq \mathcal{U}$ in the sense that each set $V \in \mathcal{V}$ belongs also to $\mathcal{U}$ then we say that $\mathcal{U}$ \textsf{reduces} to $\mathcal{V}$, and that $\mathcal{V}$ is a \textsf{subcovering} of $\mathcal{U}$.
\end{defn}

\begin{defn}
  If all the sets in a covering $\mathcal{U}$ of $A$ are open then $\mathcal{U}$ is an \textsf{open covering} of $A$.
  If every open covering of $A$ reduces to a finite subcovering of $A$ then we say that $A$ is \textsf{(covering) compact}.\footnote{You will frequently find it said that an open covering of $A$ \textit{has} a finite subcovering.  ``Has'' means ``reduces to.''}
\end{defn}

The idea is that if $A$ is compact and $\mathcal{U}$ is an open covering of $A$ then just a finite number of the open sets are actually doing the work of covering $A$.
The rest are redundant.

The mere existence of a finite open covering of $A$ is trivial and utterly worthless.
{\em Every} set $A$ has such a covering, namely the single open set $M$.
Rather, for $A$ to be compact, each and every open covering of $A$ must reduce to a finite subcovering of $A$.
Deciding directly whether this is so is daunting. 
How could you hope to verify the finite reducibility of all open coverings of $A$? 
There are so many of them. 
For this reason we concentrated on sequential compactness; it is relatively easy to check by inspection whether every sequence in a set has a convergent subsequence.

To check that a set is not compact it suffices to find an open covering which fails to reduce to a finite subcovering. 
Occasionally this is simple.
For example, the set $(0,1)$ is not compact in $\mathbb{R}$ because its covering
\[
  \mathcal{U} = \{ (1/n, 1) \colon n \in \mathbb{N} \}
\]
fails to reduce to a finite subcovering.

Indeed, the Heine-Borel theorem can be proved directly from the topological properties of compact sets.
This treatment can be found in many books about real analysis and point set topology.
Here we prove the following equivalence and the Heine-Borel theorem follows immediately.

\begin{thm}
  \label{thm:cpt}
  For a subset $A$ of a metric space $M$ the following are equivalent.
  \begin{enumerate}[(i)]
    \item $A$ is (covering) compact.
    \item $A$ is sequentially compact.
  \end{enumerate}
\end{thm}

\begin{proof}[Proof that (i) $\Rightarrow$ (ii).]
  We assume that $A$ is compact.
  If $A$ is not sequentially compact, then there is a sequence $\langle p_n \rangle$ of $A$ that does not have any subsequence that converges in $A$.
  If this is so, then for any point $a \in A$ there is some neighborhood $U_a$ of $a$ which contains only a finite number of terms of $\langle p_n \rangle$.
  Clearly $\{ U_a \colon a \in A \}$ is an open covering of $A$.
  Because $A$ is compact, there are a finite number of points $a_1, a_2, \dots, a_N \in A$ such that the finite collection $\mathcal{V} = \{ U_{a_1}, \dots, U_{a_N} \}$ also covers $A$.
  However, the collection $\mathcal{V}$ can contain only a finite number of terms of $\langle p_n \rangle$, which is absurd.
  Hence such a sequence $\langle p_n \rangle$ cannot exist in $A$, and $A$ is sequentially compact.
\end{proof}

Before we present the reverse implication, we need a notion called a Lebesgue number which is defined as follows.

\begin{defn}
  Let $\mathcal{U}$ be a covering of a subset $A$ in a metric space $M$.
  A \textsf{Lebesgue number} for the covering $\mathcal{U}$ of $A$ is a positive real number $\lambda$ such that for each $a \in A$ there is some $U \in \mathcal{U}$ with $B_\lambda(a) \subseteq U$.
\end{defn}

How do we interpret the above definition?
When $\mathcal{U}$ is an open covering of $A$, of course for each $a \in A$ there is a positive number $\lambda(a) > 0$ such that $B_{\lambda(a)}(a)$ lies in some member of $\mathcal{U}$.
But $\lambda(a)$ depends on $a$.
A Lebesgue number $\lambda$ for a covering $\mathcal{U}$ of $A$ can be thought a \textit{uniform} radius for every point in $A$ that produces an open neighborhood that lies in a single member of $\mathcal{U}$.
If $\lambda > 0$ is a Lebesgue number for a covering $\mathcal{U}$ of $A$, obviously any number $\lambda'$ with $0 < \lambda' \leqslant \lambda$ is also a Lebesgue number for $\mathcal{U}$ of $A$.

\begin{lem}
  \label{lem:lebesgue}
  Every open covering of a sequentially compact set has a Lebesgue number.
\end{lem}

\begin{proof}
  Suppose not: $\mathcal{U}$ is an open covering of a sequentially compact set $A$, and yet for each $\lambda > 0$ there exists an $a \in A$ such that no $U \in \mathcal{U}$ contains $B_\lambda(a)$.
  Take $\lambda = 1/n$ and let $a_n \in A$ be a point such that no $U \in \mathcal{U}$ contains $B_{1/n}(a_n)$.
  By sequentially compactness, there is a subsequence $\langle a_{n_k} \rangle$ of $\langle a_n \rangle$ that converges to some point $p \in A$.
  Since $\mathcal{U}$ is an open covering of $A$, there are an $r > 0$ and a $U \in \mathcal{U}$ such that $B_r(p) \subseteq U$.
  When $k$ is large enough, we have $d(a_{n_k}, p) < r/2$ and $1/n_k < r/2$.
  So if $d(x, a_{n_k}) < 1/n_k$, then
  \[
    d(x,p) \leqslant d(x,a_{n_k}) + d(a_{n_k},p) < \frac{1}{n_k} + \frac{r}{2} < \frac{r}{2} + \frac{r}{2} = r,
  \]
  that is, $B_{1/n_k}(a_{n_k}) \subseteq B_r(p) \subseteq U$.
  This contradicts to the supposition that no $U \in \mathcal{U}$ contains $B_{1/n_k}(a_{n_k})$.
  We conclude that, after all, $\mathcal{U}$ does have a Lebesgue number.
\end{proof}

\begin{proof}[Proof that (ii) $\Rightarrow$ (i) in Theorem~\ref{thm:cpt}.]
  Let $\mathcal{U}$ be an open covering of the sequentially compact set $A$.
  By Lemma~\ref{lem:lebesgue}, $\mathcal{U}$ has a Lebesgue number $\lambda > 0$.

  Let us assume the contrary, i.e., $\mathcal{U}$ cannot reduce to any finite subcovering that covers $A$.
  Choose any $a_1 \in A$ and some $U_1 \in \mathcal{U}$ such that $B_\lambda(a_1) \subseteq U_1$.
  Since $U_1$ alone cannot cover $A$, there is a point $a_2 \in A \setminus U_1$.
  We can proceed this procedure inductively as follows.

  Let $n \in \mathbb{N}$ and $n \geqslant 2$, and suppose that $a_1, \dots, a_{n-1} \in A$ and $U_1, \dots, U_{n-1} \in \mathcal{U}$ have been constructed such that
  \[
    a_k \notin \bigcup_{i=1}^{k-1} U_i, \quad
    B_\lambda(a_k) \subseteq U_k, \qquad k = 1, 2, \dots, n-1.
  \]
  Since $\{ U_1, \dots, U_{n-1} \}$ cannot cover $A$, there is a point $a_n \in A$ outside this union.
  Therefore there is a $U_n \in \mathcal{U}$ such that $B_\lambda(a_n) \subseteq U_n$.
  Thus a sequence $\langle a_n \rangle$ in $A$ and a sequence $\langle U_n \rangle$ in $\mathcal{U}$ have been constructed.

  We now argue that they lead to a contradiction.
  Since $A$ is sequentially compact, there is a subsequence $\langle a_{n_k} \rangle$ that converges to some point $p \in A$.
  For large $k$ we have $d(a_{n_k},p) < \lambda$ and
  \[
    p \in B_\lambda(a_k) \subseteq U_{n_k}.
  \]
  However, $a_{n_\ell}$ all lie outside $U_{n_k}$ whenever $\ell > k$, which contradicts to $a_{n_k} \to p$ as $k \to \infty$.
  Therefore $\mathcal{U}$ can indeed be reduced to a finite subcovering, and $A$ is then compact.
\end{proof}

After Theorem~\ref{thm:cpt}, we can now use the word ``compact'' for both covering compact and sequentially compact.\footnote{Note that the equivalence between these two compactnesses requires a Lebesgue number, hence this equivalence holds only for metric spaces.
They might be inequivalent in topological spaces whose topology is not induced by any metric.}
In the end we present two results about compactness that can be proved without appealing to sequential compactness.

\begin{thm}
  Every compact subset in a metric space is bounded and closed.
\end{thm}

\begin{proof}
  Let $A$ be a compact subset of a metric space $M$.
  Fix a point $m \in M$.  Clearly $A$ is covered by the open covering
  \[
    \mathcal{U} = \{ B_n(m) \colon n \in \mathbb{N} \}.
  \]
  Since $A$ is compact, $\mathcal{U}$ reduces to a finite subcovering
  \[
    \{ B_{n_k}(m) \colon k = 1, 2, \dots, N \}, \quad \text{where $n_1, \dots, n_N \in \mathbb{N}$.}
  \]
  If we set $R = \max \{ n_1, n_2, \dots, n_k \}$, we see that $A \subseteq B_R(m)$, which shows that $A$ is bounded.

  Next we show that $A$ is closed; this is equivalent to that $A^c$ is open.
  Consider an arbitrary point $p \in A^c$.
  For each $a \in A$, we take $\varepsilon(a) = d(p,a)/2 > 0$ and observe that $B_{\varepsilon(a)}(p) \cap B_{\varepsilon(a)}(a) = \varnothing$.
  Now $\{ B_{\varepsilon(a)}(a) \colon a \in A \}$ is an open covering of $A$.
  Since $A$ is compact, that covering can be reduced to a finite subcovering, i.e., there are a finite number of points $a_1, a_2, \dots, a_N \in A$ such that
  \[
      A \subseteq \bigcup_{n=1}^N B_{\varepsilon_n}(a_n), \qquad \text{where } \varepsilon_n = \varepsilon(a_n).
  \]
  Then if we take $\varepsilon = \min \{ \varepsilon_1, \dots, \varepsilon_N \} > 0$, we see that
  \[
    B_{\varepsilon}(p) \cap A
    \subseteq \bigcap_{n=1}^N B_{\varepsilon_n}(p) \cap \bigcup_{n=1}^N B_{\varepsilon_n}(a_n) = \varnothing,
  \]
  i.e., $B_{\varepsilon}(p) \subseteq A^c$.
  This shows that $p$ is an interior point of $A^c$.
  Since $p$ is arbitrary, we conclude that $A^c$ is open, i.e., $A$ is closed.
\end{proof}

\begin{thm}
  Let $A$ be a closed subset of a compact space $M$.
  Then $A$ is also compact.
\end{thm}

\begin{proof}
  Let $\mathcal{U}$ be an open covering of $A$.
  Then $\mathcal{U}_M := \mathcal{U} \cup \{ M \setminus A\}$ becomes an open covering of $M$, since $M \setminus A$ is an open set.
  Since $M$ is compact, $\mathcal{U}_M$ reduces to a finite subcovering $\mathcal{V}_M$.
  Then $\mathcal{V}_M \setminus \{ M \setminus A \}$ is a finite subcovering of $A$ which can be reduced from $\mathcal{U}$.
  Therefore $A$ is compact.
\end{proof}
