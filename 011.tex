\section{Connectness}
\label{sec:conn}

We now consider the general notion of connectedness.
Let $A$ be a subset of a metric space $M$.
If $A$ is neither the empty set nor $M$ then $A$ is a \textsf{proper} subset of $M$.
Recall that if $A$ is both open and closed in $M$ it is said to be \textit{clopen}.
The complement of a clopen set is clopen.
The complement of a proper subset is proper.

\begin{defn}
  A \textsf{separation} of a metric space $M$ is a pair of disjoint open proper subsets of $M$ whose union is $M$.
  $M$ is \textsf{disconnected} if $M$ has a separation.
  $M$ is \textsf{connected} if $M$ has no separation.
\end{defn}

Equivalently, $M$ is disconnected if it has a proper clopen subset.
Connectedness of $M$ does not mean that $M$ is connected to something, but rather $M$ is one unbroken thing.

Another common criterion for a connected subset $E$ of a metric space $M$ is expressed in the following way: $E$ is connected if and only if there are no two disjoint open sets $U$ and $V$ in $M$ such that $E \subseteq U \cup V$.

We can characterize connected subsets of the reals $\mathbb{R}$.
But we need a formal definition of interval.

\begin{defn}
  A subset $I$ of $\mathbb{R}$ is called an \textsf{interval} if the following holds: for any $x, y \in I$ with $x < y$, $z \in I$ whenever $x < z < y$.
\end{defn}

Intuitively, if an interval contains two points, it also contains every point in-between.
Therefore an interval in $\mathbb{R}$ must be of the forms $(a,b), [a,b], [a,b)$, or $(a,b]$, depending on whether its supremum and infimum (which can be $\pm\infty$) belong to it or not.
In fact, intervals are the only connected subsets in $\mathbb{R}$.

\begin{thm}
  \label{thm:conn-R-interval}
  Let $E$ be a subset of $\mathbb{R}$.
  $E$ is connected if and only if $E$ is an interval.
\end{thm}

\begin{proof}
  ($\Longrightarrow$).
  If $E$ is not an interval, then there are three points $x < z < y$ such that $x,y \in E$ but $z \notin E$.
  Then $E \cap (-\infty, z)$ and $E \cap (z, \infty)$ form a separation of $E$, and hence $E$ is disconnected.

 \medskip
 \noindent ($\Longleftarrow$).
 Now we need to show that every nonempty interval $E$ in $\mathbb{R}$ is connected.  The empty set is always connected by definition, since it admits no separation, obviously.

 Suppose $E$ is not connected, i.e., there are disjoint open sets $U, V$ in $\mathbb{R}$ such that $E \subseteq U \cup V$ and both $E \cap U$ and $E \cap V$ are nonempty.
 Pick $x \in E \cap U$ and $y \in E \cap V$; by symmetry we may assume that $x < y$.
 Since $E$ is an interval, we have $[x,y] \subseteq E$.
 Consider the set
 \[
   X = \{ u \in [x,y] \colon u \in U \}.
 \]
 Obviously $x \in X \ne \varnothing$ and $X$ is bounded above by $y$.
 Therefore $z = \sup X$ exists by the least upper bound property.
 Since $z \in [x,y] \subseteq E$ as well, either $z \in U$ or $z \in V$.

 If $z \in U$ then $z$ cannot be $y$ since $y \notin U$.
 As $U$ is open, there is a $\delta > 0$ such that $z + \delta \in [x,y] \cap U$.  This contradicts to the fact that $z$ is an upper bound for $X$.
 On the other hand, if $z \in V$ then $z$ cannot be $x$ since $x \notin V$.
 As $V$ is open, there is $\delta' > 0$ such that $(z-\delta', z] \subseteq [x,y] \cap V$.  This implies that $X$ is also bounded above by $z - \delta'$, which is less than $z$; this contradicts to the fact that $z = \sup X$.

 Therefore the assumption that $E$ is disconnected is incorrect, that is, $E$ is in fact connected.
\end{proof}

Next we discuss the interplay between continuous mappings and connected sets.
Here is the main theorem.

\begin{thm}
  \label{thm:cont-conn}
  Let $M$ be connected.  If a mapping $f: M \to N$ is continuous and surjective, then $N$ is connected.  That is, the continuous image of a connected set is connected.
\end{thm}

\begin{proof}
  Let $A$ be a nonempty clopen subset of $N$.
  Then by the topological properties of continuous mappings, $f^{-1}(A)$ is again a clopen subset of $M$.  Since $f$ is onto and $A$ is nonempty, $f^{-1}(A)$ is nonempty as well.
  Since $M$ itself is connected, $f^{-1}(A)$ must be the whole space $M$.
  Therefore $A$ must be $N$ since $f$ is onto.
  This shows that $N$ cannot have any proper clopen subset, hence $N$ is connected.
\end{proof}

\begin{cor}
  If $M$ is connected and $M$ is homeomorphic to $N$ then $N$ is also connected.
  Connectedness is a topological property.
\end{cor}

Combining Theorems~\ref{thm:conn-R-interval} and \ref{thm:cont-conn}, we obtain a generalization of the intermediate value theorem.

\begin{cor}[Generalized intermediate value theorem]
  Every continuous real-valued function defined on a connected domain has the intermediate value property.
\end{cor}

\begin{proof}
  Let $M$ be a connected set and $f: M \to \mathbb{R}$ be a continuous function.
  By Theorem~\ref{thm:cont-conn} $f(M)$ is a connected subset of $\mathbb{R}$.
  By Theorem~\ref{thm:conn-R-interval} $f(M)$ is an interval.
  Therefore $f$ enjoys the intermediate value property.
\end{proof}

Now we see the very reason why the intermediate value theorem holds---the interval $[a,b]$ is connected!
This is different from the extreme value theorem, whose validity is based on the fact that $[a,b]$ is sequentially compact.

Connectedness properties give a good way to distinguish nonhomeomorphic sets.

\noindent\textbf{Example.} The union of two disjoint closed intervals is not homeomorphic to a single interval.
The former is disconnected while the latter is connected.

\medskip
\noindent\textbf{Example.} The closed interval $[a,b]$ is not homeomorphic to the unit circle $S^1 \subseteq \mathbb{R}^2$.
The former becomes disconnected when an interior point of it is removed, but the latter remains connected if any one of its points is removed.

\bigskip
Now we return to more properties about connected sets.

\begin{thm}
  The closure of a connected set is connected.
  More generally, if $S \subseteq M$ is connected and $S \subseteq T \subseteq \overline{S}$ then $T$ is connected.
\end{thm}

\begin{proof}
  Suppose there are two disjoint open sets $U, V \subseteq M$ such that $T \subseteq U \cup V$.
  Since $S \subseteq U \cup V$ as well, either $S \subseteq U$ or $S \subseteq V$; without loss of generality let us assume that $S \subseteq U$.
  Any point $p \in T \setminus S$ is a limit point of $S$.
  Therefore any neighborhood of $p$ contains a point $q \in S$ and $q \ne p$.
  Thus $p$ cannot belong to $V$, otherwise there would be a neighborhood of $p$ that lies completely in $V$, thus it has an empty intersection with $U$, contradiction.  Therefore $T \subseteq U$ and $T \cap V = \varnothing$.
  This shows that $T$ cannot have nontrivial separation, hence $T$ is connected.
\end{proof}

\begin{thm}
  The union of arbitrarily many connected sets that share a common point is connected.
\end{thm}

\begin{proof}
  Let $\{ C_\alpha \colon \alpha \in A \}$ be an arbitrary collection of connected sets such that $p \in \cap_{\alpha \in A} C_\alpha \ne \varnothing$.
  Write $C = \cup_{\alpha \in A} C_\alpha$.
  Suppose that there are two disjoint open sets $U, V$ such that $C \subseteq U \cup V$.
  Since $p \in C \subseteq U \cup V$, by symmetry we may assume that $p \in U$.
  Because for each $\alpha \in A$ we have $p \in C_\alpha \subseteq U \cup V$, $C_\alpha$ must also be a subset of $U$ because $C_\alpha$ is connected.
  Therefore $C \subseteq U$ and $C \cap V = \varnothing$.
  This shows that $C$ has no nontrivial separation and $C$ is connected.
\end{proof}

Lastly there is a similar but different notion of connectivity, which we define below.

\begin{defn}
  A \textsf{path} joining points $p$ to $q$ in a metric space $M$ is a continuous mapping $f: [a,b] \to M$ such that $f(a) = p$ and $f(b) = q$.
  If each pair of points in $M$ can be joined by a path that lies completely in $M$, then $M$ is called \textsf{path-connected}.
\end{defn}

\begin{thm}
  Path-connected sets are connected.
\end{thm}

\begin{proof}
  Let $M$ be a path-connected set.
  If $M$ is empty it is connected.
  So below we assume that $M$ is nonempty.

  If $M$ is not connected, then there are two disjoint clopen proper subsets $U, V$ of $M$ such that $M = U \cup V$.
  Pick a point $p \in U$ and $q \in V$.
  Since $M$ is path-connected there is a continuous mapping $f: [a,b] \to M$ such that $f(a) = p$ and $f(b) = q$.
  Therefore $f^{-1}(U)$ and $f^{-1}(V)$ form a separation of $[a,b]$ by two disjoint clopen proper subsets, which is a contradiction since $[a,b]$ is connected.
  Hence $M$ must be connected.
\end{proof}

Although any connected subset of $\mathbb{R}$ is path-connected, there are connected sets in $\mathbb{R}^2$ which are not path-connected.
Nevertheless, if we add one more condition these notions are the same.

\begin{thm}
  Every open connected subset of $\mathbb{R}^m$ is path-connected.
\end{thm}

\begin{proof}
  We first note here that any convex set in $\mathbb{R}^m$ is path-connected since any pair of points can be joined by a segment that lies totally in that convex set.
  Because an open ball is convex, it is also path-connected.

  Let $D$ be a nonempty, open, and connected subset of $\mathbb{R}^m$ (the empty set is path-connected by definition).
  Fix a point $p \in D$ and consider the following subsets of $D$:
  \[
    U = \{ u \in D \colon \text{ $u$ and $p$ can be joined by a path in $D$} \}, \quad V = D \setminus U.
  \]

  We argue that both $U$ and $V$ are open subsets of $D$ as follows.
  Let $u \in U$.  Since $u \in D$ and $D$ is open, there is an $r > 0$ such that $B_r(u) \subseteq D$.  Then every point $x$ in $B_r(u)$ can be joined to $p$ by a path from $p$ to $u$ followed by the radius from $u$ to $x$.
  Hence $B_r(u) \subseteq U$ as well.  This shows that $U$ is an open subset of $D$.

  On the other hand, suppose $v \in V$.  Again there is an $s > 0$ such that $B_s(v) \subseteq D$.
  No point in $B_s(v)$ can be joined to $p$ by a path in $D$ because otherwise $v \in U$.  Hence $B_s(v) \subseteq V$.
  This shows that $V$ is also an open subset of $D$.

  Now $D$ is written as a union of two disjoint open subsets $U, V$, and $U \ne \varnothing$.
  Since $D$ is connected we must have $V = \varnothing$, i.e. $U = D$.
  This says any point in $D$ can be joined to $p$ by a path in $D$, hence $D$ is path-connected.
\end{proof}
